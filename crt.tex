\chapter{The Chinese remainder theorem}
\epigraph[author={Brahmagupta (580CE--670CE)}, source={Brahma-Sphuta-Siddhanta (Brahma's Correct System)}]{An old woman goes to market and a horse steps on her basket and  crushes the eggs. The rider offers to pay for the damages and asks  her how many eggs she had brought. She does not remember the 
exact number, but when she had taken them out two at a time,  there was one egg left. The same happened when she picked them out three, four, five, and six at a time, but when she took them seven at a time, there were none left. What is the smallest number of eggs she could have had?}\SubIndex{Brahmagupta}\SubIndex{Brahma-Sphuta-Siddhanta}\SubIndex{Brahma's correct system}
\section{The Chinese remainder theorem}
Take a look at some numbers and their remainders modulo \(3\) and \(5\):
\[
\begin{array}{@{}rrr@{}}
\toprule
n & n \operatorname{mod} 3 & n \operatorname{mod} 5 \\ 
\cmidrule(r){1-1}\cmidrule(lr){2-2}\cmidrule(l){3-3}
0 & 0 & 0 \\
1 & 1 & 1 \\
2 & 2 & 2 \\
3 & 0 & 3 \\
4 & 1 & 4 \\
5 & 2 & 0 \\
6 & 0 & 1 \\
7 & 1 & 2 \\
8 & 2 & 3 \\
9 & 0 & 4 \\
10 & 1 & 0 \\
11 & 2 & 1 \\
12 & 0 & 2 \\
13 & 1 & 3 \\
14 & 2 & 4 \\
15 & 0 & 0 \\
\bottomrule
\end{array}
\]
Remainders modulo \(3\) repeat every \(3\), and remainders modulo \(5\) repeat every \(5\), but the pair of remainders modulo \(3\) and \(5\) together repeat every \(15\).
\begin{example}
How can you find an integer \(x\) so that 
\begin{align*}
&\congmod[3]{x}{1}, \\
&\congmod[4]{x}{2}, \\
&\congmod[7]{x}{1}?
\end{align*}
\end{example}
\begin{example}
A recipe: how to find an unknown integer \(x\) given only the knowledge of its remainders modulo various integers.
Suppose we know \(x\) has remainder \(r_1\) modulo \(m_1\), \(r_2\) modulo \(m_2\), \dots \(r_n\) modulo \(m_n\).
Let 
\[
m\defeq m_1 m_2 \dots m_n.
\]
For each \(i\), let
\[
u_i\defeq \frac{m}{m_i}=m_1 m_2 \dots m_{i-1} \hcancel{m_i} m_{i+1} m_{i+2} \dots m_n,
\]
dropping the \(m_i\) factor.
Each \(u_i\) has some reciprocal modulo \(m_i\), given as the remainder of some integer \(v_i\).
Let 
\[
x\defeq r_1 u_1 v_1 + r_2 u_2 v_2 + \dots + r_n u_n v_n.
\]
We can simplify this a little: add or subtract multiples of \(m\) until \(x\) is in the range \(0 \le x \le m-1\).
\end{example}
\begin{theorem}
Take some positive integers \(m_1, m_2, \dots, m_n\), coprime to one another.
The recipe above works: it finds an integer \(x\) with required remainder, and \(x\) is unique modulo 
\[
m_1 m_2 \dots m_n.
\]
\end{theorem}
\begin{proof}
If there are two choices for such an integer \(x\), then their difference vanishes modulo all of the \(m_i\), so is a multiple of every one of the \(m_i\), which are coprime, so is a multiple of their product \(m\).
So there is at most one solution \(x\), modulo \(m\).

Note that \(u_2\) has no factor of \(m_2\), but has a factor of \(m_1\).
So modulo \(m_1\), \(u_2=0\).
So modulo \(m_1\), \(r_2u_2v_2=0\).
The same holds for \(r_3u_3v_3\), and so on.
So modulo \(m_1\), all terms in \(x\) drop out except maybe \(r_1u_1v_1\).
All of the other \(m_j\) are coprime to \(m_1\), so their product is coprime to \(m_1\), i.e. \(u_1\) is coprime to \(m_1\), so has a reciprocal modulo \(m_1\): so \(v_i\) exists.
But then modulo \(m_1\), \(r_1u_1v_1=r_1\).
So modulo \(m_1\):
\[
\begin{array}{@{}l@{\,}c@{\,}l@{\,}l@{\,}l@{\,}l@{}}\\
x&=&r_1 u_1 v_1 &+ r_2 u_2 v_2 &+ \dots &+ r_n u_n v_n,\\
 &=&r_1         &+ 0           &+ \dots &+ 0,\\
 &=&r_1.        &              &  
\end{array}
\]
The same holds replacing \(m_1\) by \(m_2\), and so on.
So the recipe gives the required integer \(x\).
\end{proof}
\begin{example}
Let's find an integer \(x\) so that 
\begin{align*}
&\congmod[3]{x}{1}, \\
&\congmod[4]{x}{2}, \\
&\congmod[7]{x}{1}.
\end{align*}
So in this problem we have to work modulo \(\pr{m_1,m_2,m_3}=\pr{3,4,7}\), and get remainders \(\pr{r_1,r_2,r_3}=\pr{1,2,1}\).
First, no matter what the remainders, we have to work out the reciprocal mod each \(m_i\) of the product of all of the other \(m_j\).
So let's reduce these products down to their remainders:
\begin{align*}
u_1&=\hcancel{3} \cdot{} 4 \cdot 7 = 28,\\
u_2&= 3 \cdot \hcancel{4} \cdot{} 7 = 21,\\
u_3&= 3 \cdot 4 \cdot{} \hcancel{7} = 12.
\end{align*}
Reduce to remainders:
\begin{align*}
u_1&=9\cdot 3 + 1 \equiv 1 \pmod{3}, \\
u_2&=5\cdot 4 + 1 \equiv 1 \pmod{4}, \\
u_3&=1\cdot 7 + 5 \equiv 5 \pmod{7}.
\end{align*}
We need the reciprocals of these, which, to save ink, we just write down for you without writing out the calculations:
\begin{align*}
v_1&\equiv 1^{-1} \equiv 1 \pmod{3}, \\
v_2&\equiv 1^{-1} \equiv 1 \pmod{4}, \\
v_3&\equiv 5^{-1} \equiv 3 \pmod{7}.
\end{align*}
(You can easily check those.)
Compute out the products \(u_i v_i\):
\begin{align*}
u_1 v_1 &= 28 \cdot 1 = 28,\\
u_2 v_2 &= 21 \cdot 1 = 21,\\
u_3 v_3 &= 12 \cdot 3 = 36.
\end{align*}
It is only at this final stage that we use the remainders: add up remainder times product:
\begin{align*}
x
&=
r_1 u_1 v_1
+
r_2 u_2 v_2
+
r_3 u_3 v_3,
\\
&=
1 \cdot 28
+
2 \cdot 21 
+
1 \cdot 36,
\\
&=
106.
\end{align*}
We can now check to be sure:
\begin{align*}
&\congmod[3]{106}{1}, \\
&\congmod[4]{106}{2}, \\
&\congmod[7]{106}{1}.
\end{align*}
The Chinese remainder theorem tells us also that \(106\) is the unique solution modulo \(3 \cdot 4 \cdot 7=84\).
But then \(106-84=22\) is also a solution, the smallest positive solution.
\end{example}
\begin{problem}{modular.arithmetic:eggs}
Solve the problem about eggs. Hint: ignore the information about eggs taken out six at a time.
\end{problem}
\begin{problem}{modular.arithmetic:army}
Use the Chinese remainder theorem to determine the smallest number of soldiers possible in Han Xin's army if the following facts are true.
When they parade in rows of three soldiers, two soldiers will be left. When they parade in rows of five, three will be left, and in rows of seven, two will be left.
\end{problem}
\begin{problem}{modular.arithmetic:army.2}
Use the Chinese remainder theorem to determine the smallest number of soldiers possible in Han Xin's army if the following facts are true.
When they parade in rows of three soldiers, one soldier will be left. When they parade in rows of seven, two will be left, and in rows of \(19\), three will be left.
\end{problem}
\begin{answer}{modular.arithmetic:army.2}
We want to find remainders \((1,2,3)\) modulo \((3,7,19)\).
We let 
\begin{align*}
u_1 &= 7 \cdot 19 = 133, \\
u_2 &= 3 \cdot 19 = 57, \\
u_3 &= 3 \cdot 7 = 21.
\end{align*}
Modulo \(3,7,19\) these are
\begin{align*}
u_1 &= 1 \mod{3}, \\
u_2 &= 1 \mod{7}, \\
u_3 &= 2 \mod{19}.
\end{align*}
The multiplicative inverses of these, modulo \(3,7,19\), are obvious by reducing and inspection:
\begin{align*}
v_1 &= 1, \\
v_2 &= 1, \\
v_3 &= 10.
\end{align*}
By the Chinese remainder theorem, the answer, up to multiples of \(3 \cdot 7 \cdot 19=399\), is
\begin{align*}
r_1 u_1 v_1 + r_2 u_2 v_2 + r_3 u_3 v_3 
&=
1 \cdot 133 \cdot 1 
+
2 \cdot 57 \cdot 1
+
3 \cdot 21 \cdot 10,
\\
&=
133+114+630,
\\
&=877,
\\
&=79+2 \cdot 399.
\end{align*}
So there are at least 79 soldiers in Han Xin's army.
\end{answer}
\begin{problem}{modular:CRT.22.39}
Use the Chinese remainder theorem to find the smallest positive integer \(x\) so that
\begin{align*}
x&=2\pmod{22},\\
x&=8\pmod{39}.
\end{align*}
\end{problem}
\begin{answer}{modular:CRT.22.39}
We have \(r_1,r_2=2,8\), \(m_1,m_2=22,39\), so \(u_1=39\), \(u_2=22\).
Modulo \(22\):
\[
v_1=39^{-1}=(22+17)^{-1}=17^{-1}.
\]
B\'ezout:
\begin{align*}
\begin{pmatrix}
1&0&17\\
0&1&22
\end{pmatrix}
&\quad\text{add \(-\)row 1 to row 2},\\
\begin{pmatrix}
1&0&17\\
-1&1&5
\end{pmatrix}
&\quad\text{add \(-3\cdot\)row 2 to row 1},\\
\begin{pmatrix}
4&-3&2\\
-1&1&5
\end{pmatrix}
&\quad\text{add \(-2\cdot\)row 1 to row 2},\\
\begin{pmatrix}
4&-3&2\\
-9&7&1
\end{pmatrix}
\end{align*}
So \((-9)(17)+(7)(22)=1\), so mod \(22\), \(v_1=-9=22-9=13\).
Modulo \(39\), \(v_2=22^{-1}\), 
\begin{align*}
\begin{pmatrix}
1&0&22\\
0&1&39
\end{pmatrix}
&\quad\text{add \(-\)row 1 to row 2},\\
\begin{pmatrix}
1&0&22\\
-1&1&17
\end{pmatrix}
&\quad\text{add \(-\)row 2 to row 1},\\
\begin{pmatrix}
2&-1&5\\
-1&1&17
\end{pmatrix}
&\quad\text{add \(-3\cdot\)row 1 to row 2},\\
\begin{pmatrix}
2&-1&5\\
-7&4&2
\end{pmatrix}
&\quad\text{add \(-2\cdot\)row 2 to row 1},\\
\begin{pmatrix}
16&-9&1\\
-7&4&2
\end{pmatrix}
\end{align*}
So \(v_2=16\).
Modulo \(39\cdot 22=858\), 
\begin{align*}
x&=r_1u_1v_1+r_2u_2v_2,\\
 &=(2)(39)(13)+(8)(22)(16),\\
 &=1014+2816,\\
 &=3830,\\
 &=(4)(858)+398,\\
 &=398.
\end{align*}
\end{answer}
\begin{problem}{modular:CRT.4.5.17}
Use the Chinese remainder theorem to find the smallest positive integer \(x\) so that
\begin{align*}
x&=3\pmod{4},\\
x&=4\pmod{5},\\
x&=0\pmod{17}.
\end{align*}
\end{problem}
\begin{answer}{modular:CRT.4.5.17}
Moduli \(m_1,m_2,m_3=4,5,17\), so \(u_1,u_2,u_3=85,68,20\).
Reduce modulo those to get to \(1,3,3\); we only need reciprocals of these remainders.
Easily see that we can take \(v_1,v_2,v_3=1,2,6\) as reciprocals of those remainders.
We have \(r_1,r_2,r_3=3,4,0\).
Modulo \(4\cdot5\cdot 17=340\), 
\begin{align*}
x&=r_1u_1v_1+r_2u_2v_2,\\
&=799,\\
&=119
\end{align*}
modulo \(340\).
\end{answer}

\section{Bird's eye view of the Chinese remainder theorem}
Given some integers \(m_1, m_2, \dots, m_n\), we consider sequences \(\pr{b_1, b_2, \dots, b_n}\) consisting of remainders: \(b_1\) a remainder modulo \(m_1\), and so on.
Add sequences of remainders in the obvious way:
\[
\pr{b_1,b_2,\dots,b_n}
+
\pr{c_1,c_2, \dots, c_n}
=
\pr{b_1+c_1,b_2+c_2, \dots, b_n+c_n}.
\]
Similarly, we can subtract and multiply sequences of remainders:
\[
\pr{b_1,b_2,\dots,b_n}
\pr{c_1,c_2, \dots, c_n}
=
\pr{b_1c_1,b_2c_2, \dots, b_nc_n},
\]
by multiplying remainders as usual, modulo the various \(m_1, m_2, \dots, m_n\).
\begin{example} 
Modulo \(\pr{3,5}\), we multiply 
\begin{align*}
(2,4)(3,2)
&=(2 \cdot 3, 4 \cdot 2),
\\
&=(6, 8),
\\
&=(0,3).
\end{align*}
\end{example}
Let \(m \defeq m_1 m_2 \dots m_n\).
To each remainder modulo \(m\), say \(b\), associate its remainder \(b_1\) modulo \(m_1\), \(b_2\) modulo \(m_2\), and so on.
Associate the sequence \(\vec{b}\defeq\pr{b_1,b_2,\dots,b_n}\) of all of those remainders.
In this way we make a map taking each remainder \(b\) modulo \(m\) to its sequence \(\vec{b}\) of remainders modulo all of the various \(m_i\).
Moreover, \(\overrightarrow{b+c}=\vec{b}+\vec{c}\) and \(\overrightarrow{b-c}=\vec{b}-\vec{c}\) and \(\overrightarrow{bc}=\vec{b}\vec{c}\), since each of these works when we take remainder modulo anything.
\begin{example}
Take \(m_1,m_2,m_3\) to be \(3,4,7\).
Then \(m=3 \cdot 4 \cdot 7 = 84\).
If \(b=8\) modulo \(84\), then 
\begin{align*}
b_1 &= 8 \mod{3}, \\
    &= 2, \\
b_2 &= 8 \mod{4}, \\
    &= 0, \\
b_3 &= 8 \mod{7}, \\
    &= 1, \\
\vec{b} &= \pr{b_1,b_2,b_3},\\ 
	&= \pr{2,0,1} \mod{(3,4,7)}.
\end{align*}
\end{example}
\begin{corollary}\label{corollary:CRT}
Take some positive integers \(m_1, m_2, \dots, m_n\), so that any two of them are coprime.
Let \(m \defeq m_1 m_2 \dots m_n\).
The map taking \(b\) to \(\vec{b}\), from remainders modulo \(m\) to sequences of remainders modulo \(m_1, m_2, \dots, m_n\), is one-to-one and onto, identifies sums with sums, products with products, differences with differences, units with sequences of units.
\end{corollary}

\section{Euler's totient function}
\emph{Euler's totient function}\define{Euler's totient function}\define{totient function} \(\phi\)\Notation{phi(m)}{\phi(m)}{Euler's totient function} assigns to each positive integer \(m=2,3, \dots\) the number of all remainders modulo \(m\) which are units (in other words, coprime to \(m\)) [in other words, which have reciprocals].
It is convenient to \emph{define} \(\phi(1)\defeq 1\).
\begin{problem}{modular.arithmetic:totient.values}
Explain by examining the remainders that the first few values of \(\phi\) are
\[
\begin{array}{@{}rl@{}}
\toprule
m & \phi(m) \\
\cmidrule(r){1-1}\cmidrule(l){2-2}
1 & 1 \\
2 & 1 \\
3 & 2 \\
4 & 2 \\
5 & 4 \\
6 & 2 \\
7 & 6 \\
\bottomrule
\end{array}
\]
\end{problem}
\begin{problem}{modular.arithmetic:prime.phi}
Prove that a positive integer \(m \ge 2\) is prime just when \(\phi(m)=m-1\).
\end{problem}
\begin{theorem}\label{theorem:totient}
Suppose that \(m\ge 2\) is an integer with prime factorizaton
\[
m = p_1^{a_1} p_2^{a_2} \dots p_n^{a_n},
\]
so that \(p_1, p_2, \dots, p_n\) are prime numbers and \(a_1, a_2, \dots, a_n\) are positive integers.
Then
\[
\phi(m)=
\pr{p_1^{a_1}-p_1^{a_1-1}}
\pr{p_2^{a_2}-p_2^{a_2-1}}
\dots
\pr{p_n^{a_n}-p_n^{a_n-1}}.
\]
\end{theorem}
\begin{proof}
If \(m\) is prime, this follows from problem~\vref{problem:modular.arithmetic:prime.phi}.

If \(b,c\) are coprime integers \(\ge 2\), then corollary~\vref{corollary:CRT} maps units modulo \(bc\) to pairs of a unit modulo \(b\) and a unit modulo \(c\), and is one-to-one and onto.
Therefore counting units: \(\phi(bc)=\phi(b)\phi(c)\).
Apply this repeatedly, by induction, to reduce to \(m\) having a single prime factor.

Suppose that \(m\) has just one prime factor, or in other words that \(m=p^a\) for some prime number \(p\) and integer \(a\).
It is tricky to count the remainders coprime to \(m\), but easier to count those not coprime, i.e. those which have a factor of \(p\).
Clearly these are the multiples of \(p\) between \(0\) and \(p^a-p\), so the numbers \(pj\) for \(0 \le j \le p^{a-1}-1\).
So there are \(p^{a-1}\) such remainders.
We take these out and we are left with \(p^a-p^{a-1}\) remainders left, i.e. coprime.
\end{proof}
\begin{theorem}[Euler]\label{theorem:Euler}
For any positive integer \(m\) and any integer \(b\), \(b\) is coprime to \(m\) just when
\[
\congmod[m]{b^{\phi(m)}}{1}.
\]
\end{theorem}
\begin{proof}
Working with remainders modulo \(m\), we have to prove that for any unit remainder \(b\), \(b^{\phi(m)}=1\) modulo \(m\).

Let \(U\) be the set of all units modulo \(m\), so \(U\) is a subset of the remainders \(0,1,2,\dots,m-1\).
The product of units is a unit, since it has a reciprocal (the product of the reciprocals).
Therefore the map
\[
u \in U \mapsto bu \in U
\]
is defined.
It has an inverse:
\[
u \in U \mapsto b^{-1}u \in U,
\]
where \(b^{-1}\) is the reciprocal of \(b\).
Writing out the elements of \(U\), say as
\[
u_1, u_2, \dots, u_q
\]
note that \(q=\phi(m)\).
Then multiplying by \(b\) scrambles these units into a different order:
\[
bu_1, bu_2, \dots, bu_q.
\]
If we multiply them all, and then scramble them back into order:
\[
\pr{bu_1}\pr{bu_2} \dots \pr{bu_q}= u_1 u_2 \dots u_q.
\]
Divide every unit \(u_1, u_2, \dots, u_q\) out of boths sides to find \(b^q=1\).
\end{proof}
\begin{problem}{modular.arithmetic:find.reciprocal}
Take an integer \(m \ge 2\).
Suppose that \(b\) is a unit in the remainders modulo \(m\).
Prove that the reciprocal of \(b\) is \(b^{\phi(m)-1}\).
\end{problem}
Euler's theorem is important because we can use it to calculate quickly modulo prime numbers (and sometimes even modulo numbers which are not prime).
\begin{example}
Modulo \(19\), let's find \(123456789^{987654321}\).
First, the base of this expression is \(123456789=6497725 \cdot 19 + 14\)
So modulo \(19\):
\[
123456789^{987654321}=14^{987654321}.
\]
That helps with the base, but the exponent is still large.
According to Euler's theorem, since \(19\) is prime, modulo \(19\):
\[
b^{19-1} = 1
\]
for any remainder \(b\ne 0\).
In particular, modulo \(19\),
\[
14^{18}=1.
\]
So every time we get rid of \(18\) copies of \(14\) multiplied together, we don't change our result. 
Divide \(18\) into the exponent:
\[
987654321 = 54869684 \cdot 18 + 9.
\]
So then modulo \(19\):
\[
14^{987654321} = 14^9.
\]
We leave the reader to check that \(14^9=18\) modulo \(19\), so that finally, modulo \(19\),
\[
123456789^{987654321}=18.
\]
\end{example}
\begin{problem}{modular:use.totient}
By hand, using Euler's totient function, compute \(127^{162}\) modulo \(120\).
\end{problem}
\begin{answer}{modular:use.totient}
Factor \(120 =2^3 \, 3^1 \, 5^1\).
So Euler's totient is \(\phi(120)=(2^3-2^2)(3^1-3^0)(5^1-5^0)=32\).
Since \(127\) is prime, it is coprime to \(120\).
The theorem of Euler's totient function says that \(127^{\phi(120)}=1\) modulo \(120\). 
So \(127^{32}=1\) modulo \(120\).
In other words, every time you multiply together \(32\) copies of \(127\), modulo \(120\), it is as if you multiplied together no copies.
Check that \(162 = 5\cdot 32+2\).
So modulo 120, \(127^{162}=127^{5 \cdot 32+2}=127^2\). 
This is not quite the answer. 
But we know that \(127=7\) modulo \(120\). 
So modulo 120, \(127^{162}=7^2=49\).
\end{answer}
\begin{lemma}\label{lemma:multiply.by.a}
For any prime number \(p\) and integers \(b\) and \(k\), \(b^{1+k(p-1)}=b\) modulo \(p\).
\end{lemma}
\begin{proof}
If \(b\) \emph{is} a multiple of \(p\) then \(b=0\) modulo \(p\) so both sides are zero. 
If \(b\) is not a multiple of \(p\) then \(b^{p-1}=1\) modulo \(p\), by Euler's theorem.
Take both sides to the power \(k\) and multiply by \(b\) to get the result.
\end{proof}
\begin{theorem}\label{theorem:generalized.Euler}
Suppose that \(m=p_1 p_2 \dots p_n\) is a product of distinct prime numbers.
Then for any integers \(b\) and \(k\) with \(k \ge 0\),
\[
\congmod[m]{b^{1+k\phi(m)}}{b}.
\]
\end{theorem}
\begin{proof}
By lemma~\vref{lemma:multiply.by.a}, the result is true if \(m\) is prime.
So if we take two prime numbers \(p\) and \(q\), then \(b^{1+k(p-1)(q-1)}=b\) modulo \(p\), but also modulo \(q\), and therefore modulo \(pq\).
The same trick works if we start throwing in more distinct prime factors into \(m\).
\end{proof}
\begin{problem}{modular.arithmetic:example.not.Euler}
Give an example of integers \(b\) and \(m\) with \(m \ge 2\) for which \(b^{1+\phi(m)} \ne b\) modulo \(m\).
\end{problem}

\section{Sage}
In Sage, the quotient of \(71\) modulo \(13\) is \verb!mod(71,13)!.
The tricky bit: it returns a ``remainder modulo \(13\)'', so if we write
\begin{sageblock}
a=mod(71,13)
\end{sageblock}
this will define \(a\) to be a ``remainder modulo 13''. 
The value of \(a\) is then \(\sage{a}\), but the value of \(a^2\) is \(\sage{a^2}\), because the result is again calculated modulo 13.

Euler's totient function is
\begin{sageblock}
euler_phi(777)
\end{sageblock}
which yields \(\phi(777)=\sage{euler_phi(777)}\).
To find \(14^{-1}\) modulo \(19\),
\begin{sageblock}
inverse_mod(14,19)
\end{sageblock}
yields
\(14^{-1}=\sage{inverse_mod(14,19)}\) modulo \(19\).

We can write our own version of Euler's totient function, just to see how it might look:
\begin{sageblock}
def phi(n):
    return prod(p^a-p^(a-1) for (p,a) in factor(n))
\end{sageblock}
where \verb!prod! means product, so that \verb!phi(666)! yields \(\phi(666)=\sage{phi(666)}\).
The code here uses the function \verb!factor()!, which takes an integer \(n\) and returns a list \verb!p=factor(n)! of its prime factors.
In our case, the prime factorisation of \(666\) is \(666=2 \cdot 3^2 \cdot 37\).
The expression \verb!p=factor(n)! when \(n=666\) yields a list \verb!p=[(2,1), (3,2), (37,1)]!, a list of the prime factors together with their powers.
To find these values, \verb!p[0]! yields \verb!(2,1)!, \verb!p[1]! yields \verb!(3,2)!, and \verb!p[2]! yields \verb!(37,1)!.
The expression \verb!len(p)! gives the length of the list \verb!p!, which is the number of entries in that list, in this case \(3\).
For each entry, we set \(b=p_i\) and \(e=a_i\) and then multiply the result \(r\) by \(b^e-b^{e-1}\).

To use the Chinese remainder theorem, suppose we want to find a number \(x\) so that \(x\) has remainders \(1\) mod \(3\) and \(2\) mod \(7\),
\begin{sageblock}
crt([1,2],[3,7])
\end{sageblock}
gives you \(x=\sage{crt([1,2],[3,7])}\); you first list the two remainders \verb!1,2! and then list the moduli \verb!3,7!.

Another way to work with modular arithmetic in sage: we can create an object which represents the remainder of \(9\) modulo \(17\):
\begin{sageblock}
a=mod(9,17)
a^(-1)
\end{sageblock}
yielding \(\sage{a^(-1)}\).
As long as all of our remainders are modulo the same number \(17\), we can do arithmetic directly on them:
\begin{sageblock}
b=mod(7,17)
a*b
\end{sageblock}
yields \(\sage{a*b}\).


