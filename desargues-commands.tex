% Commands for the discussion of Desargues's theorem:
% highest x value of any coordinate of any point in the picture:
\newcommand*{\maxX}{6}
% lowest x value of any coordinate of any point in the picture:
\newcommand*{\minX}{0.1}

\newcommand*{\AoneX}{2.99}
\newcommand*{\AoneY}{0}
\newcommand*{\Aone}{\AoneX,\AoneY}
\newcommand*{\AtwoX}{4.22}
\newcommand*{\AtwoY}{.6}
\newcommand*{\Atwo}{\AtwoX,\AtwoY}
\newcommand*{\AthreeX}{4.68}
\newcommand*{\AthreeY}{1.41}
\newcommand*{\Athree}{\AthreeX,\AthreeY}
\newcommand*{\PnoughtX}{4.05}
\newcommand*{\PnoughtY}{3.11}
\newcommand*{\Pnought}{\PnoughtX,\PnoughtY}
\newcommand*{\Tone}{1.5627}
\newcommand*{\Ttwo}{2.1116}
\newcommand*{\Tthree}{2.9353}

\newcommand*{\drawOrigin}
{
\coordinate (p) at (\Pnought){};
\DrawNode{p}
}

\newcommand*{\drawFirstTriangle}
{
\coordinate (a1) at (\Aone){};
\DrawNode{a1}
\coordinate (a2) at (\Atwo){};
\DrawNode{a2}
\coordinate (a3) at (\Athree){};
\DrawNode{a3}
\begin{scope}[on background layer]
\fill[shapeOne] (a1) -- (a2) -- (a3) -- cycle;
\end{scope}
}

\newcommand*{\Bone}{{\Tone*(\AoneX-\PnoughtX)+\PnoughtX},{\Tone*(\AoneY-\PnoughtY)+\PnoughtY}}
\newcommand*{\Btwo}{{\Ttwo*(\AtwoX-\PnoughtX)+\PnoughtX},{\Ttwo*(\AtwoY-\PnoughtY)+\PnoughtY}}
\newcommand*{\Bthree}{{\Tthree*(\AthreeX-\PnoughtX)+\PnoughtX},{\Tthree*(\AthreeY-\PnoughtY)+\PnoughtY}}

\newcommand*{\drawSecondTriangle}
{
\coordinate (b1) at (\Bone){};
\DrawNode{b1}
\coordinate (b2) at (\Btwo){};
\DrawNode{b2}
\coordinate (b3) at (\Bthree){};
\DrawNode{b3}
\begin{scope}[on background layer]
\fill[shapeTwo] (b1) -- (b2) -- (b3) -- cycle;
\end{scope}
}

\newcommand*{\connectVantageAndFirstTwoTriangles}
{
\begin{scope}[on background layer]
\draw[curveZero,very thick] (p) -- (a1) -- (b1);
\draw[curveZero,very thick] (p) -- (a2) -- (b2);
\draw[curveZero,very thick] (p) -- (a3) -- (b3);
\end{scope}
}

\newcommand*{\findFirstIntersection}
{
\coordinate (c1) at (intersection of a2--a3 and b2--b3) {};
}
\newcommand*{\findSecondIntersection}
{
\coordinate (c2) at (intersection of a1--a3 and b1--b3) {};
}
\newcommand*{\findThirdIntersection}
{
\coordinate (c3) at (intersection of a1--a2 and b1--b2) {};
}

\newcommand*{\findAllIntersections}
{
\findFirstIntersection
\findSecondIntersection
\findThirdIntersection
}

\newcommand*{\showFirstIntersectionPoint}
{
\DrawNode{c1}
}

\newcommand*{\showSecondIntersectionPoint}
{
\DrawNode{c2}
}

\newcommand*{\showThirdIntersectionPoint}
{
\DrawNode{c3}
}


\newcommand*{\showAllIntersectionPoints}
{
\showFirstIntersectionPoint
\showSecondIntersectionPoint
\showThirdIntersectionPoint
}

\newcommand*{\showHowFirstIntersectionIsConstructed}
{
\begin{scope}[on background layer]
\draw[curveOne,very thick] (a3) -- (c1); % line through c1
\draw[curveOne,very thick] (a2) -- (c1); % line through c1
\draw[curveOne,very thick] (b2) -- (c1); % line through c1
\draw[curveOne,very thick] (b3) -- (c1); % line through c1
\end{scope}
\showFirstIntersectionPoint
}

\newcommand*{\showHowSecondIntersectionIsConstructed}
{
\begin{scope}[on background layer]
\draw[curveTwo,very thick] (a3) -- (c2); % line through c2
\draw[curveTwo,very thick] (a1) -- (c2); % line through c2
\draw[curveTwo,very thick] (b1) -- (c2); % line through c2
\draw[curveTwo,very thick] (b3) -- (c2); % line through c2
\end{scope}
\showSecondIntersectionPoint
}

\newcommand*{\showHowThirdIntersectionIsConstructed}
{
\begin{scope}[on background layer]
\draw[curveThree,very thick] (a1) -- (c3); % line through c3
\draw[curveThree,very thick] (a2) -- (c3); % line through c3
\draw[curveThree,very thick] (b1) -- (c3); % line through c3
\draw[curveThree,very thick] (b2) -- (c3); % line through c3
\end{scope}
\showThirdIntersectionPoint
}

\newcommand*{\showHowAllIntersectionsAreConstructed}
{
\showHowFirstIntersectionIsConstructed
\showHowSecondIntersectionIsConstructed
\showHowThirdIntersectionIsConstructed
}

\newcommand*{\showTheLine}
{
\begin{scope}[on background layer]
\draw[curveZero,very thick] (c1) -- (c2);
\draw[curveZero,very thick] (c1) -- (c3);
\draw[curveZero,very thick] (c2) -- (c3);
\end{scope}
\showAllIntersectionPoints
}

\newcommand*{\showThirdTriangle}
{
\begin{scope}[on background layer]
\fill[
shapeThree,draw=curveThree,very thick
%green!50!gray,opacity=.1
] (c1) to[bend left=85] (c2) to[bend right=85] (c3) to[bend left=90] (c1);
\end{scope}
\showAllIntersectionPoints
}


\newcommand{\Desargue}[1]{%
\begin{center}
\begin{tikzpicture}
% Draw a strut to make all of the diagrams have the same width, so the same x coordinate in each diagram gives a point the same distant across the page.
\begin{scope}[on background layer]
\draw[white] (\minX,0) -- (\maxX,0); 
\end{scope}
\IfStrEqCase{#1}{%
{1}%
{%%
\drawOrigin
\drawFirstTriangle
}%%
{2}%
{%
\drawOrigin
\drawFirstTriangle
\drawSecondTriangle
\connectVantageAndFirstTwoTriangles
\drawFirstTriangle
\drawSecondTriangle
}%
{3}%
{%%
\drawOrigin
\drawFirstTriangle
\drawSecondTriangle
\findFirstIntersection
\showHowFirstIntersectionIsConstructed
}%%
{4}%
{%%
\drawOrigin
\drawFirstTriangle
\drawSecondTriangle
\findSecondIntersection
\showHowSecondIntersectionIsConstructed
}%%
{5}%
{%%
\drawOrigin
\drawFirstTriangle
\drawSecondTriangle
\findThirdIntersection
\showHowThirdIntersectionIsConstructed
}%%
{6}%
{%%
\drawOrigin
\drawFirstTriangle
\drawSecondTriangle
\findAllIntersections
\showTheLine
}%%
{7}%
{%%
\drawOrigin
\drawFirstTriangle
\drawSecondTriangle
\connectVantageAndFirstTwoTriangles
\findAllIntersections
\showHowAllIntersectionsAreConstructed
\showTheLine
}%%
{8}%
{%%
\drawOrigin
\drawFirstTriangle
\drawSecondTriangle
\connectVantageAndFirstTwoTriangles
\findAllIntersections
\showHowAllIntersectionsAreConstructed
\showThirdTriangle
}%%
{9}%
{%%
\drawFirstTriangle
\drawSecondTriangle
\findAllIntersections
\showHowAllIntersectionsAreConstructed
\showTheLine
}%%
{10}%
{%%
\drawFirstTriangle
\drawSecondTriangle
\findAllIntersections
\showHowAllIntersectionsAreConstructed
\showTheLine
\begin{scope}[on background layer]
\draw[opacity=0] (c1) -- ++(6cm,0) node[black] (c4) {\(c_4\)};
\draw[opacity=0] (c3) -- ++(6cm,0) node[brown] (c5) {\(c_5\)};
\fill[shapeZero,draw=curveZero] (c4.center) -- (c5.center) -- (c3.center) -- (c1.center) -- cycle;
\end{scope}
\begin{scope}[on background layer]
\draw[opacity=0] (c1) -- ++(6cm,4cm) node[black] (d1) {\(d_1\)};
\draw[opacity=0] (c3) -- ++(6cm,4cm) node[brown] (d3) {\(d_3\)};
\fill[shapeZero,draw=curveZero] (d1.center) -- (d3.center) -- (c3.center) -- (c1.center) -- cycle;
\end{scope}
}%%
}%% end IfStrEqCase
\end{tikzpicture}
\end{center}
}
