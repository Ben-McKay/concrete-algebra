\documentclass{article}
\usepackage{xstring}
\usepackage{tikz}
\usetikzlibrary{
arrows,
backgrounds,
decorations.pathmorphing,
fit,
intersections,
petri,
positioning,
through}
\begin{document}

% highest x value of any coordinate of any point in the picture:
\newcommand*{\maxX}{5.5}
% lowest x value of any coordinate of any point in the picture:
\newcommand*{\minX}{0}
\newcommand*{\tikzscale}{.5}
\newcommand*{\dotr}{2pt}
\newcommand*{\Aone}{0,6}
\newcommand*{\Atwo}{2,5.5}
\newcommand*{\Athree}{1.5,4}
\newcommand*{\Pnought}{0,8}

\newcommand*{\drawOrigin}
{
\fill[blue!40] (\Pnought) circle (\dotr) node (p) {};
}

\newcommand*{\drawFirstTriangle}
{
\fill[cyan!5,opacity=.75] (\Aone) -- (\Atwo) -- (\Athree) -- cycle;
\fill[cyan!40] (\Aone) circle (\dotr) node (a1) {};
\fill[cyan!40] (\Atwo) circle (\dotr) node (a2) {};
\fill[cyan!40] (\Athree) circle (\dotr) node (a3) {};
}


\newcommand*{\Bone}{0,0}
\newcommand*{\Btwo}{2.75*2,8-2.75*2.5}
\newcommand*{\Bthree}{1.5*1.5,8-1.5*4}

\newcommand*{\drawSecondTriangle}
{
\fill[brown!15] (\Bone) -- (\Btwo) -- (\Bthree) -- cycle;
\fill[brown!50] (\Bone) circle (\dotr) node (b1) {};
\fill[brown!50] (\Btwo) circle (\dotr) node (b2) {};
\fill[brown!50] (\Bthree) circle (\dotr) node (b3) {};
}

\newcommand*{\connectVantageAndFirstTwoTriangles}
{
\draw[gray,opacity=.5] (p) -- (a1) -- (b1);
\draw[gray,opacity=.5] (p) -- (a2) -- (b2);
\draw[gray,opacity=.5] (p) -- (a3) -- (b3);
}

\newcommand*{\findFirstIntersection}
{
\node (c1) at (intersection of a2--a3 and b2--b3) {};
}
\newcommand*{\findSecondIntersection}
{
\node (c2) at (intersection of a1--a3 and b1--b3) {};
}
\newcommand*{\findThirdIntersection}
{
\node (c3) at (intersection of a1--a2 and b1--b2) {};
}

\newcommand*{\findAllIntersections}
{
\findFirstIntersection
\findSecondIntersection
\findThirdIntersection
}

\newcommand*{\showFirstIntersectionPoint}
{
%\draw[white] (c1) circle (1.4pt);
\fill[gray!30] (c1) circle (\dotr);
}

\newcommand*{\showSecondIntersectionPoint}
{
%\draw[white] (c2) circle (1.4pt);
\fill[gray!30] (c2) circle (\dotr);
}

\newcommand*{\showThirdIntersectionPoint}
{
%\draw[white] (c3) circle (1.4pt);
\fill[gray!30] (c3) circle (\dotr);
}


\newcommand*{\showAllIntersectionPoints}
{
\showFirstIntersectionPoint
\showSecondIntersectionPoint
\showThirdIntersectionPoint
}

\newcommand*{\showHowFirstIntersectionIsConstructed}
{
\draw[gray] (a3) -- (c1); % line through c1
\draw[gray] (b3) -- (c1); % line through c1
\showFirstIntersectionPoint
}

\newcommand*{\showHowSecondIntersectionIsConstructed}
{
\draw[gray] (a3) -- (c2); % line through c1
\draw[gray] (b3) -- (c2); % line through c1
\showSecondIntersectionPoint
}

\newcommand*{\showHowThirdIntersectionIsConstructed}
{
\draw[gray] (a2) -- (c3); % line through c1
\draw[gray] (b2) -- (c3); % line through c1
\showThirdIntersectionPoint
}

\newcommand*{\showHowAllIntersectionsAreConstructed}
{
\showHowFirstIntersectionIsConstructed
\showHowSecondIntersectionIsConstructed
\showHowThirdIntersectionIsConstructed
}

\newcommand*{\showTheLine}
{
\draw[green!50!gray] (c1) -- (c2);
\draw[green!50!gray] (c2) -- (c3);
\showAllIntersectionPoints
}

\newcommand*{\showThirdTriangle}
{
\fill[green!50!gray,opacity=.1] (c1) to[bend left=45] (c2) to[bend right=45] (c3) to[bend left=90] (c1);
%\draw[green!50!gray] (c1) to[bend left=45] (c2);
%\draw[green!50!gray] (c2) to[bend right=45] (c3);
%\draw[green!50!gray] (c3) to[bend left=90] (c1);
\showAllIntersectionPoints
}

\newcommand*{\firstPicture}
{
\drawOrigin
\drawFirstTriangle
}

\newcommand*{\secondPicture}
{
\drawOrigin
\drawFirstTriangle
\drawSecondTriangle
\connectVantageAndFirstTwoTriangles
\drawFirstTriangle
\drawSecondTriangle
}

\newcommand*{\thirdPicture}
{
\drawOrigin
\drawFirstTriangle
\drawSecondTriangle
\findFirstIntersection
\showHowFirstIntersectionIsConstructed
}

\newcommand*{\fourthPicture}
{
\drawOrigin
\drawFirstTriangle
\drawSecondTriangle
\findSecondIntersection
\showHowSecondIntersectionIsConstructed
}

\newcommand*{\fifthPicture}
{
\drawOrigin
\drawFirstTriangle
\drawSecondTriangle
\findThirdIntersection
\showHowThirdIntersectionIsConstructed
}

\newcommand*{\sixthPicture}
{
\drawOrigin
\drawFirstTriangle
\drawSecondTriangle
\findAllIntersections
\showTheLine
}

\newcommand*{\seventhPicture}
{
\drawOrigin
\drawFirstTriangle
\drawSecondTriangle
\connectVantageAndFirstTwoTriangles
\findAllIntersections
\showHowAllIntersectionsAreConstructed
\showTheLine
}

\newcommand*{\eighthPicture}
{
\drawOrigin
\drawFirstTriangle
\drawSecondTriangle
\connectVantageAndFirstTwoTriangles
\findAllIntersections
\showHowAllIntersectionsAreConstructed
\showThirdTriangle
}

\newcommand{\Desargue}[1]{%
\begin{center}
\begin{tikzpicture}[scale=\tikzscale]
% Draw a strut to make all of the diagrams have the same width, so the same x coordinate in each diagram gives a point the same distant across the page.
\draw[white] (\minX,0) -- (\maxX,0); 
\IfStrEqCase{#1}{%
{1}%
{%%
\firstPicture
}%%
{2}%
{%
\secondPicture
}%
{3}%
{%%
\thirdPicture
}%%
{4}%
{%%
\fourthPicture
}%%
{5}%
{%%
\fifthPicture
}%%
{6}%
{%%
\sixthPicture
}%%
{7}%
{%%
\seventhPicture
}%%
{8}%
{%%
\eighthPicture
}%%
}%% end IfStrEqCase
\end{tikzpicture}
\end{center}
}

\end{document}
