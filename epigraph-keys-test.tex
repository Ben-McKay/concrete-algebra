\documentclass{article}

\RequirePackage[utf8]{inputenc}
% Default input file encoding

\RequirePackage[T1]{fontenc}
% Default font encoding

\RequirePackage{strict}

\RequirePackage{lmodern}
% Nice fonts.

\RequirePackage{fixmath}
% Make Greek letters look right.

\RequirePackage[mathscr]{eucal}

\RequirePackage[%
activate={true,nocompatibility},%
final,%
tracking=true,%
kerning=true,%
spacing=true,%
factor=1100,%
stretch=10,%
shrink=10]{microtype}
\microtypecontext{spacing=nonfrench}


\usepackage{lipsum}
\usepackage{siunitx}


\usepackage{epigraph-keys}




\begin{document}
\section{Introduction}
\epigraph[author={Ludwig Wittgenstein}, source={Culture and Value}]{With my full philosophical rucksack I can only climb slowly up the mountain of mathematics.}
\epigraph[author={Ludwig Wittgenstein}, source={Culture and Value}]{With my full philosophical rucksack I can only climb slowly up the mountain of mathematics.}
\epigraph[author={Ludwig Wittgenstein}, source={Culture and Value}]{With my full philosophical rucksack I can only climb slowly up the mountain of mathematics.}
\epigraph[author={Paul Painlev\'e}, source={Analyse des travaux scientifiques}, translation={The shortest and easiest path between any two facts about the real domain passes through the complex domain.}]{Entre deux vérités du domaine réel, le chemin le plus facile et le plus court passe bien souvent par le domaine complexe.}
\lipsum
\newpage
\begin{epigraphs}
\qitem[source={Applied Optics}, etc={vol. 11, A14, 1972}]
{The temperature of Heaven can be rather accurately computed.  Our authority is Isaiah 30:26, ``Moreover, the light of the Moon shall be as the light of the Sun and the light of the Sun shall be sevenfold, as the light of seven days.''  Thus Heaven receives from the Moon as much radiation as we do from the Sun, and in addition \(7 \times 7=49\) times as much as the Earth does from the Sun, or 50 times in all.  The light we receive from the Moon is one \(1/\num{10000}\) of the light we receive from the Sun, so we can ignore that\dots.  The radiation falling on Heaven will heat it to the point where the heat lost by radiation is just equal to the heat received by radiation, i.e., Heaven loses 50 times as much heat as the Earth by radiation.  Using the Stefan-Boltzmann law for radiation, \((H/E)\) temperature of the earth \((\SI{-300}{\kelvin})\), gives \(H\) as \(\SI{798}{\kelvin} (\SI{525}{\celsius})\).  The exact temperature of Hell cannot be computed\dots.  [However] Revelations 21:8 says ``But the fearful, and unbelieving \dots shall have their part in the lake which burneth with fire and brimstone.''  A lake of molten brimstone means that its temperature must be at or below the boiling point, \(\SI{444.6}{\celsius}\).  We have, then, that Heaven, at \(\SI{525}{\celsius}\) is hotter than Hell at \(\SI{445}{\celsius}\).}
\qitem[author={Hermann Weyl}, source={Invariants}, etc={Duke Mathematical Journal 5, 1939, 489--502}]{In these days the angel of topology and the devil of abstract algebra fight for the soul of every individual discipline of mathematics.}
\qitem[author={Goethe}, source={Faust}]{--- and so who are you, after all? \\
--- I am part of the power which forever wills evil and forever works good.} 
\qitem[source={Quran}, etc={2:1/2:6-2:10 \emph{The Cow}}]{This Book is not to be doubted.}
\end{epigraphs}
\lipsum
\end{document}
