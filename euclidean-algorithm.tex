\chapter{Greatest common divisors}
\epigraph[author={Euclid}]{The laws of nature are but the mathematical thoughts of God.}\SubIndex{Euclid}
\epigraph[author={Edna St. Vincent Millay}, source={Euclid Alone}]
{Euclid alone has looked on Beauty bare. \\
Let all who prate of Beauty hold their peace, \\
And lay them prone upon the earth and cease \\
To ponder on themselves, the while they stare \\
At nothing, intricately drawn nowhere \\
In shapes of shifting lineage; let geese \\
Gabble and hiss, but heroes seek release \\
From dusty bondage into luminous air. \\
O blinding hour, O holy, terrible day, \\
When first the shaft into his vision shone \\
Of light anatomized! Euclid alone \\
Has looked on Beauty bare. Fortunate they \\
Who, though once only and then but far away, \\
Have heard her massive sandal set on stone.}\SubIndex{St. Vincent Millay, Edna}\SubIndex{beauty}
\section{The extended Euclidean algorithm}\define{extended Euclidean algorithm}\define{Euclidean algorithm!extended}\define{algorithm!extended Euclidean}
\begin{theorem}[B\'ezout]
For any two integers \(b, c\), not both zero, there are integers \(s,t\) so that \(sb+tc=\gcd{b,c}\).
\end{theorem}
These \emph{B\'ezout coefficients}\define{B\'ezout!coefficients} \(s, t\) play an essential role in advanced arithmetic, as we will see.
\begin{example}
To find the B\'ezout coefficients of \(12, 8\), write out a matrix
\[
\begin{pmatrix}
1 & 0 & 12 \\
0 & 1 & 8
\end{pmatrix}.
\]
Repeatedly add some integer multiple of one row to the other, to try to make the bigger number in the last column (bigger by absolute value) get smaller (by absolute value).
In this case, we can subtract row 2 from row 1, to get rid of as much of the \(12\) as possible.
All operations are carried out a whole row at a time:
\[
\begin{pmatrix}
1 & -1 & 4 \\
0 & 1 & 8
\end{pmatrix}
\]
Now the bigger number (by absolute value) in the last column is the \(8\).
We subtract as large an integer multiple of row 1 from row 2, in this case subtract \(2\pr{\text{row 1}}\) from row 2, to kill as much of the \(8\) as possible:
\[
\begin{pmatrix}
1 & -1 & 4 \\
-2 & 3 & 0
\end{pmatrix}
\]
Once we get a zero in the last column, the other entry in the last column is the  greatest common divisor, and the entries in the same row as the greatest common divisor are the B\'ezout coefficients.
In our example, the B\'ezout coefficients of \(12, 8\) are \(1,-1\) and the greatest common divisor is \(4\).
This trick to calculate B\'ezout coefficients is the \emph{extended Euclidean algorithm}.\define{extended Euclidean algorithm}\define{Euclidean algorithm!extended}
\end{example}

\begin{example}
Again, we have to be a little bit careful about minus signs.
Find B\'ezout coefficients of \(-8,-4\):
\[
\begin{pmatrix}
1 & 0 & -8 \\
0 & 1 & -4
\end{pmatrix} \to
\begin{pmatrix}
1 & -2 & 0 \\
0 & 1 & -4
\end{pmatrix},
\]
the process stops here, and where we expect to get an equation \(sb+tc=\gcd{b,c}\), instead we get
\[
0(-8)+1(-4)=-4, 
\]
which has the wrong sign, a negative greatest common divisor.
So if the answer pops out a negative for the greatest common divisor, we change signs: \(s=0, t=-1, \gcd{b,c}=4\).
\end{example}

\begin{theorem}\label{theorem:B\'ezout.euclidean}
The extended Euclidean algorithm calculates B\'ezout coefficients.
In particular, B\'ezout coefficients \(s,t\) exist for any integers \(b,c\) with \(c \ne 0\).
\end{theorem}
\begin{proof}
At each step in the extended Euclidean algorithm, the third column proceeds exactly by the Euclidean algorithm, replacing the larger number (in absolute value) with the remainder by the division, except for perhaps a minus sign.
So at the last step, the last nonzero number in the third column is the greatest common divisor, except for perhaps a minus sign.

Our extended Euclidean algorithm starts with
\[
\begin{pmatrix}
1 & 0 & b \\
0 & 1 & c
\end{pmatrix},
\]
a matrix which satisfies
\[
\begin{pmatrix}
1 & 0 & b \\
0 & 1 & c
\end{pmatrix}
\begin{pmatrix}
b \\
c \\
-1
\end{pmatrix}
=
\begin{pmatrix}
0 \\
0
\end{pmatrix},
\]
It is easy to check that if some \(2 \times 3\) matrix \(M\) satisfies
\[
M
\begin{pmatrix}
b \\
c \\
-1
\end{pmatrix}
=
\begin{pmatrix}
0 \\
0
\end{pmatrix},
\]
then so does the matrix you get by adding any multiple of one row of \(M\) to the other row.

Eventually we get a zero in the final column, say
\[
\begin{pmatrix}
s & t & d \\
S & T & 0
\end{pmatrix},
\]
or
\[
\begin{pmatrix}
s & t & 0 \\
S & T & D
\end{pmatrix}.
\]
It doesn't matter which since, at any step, we could always swap the two rows without changing the steps otherwise.
So suppose we end up at
\[
\begin{pmatrix}
s & t & d \\
S & T & 0
\end{pmatrix}.
\]
But
\[
\begin{pmatrix}
s & t & d \\
S & T & 0
\end{pmatrix}
\begin{pmatrix}
b \\
c \\
-1
\end{pmatrix}
=
\begin{pmatrix}
sb+tc-d \\
0
\end{pmatrix}.
\]
This has to be zero, so: \(sb+tc=d\).
\end{proof}


\begin{proposition}
Take two integers \(b,c\), not both zero.
The number \(\gcd{b,c}\) is precisely the smallest positive integer which can be written as \(sb+tc\) for some integers \(s,t\).
\end{proposition}
\begin{proof}
Let \(d \defeq \gcd{b,c}\).
Since \(d\) divides into \(b\), we divide it, say as \(b=Bd\), for some integer \(B\).
Similarly, \(c=Cd\) for some integer \(C\).
Imagine that we write down some positive integer as \(sb+tc\) for some integers \(s,t\).
Then \(sb+sc=sBd+tCd=(sB+tC)d\) is a multiple of \(d\), so no smaller than \(d\).
\end{proof}

\begin{sagesilent}
def bezpretty(b,c):
    p,q,r=1,0,b
    s,t,u=0,1,c
    M = matrix([[p,q,r],[s,t,u]])
    result="\\begin{align*}\n"
    result=result+"&{}\\\\ ".format(latex(M))
    while u!=0:
        if abs(r)>abs(u):
            p,q,r,s,t,u=s,t,u,p,q,r
        Q=u//r
        s,t,u=s-Q*p,t-Q*q,u-Q*r
        M = matrix([[p,q,r],[s,t,u]])
        result=result+"&{}\\\\ \n ".format(latex(M))
    result=result+"&({})({})+({})({})={}\\\\ \n ".format(p,b,q,c,r)
    result=result+"\\end{align*}"
    return result
\end{sagesilent}

\begin{problem}{euclidean.algorithm:try.Bezout}
Find B\'ezout coefficients and greatest common divisors of
\begin{enumerate}
\item \(2468, 180\)
\item \(79, -22\)
\item \(45,16\)
\item \(-1000,2002\)
\end{enumerate}
\end{problem}
\begin{answer}{euclidean.algorithm:try.Bezout}
\begin{enumerate}
\item
\sagestr{bezpretty(2468,180)}
\item
\sagestr{bezpretty(79,-22)}
\item
\sagestr{bezpretty(45,16)}
\item
\sagestr{bezpretty(-1000,2002)}
\end{enumerate}
\end{answer}

\begin{problem}{euclidean.algorithm:several}
How would you find the greatest common divisor of several integers? The B\'ezout coefficients?
\end{problem}

\section{Sage}\label{section:euclid.sage}
\epigraph[author={Edsger Dijkstra}]{Computer science is no more about computers than astronomy is about telescopes.}\SubIndex{Dijkstra, Edsger}
The extended Euclidean algorithm is built in to sage:
\verb!xgcd(12,8)! returns \(\sage{xgcd(12,8)}\), the greatest common divisor and the B\'ezout coefficients, so that \(4=1 \cdot 12 + (-1) \cdot 8\).
We can write our own function to compute B\'ezout coefficients:
\begin{sageblock}
def bez(b,c):
    p,q,r=1,0,b
    s,t,u=0,1,c
    while u!=0:
        if abs(r)>abs(u):
            p,q,r,s,t,u=s,t,u,p,q,r
        Q=u//r
        s,t,u=s-Q*p,t-Q*q,u-Q*r
    return r,p,q
\end{sageblock}
so that \verb!bez(45,210)! returns a triple \((g,s,t)\) where \(g\) is the greatest commmon divisor of \(45,210\), while \(s,t\) are the B\'ezout coefficients.

