\chapter{Galois theory}\label{chapter:galois.theory}
\epigraph[author={Sim\'eon-Denis Poisson}]{We have made every effort to understand Galois's proof. His reasoning is not sufficiently clear, sufficiently developed, for us to judge its correctness and we can give no idea of it in this report.}%
\SubIndex{Poisson, Sim\'eon-Denis}
\section{Automorphisms}
\begin{example}
How can we save work in linear algebra?
The \(2\times 2\) matrix
\[
A=
\begin{pmatrix}
3&1\\
1&1
\end{pmatrix}
\]
has eigenvalues \(\lambda=2\pm\sqrt{2}\), as the reader can check.
When we compute the eigenvectors, we find that the \(\lambda=2-\sqrt{2}\) eigenspace is spanned by an eigenvector
\[
\begin{pmatrix}
1\\
\sqrt{2}+1
\end{pmatrix}.
\]
We can guess that the other eigenvalue has corresponding eigenvector given by changing \(\sqrt{2}\mapsto-\sqrt{2}\), so
\[
\begin{pmatrix}
1\\
\sqrt{2}-1
\end{pmatrix}.
\]

Why is this correct?
Note that \(A\) has only rational coefficients.
Write elements of \(K=\Q{}(\sqrt{2})\) as \(a+b\sqrt{2}\) with \(a,b\in k=\Q{}\).
Then clearly \(\sqrt{2}\mapsto-\sqrt{2}\) preserves addition and multiplication, so preserves all of the arithmetic operations used to find eigenvalues and eigenvectors.
It also preserves all entries of \(A\), since they are rational: \(a+0\sqrt{2}\).
The eigenvalues lie in \(K\).
The operations used to find eigenvectors, pure linear algebra, never take us out of \(K\).
So \(\sqrt{2}\mapsto-\sqrt{2}\), applied to all entries in the first eigenvector with the first eigenvalue give us a second eigenvector with the second eigenvalue.
\end{example}
\begin{example}
By the same argument, the eigenvalues and eigenvectors of any real matrix need not be real, but are interchanged by complex conjugation. 
\end{example}
An \emph{automorphism}\define{automorphism} of a ring \(R\) is a morphism \(f \colon R \to R\) which has an inverse \(f^{-1} \colon R \to R\), which is also a morphism.
\begin{problem}{galois:aut.gp}
Prove that the set of all automorphisms of any ring \(R\) is a group, the \emph{automorphism group}\define{automorphism!group}\define{group!automorphism} of \(R\).
\end{problem}
\begin{example}
The set of rational numbers of the form \(b+c\sqrt{2}\), for \(b,c \in \Q{}\), forms a field, usually denoted \(\Q{}(\sqrt{2})\).
Indeed, when we add such numbers
\[
(7+3\sqrt{2}) \, + \, (4-\sqrt{2}) 
=
(7+4)+(3-1)\sqrt{2}.
\]
Similarly if we subtract.
If we multiply,
\[
(7+3\sqrt{2})(4-\sqrt{2}) 
=
\pr{ 
7 \cdot 4
+ 3 \cdot 1 \cdot 2
}
+
\pr{
7 \cdot \pr{-1}+3 \cdot 4
}
\sqrt{2}.
\]
Finally, to compute reciprocals,
\[
\frac{1}{7+6\sqrt{2}}=\frac{1}{7+6\sqrt{2}}\frac{7-6\sqrt{2}}{7-6\sqrt{2}}=\frac{7-6\sqrt{2}}{7^2-2 \cdot 6^2}
\]
which the reader can simplify.
The map
\[
f(b+c\sqrt{2})=b-c\sqrt{2}
\]
is an automorphism, i.e. a morphism of the field to itself.
Note that \(f^{-1}=f\).
\end{example}
\begin{problem}{galois:preserve.minus}
Prove that every morphism \(f \colon R \to S\) of rings preserves subtraction.
\end{problem}
\begin{answer}{galois:preserve.minus}
If \(z=x-y\) then \(f(x)=f(y+z)=f(y)+f(z)\) so \(f(z)=f(x)-f(y)\).
\end{answer}
Suppose that \(R\) is a ring.
Suppose that, for any elements \(b,c\) of \(R\) with \(c\ne 0\), there is a unique element \(d\) of \(R\) so that \(dc=b\).
Write this element as \(d=b/c\) and say that \(R\) has \emph{right inverses}.
\begin{lemma}
Every morphism \(f \colon R \to S\) of rings with right inverses preserves right inverses.
\end{lemma}
\begin{proof}
If \(d=b/c\) then \(dc=b\) so \(f(dc)=f(b)=f(d)f(c)\) so \(f(d)=f(b)/f(c)\).
\end{proof}
\begin{lemma}
Every automorphism of a ring preserves \(0\).
Every automorphism of a ring with identity preserves the identity element.
\end{lemma}
\begin{proof}
Every element \(b\) of \(R\) satisfies \(b=0+b=b+0\).
Any automorphism \(h \colon R \to R\) satisfies \(h(b)=h(0+b)=h(0)+h(b)=h(b+0)=h(b)+h(0)\) for all \(b \in R\).
Since \(h\) has an inverse, \(h\) is one-to-one and onto, so every element \(c\) of \(R\) has the form \(c=h(b)\) for some \(b \in R\).
So \(c=h(0)+c=c+h(0)\) for all \(c\) in \(R\).
In particular, taking \(c=0\), \(0=h(0)+0=h(0)\).

Every element \(b\) of \(R\) satisfies \(b=1b=b1\).
Any automorphism \(h \colon R \to R\) satisfies \(h(b)=h(1b)=h(1)h(b)=h(b1)=h(b)h(1)\) for all \(b \in R\).
Since \(h\) has an inverse, \(h\) is one-to-one and onto, so every element \(c\) of \(R\) has the form \(c=h(b)\) for some \(b \in R\).
So \(c=h(1)c=ch(1)\) for all \(c\) in \(R\).
In particular, taking \(c=1\), \(1=h(1)1=h(1)\).
\end{proof}
\begin{lemma}
The only automorphism of any of the fields \(\Q{},\R{}\) or \(\Zmod{p}\) is the identity map \(b \mapsto b\).
\end{lemma}
\begin{proof}
Suppose that \(k=\Q{}\) or \(k=\R{}\) or \(k=\Zmod{p}\).
If \(f \colon k \to k\) is an automorphism, then \(f(1)=1\).
Therefore \(f(1+1)=f(1)+f(1)=1+1\).
Similarly \(f(1+1+1)=1+1+1\), and so on.
Therefore if \(k=\Zmod{p}\) we have exhausted all elements of \(k\) and \(f(b)=b\) for every element \(b\) of \(k\).
In the same way, if \(k=\Q{}\) or \(k=\R{}\), then \(k(n)=n\) for all positive integers \(n\).
Similarly, \(f(0)=0\).
But then \(f(-n)=f(0-n)=f(0)-f(n)=0-f(n)=-f(n)\).
So \(f\) is the identity map on the integers.
Apply \(f\) to a ratio \(b/c\) of integers, and since \(f\) preserves right inverses, \(f(b/c)=f(b)/f(c)=b/c\).
Hence if \(k=\Q{}\), then \(f\) is the identity map.

We can assume that \(k=\R{}\).
Say that a number \(x\) in \(\R{}\) is \emph{positive} if \(x \ne 0\) and \(x=y^2\) for some number \(y\).
Clearly positivity is preserved under any automorphism.
Write \(x < y\) to mean that \(y-x\) is positive, and \(x \le y\) to mean that \(y-x\) is positive or zero.
Automorphisms preserve the ordering of real numbers.
Take any real number \(x\) and approximate \(x\) from below by rational numbers \(r_1,r_2,\dots, \to x\) and from above by rational numbers \(s_1,s_2,\dots \to x\).
Then \(r_i < x < s_i\) so applying \(f\): \(r_i < f(x) < s_i\).
It follows that \(f(x)=x\), as real numbers as completely determined by their digits, i.e. by approximation by rationals.
\end{proof}
\begin{lemma}
There are two automorphisms of \(\Q{}(\sqrt{2})\): the identity map \(\iota(b+c\sqrt{2})=b+c\sqrt{2}\) and the map
\[
f(b+c\sqrt{2})=b-c\sqrt{2}.
\]
\end{lemma}
\begin{proof}
The same argument as above shows that any automorphism \(h\) of \(\Q{}(\sqrt{2})\) fixes all rational numbers \(\Q{}\subset \Q{}(\sqrt{2})\).
So \(h(b+c\sqrt{2})=h(b)+h(c)h(\sqrt{2})=b+ch(\sqrt{2})\): it suffices to determine \(h(\sqrt{2})\).
Note that \(\sqrt{2}^2=2\), so apply \(h\) to get \(h(\sqrt{2})^2=2\), i.e. \(h(\sqrt{2})=\pm \sqrt{2}\).
\end{proof}

\section{Galois groups}
The \emph{Galois group}\define{Galois!group}\define{group!Galois} of a field extension \(k \subset K\), denoted \(\Gal{k}{K}\) is the group of automorphisms of \(K\) which are the identity on \(k\).
\begin{example}
We found above that \(\Gal{\Q{}}{\Q{}(\sqrt{2})}=\set{\pm 1}\) where we let \(-1\) denote the transformation \(b+c\sqrt{2}=b-c\sqrt{2}\).
\end{example}
If \(b(x)\) is a polynomial with coefficients in \(k\), and \(g\) is a automorphism of \(K\) fixing all elements of \(k\), then \(g(b(x))=b(x)\), i.e. \(g\) fixes all coefficients.
Therefore if \(\alpha\) is a root of \(b(x)\) lying in \(K\), then so is \(g\alpha\):
\(
0=b(\alpha)=\sum b_j \alpha^j
\)
so
\[
0=g(b(\alpha)) = \sum b_j g(\alpha)^j.
\]
Hence the Galois group \(\Gal{k}{K}\) of a field extension \(K/k\) acts on the roots in \(K\) of all polynomials over \(k\).
If a polynomial splits into factors over \(k\), then clearly the Galois group \(\Gal{k}{K}\) permutes the \(K\)-roots of each factor individually.
\begin{example}
The Galois group of \(\Q{}(\sqrt{2})/\Q{}\) permutes \(\sqrt{2},-\sqrt{2}\), the roots of \(x^2-2\).
Note that the elements of \(\Q{}(\sqrt{2})\) have the form \(a+b\sqrt{2}\), for \(a,b\) rational.
Hence the elements of \(\Q{}(\sqrt{2}\) fixed by the Galois group are precisely the elements of \(\Q{}\).
\end{example}
A \emph{Galois extension}\define{Galois!extension}\define{extension!Galois} is an extension \(K/k\) so that the elements of \(K\) fixed by \(\Gal{k}{K}\) are precisely the elements of \(k\).
\begin{example}
In chapter~\ref{chapter:fields} we checked that the field \(K\defeq\Q{}(\sqrt[3]{2})\) consists of the numbers \(a+b2^{1/3}+c2^{2/3}\) for \(a,b,c\) rational.
Note that \(K\) is a subfield of the field of real numbers.
There is only one real number \(\alpha\) that satisfies \(\alpha^3=2\): the real number \(\alpha=2^{1/3}\).
Therefore there is only one element \(\alpha\) of \(K\) that satisfies \(\alpha^3=2\): the element \(\alpha=2^{1/3}\).
So every automorphism of \(K\) sends \(2^{1/3}\) to itself.
The Galois group of \(K/\Q{}\) is \(\Gal{\Q{}}{K}=\set{1}\).
The extension \(\Q{}(\sqrt[3]{2})/\Q{}\) is \emph{not} a Galois extension; the Galois group is too small, fixing everything.
\end{example}
\begin{example}
Let \(K\) be the splitting field of \(p(x)=x^3-2\) over \(k=\Q{}\).
There are \(3\) complex cube roots of \(2\): the real number \(2^{1/3}\), and its two rotations by a third of a revolution around the origin: call them \(\alpha\) and \(\bar\alpha\).
As we will see, by theorem~\vref{theorem:irreducible.orbit}, because \(p(x)\) is irreducible over \(k\), any two of these roots are swapped by some element of the Galois group of \(K/k\).
We can draw one of these permutations: complex conjugation preserves \(2^{1/3}\), and preserves the polynomial, and swaps the angles of the other roots, preserving their lengths, so swaps the other two roots.
Any permutation of the three roots arises as a unique element of the Galois group: once we swap roots to get any root we like to sit in the spot \(2^{1/3}\), we can then permute the other two by complex conjugation if they are not already where we want them.
So the Galois group of \(K/k\) is the symmetric group on \(3\) letters.
In particular, \(K/k\) is a Galois extension.
\end{example}
\begin{problem}{galois:symmetric.gp}
Prove that, over any field, the rational functions in finitely many variables form a Galois extension of the symmetric functions in those variables, with Galois group the group of permutations of the variables.
\end{problem}
\begin{answer}{galois:symmetric.gp}
For any field \(k\), take variables \(t=(t_1,\dots,t_n)\) and let \(e_1=e_1(t),\dots,e_n=e_n(t)\) be the elementary symmetric polynomials and \(e=(e_1,\dots,e_n)\).
The field of symmetric functions is by definition the fixed field of \(k(t)\) by permutations of the variables.
By problem~\vref{problem:permutions:rationals}, the field of symmetric functions is precisely the field \(k(e)\) generated by \(e_1,\dots,e_n\).
We want to see that \(k(t)/k(e)\) is a Galois extension, with automorphism group the group of permutations of \(t_1,\dots,t_n\).
The symmetric functions \(k(e)\) have the rational functions \(k(t)\) as a splitting field for the polynomial
\[
x^n-e_1x^{n-1}+e_2x^{n-2}-\dots+(-1)^ne_n.
\]
By lemma~\vref{lemma:auts.are.Sn}, the automorphisms of \(k(t_1,\dots,t_n)\) which fix \(k(e_1,\dots,e_n)\) are precisely the permutations of \(t_1,\dots,t_n\).
\end{answer}

\section{Adding roots}
\begin{example}
Start with a field \(k\), and take two elements \(b,c \in k\).
The roots of the equation \(0=x^2+bx+c\) lie in some extension \(K\).
If \(b^2-c\) is a square in \(k\), then \(K=k\).
Otherwise, the quadratic formula
\[
x=\frac{-b \pm \sqrt{b^2-4c}}{2}
\]
shows that the roots of our quadratic equation lie in the field \(K=k(\sqrt{b^2-4c})\), as long as \(k\) has characteristic not equal to \(2\).
In fields of characteristic \(2\), the quadratic formula doesn't hold, and the splitting field is more complicated.
\end{example}
\begin{theorem}\label{theorem:splitting.Galois}
Suppose that \(K\) is a splitting field of an irreducible polynomial \(p(x)\) over a field \(k\).
If \(\alpha,\beta \in K\) are roots of \(p(x)\) then the Galois group of \(K\) over \(k\) has an element which takes \(\alpha\) to \(\beta\).
In particular, \(K\) is a Galois extension of \(k\).
\end{theorem}
We will prove this result in chapter~\ref{chapter:quotient.rings}.
\begin{example}
Return to our previous example: in chapter~\ref{chapter:fields} we checked that the field \(K\defeq\Q{}(\sqrt[3]{2})\) consists of the numbers \(a+b2^{1/3}+c2^{2/3}\) for \(a,b,c\) rational.
Above, we saw that \(K\) has trivial Galois group over \(k=\Q{}\).
The polynomial \(p(x)=x^3-2\) has a root in \(K\), but not all of its roots.
Our theorem says that the splitting field \(L\) of \(p(x)\) over \(k\) is Galois.
Let \(\alpha=\sqrt[3]{2}\), so \(K=k(\alpha)\).
Factor \(p(x)=(x-\alpha)(x^2+\alpha x + \alpha^2)\).
But \(x^2+\alpha x + \alpha^2\) has no roots in \(K\), since its roots are not real (as we will see), while \(K\) lies inside \(\R{}\).
We need to extend to \(L\) to find roots for \(x^2+\alpha x + \alpha^2\), i.e. the two remaining roots of \(p(x)\).
Since \(x^2+\alpha x + \alpha^2\) is quadratic, \(L\) is a quadratic extension of \(K\), i.e. \(L=K(\beta)\) is given by adding
\[
\frac{-b\pm\sqrt{b^2-4ac}}{2a} = \frac{-\alpha \pm\sqrt{\alpha^2-4\alpha^2}}{2} =
-\frac{\alpha}{2}\left(1\pm \sqrt{-3} \right),
\]
i.e. \(L=K(\sqrt{-3})=k(\sqrt[3]{2},\sqrt{-3})\).
Automorphisms of \(L\) fixing \(k\) are determined by their action on \(\sqrt[3]{2}\) and \(\sqrt{-3}\).
Complex conjugation is one such automorphism.
Let \(\omega\) be the cube root of \(1\) lying in the upper half plane.
The three roots of \(p(x)\) are \(\sqrt[3]{2}, \omega \sqrt[3]{2}\) and \(\bar\omega \sqrt[3]{2}\).
Complex conjugation swaps the last two roots.
Some other automorphism in the Galois group swaps the first two roots, by the theorem.
That map then must leave the other root alone, as it must remain a root.
So we can identify two automorphisms in the Galois group.
Composing these two, we see that every permutation of the roots \(\sqrt[3]{2}, \omega \sqrt[3]{2}\) and \(\bar\omega \sqrt[3]{2}\) is obtained by a unique automorphism in the Galois group of \(L\) over \(k\), i.e. \(\Gal{k}{L}\) is the symmetric group on three letters.
\end{example}
\begin{example}
Pick a prime \(p\).
Over \(k=\Q{}\), factor \(x^p-1=(x-1)(x^{p-1}+x^{p-2}+\dots+x+1)\).
The second factor is irreducible by problem~\vref{problem:polynomials:Eisenstein.cyclotomic}.
The Galois group then takes any \(p\)-th root of \(1\) to any other (except for \(1\) itself, which is fixed).
Once we take the root \(\omega=e^{2\pi i/p}\) to some other root, \(\omega^n\), this determines the automorphism, as all other roots are powers of \(\omega\).
The Galois group is thus given by automorphisms \(\phi_n(\omega)=\omega^n\), which clearly commute, forming the cyclic group of order \(p\).
\end{example}
\begin{problem}{galois:symmetric.polynomials}
Suppose that \(p(x)\) is a polynomial with coefficients in a field \(k\), and with roots in some extension \(K\) of \(k\), say \(\alpha_1,\alpha_2,\dots,\alpha_n\).
Prove that, for any symmetric polynomial \(q\) with coefficients in \(k\), \(q(\alpha_1,\dots,\alpha_n)\) belongs to \(k\).
\end{problem}
\begin{answer}{galois:symmetric.polynomials}
The elementary symmetric polynomials of the roots are the coefficients of \(p(x)\).
All symmetric functions are polynomials in the elementary symmetric polynomials.
A different proof: all of \(\alpha_1,\dots,\alpha_n\) belong to the splitting field of \(p(x)\) inside \(K\).
So we can assume without loss of generality that \(K\) is the splitting field of \(p(x)\).
The Galois group of \(K/k\) acts as permutations of \(\alpha_1,\dots,\alpha_n\), fixing \(k\), so fixes \(q(\alpha_1,\dots,\alpha_n)\).
But \(K\) is a Galois extension of \(k\), so the fixed elements under the Galois group are precisely the elements of \(k\).
\end{answer}
\begin{theorem}
Suppose that \(K\) is the splitting field of an irreducible monic polynomial \(p(x)\) over a field \(k\).
Let \(d\) be the discriminant of \(p(x)\), i.e. the resultant of \(p(x),p'(x)\).
Then \((-1)^{n(n-1)/2}d\) is a square in \(K\), say \(\alpha^2\), so has a square root \(\alpha\) in \(K\).
The field extension \(k(\alpha) \subset K\) has Galois group \(\Gal{k(\alpha)}{K} \subset \Gal{k}{K}\) consisting precisely of the elements of \(\Gal{k}{K}\) which act as even permutations of the roots.
\end{theorem}
\begin{proof}
Problem~\vref{problem:resultants:discriminant.as.product} shows that
\[
d=(-1)^{n(n-1)/2}\prod_{i<j}(x_i-x_j)^2,
\]
so we can take
\[
\alpha=\prod_{i<j}(x_i-x_j),
\]
or any multiple by any square of \(-1\), or the same expression after any permutation of roots.
This \(\alpha\) changes sign when we permute roots, according to the sign of the permutation.

Every automorphism of \(K\) fixing all elements of \(k(\alpha)\) fixes all elements of \(k \subset k(\alpha)\), so \(\Gal{k(\alpha)}{K} \subset \Gal{k}{K}\).
Given an automorphism \(g\) in \(\Gal{k}{K}\), it permutes the roots of \(p(x)\), so acts on \(\alpha\) as the sign of that permutation, so fixes \(\alpha\) just when it is even as a permutation.
But it fixes \(k\) in any case, so fixes \(k(\alpha)\) just when it fixes \(\alpha\).
\end{proof}
\begin{example}
Return to the splitting field \(K\) of \(x^3-2\) over \(k=\Q{}\).
We saw that \(\Gal{k}{K}\) is the symmetric group on three letters.
We compute the discriminant \(\Delta=\Delta_{x^3-2}=108\).
So \(d=(-1)^{3(2)/2}\Delta=-108\), which has no square root in \(k=\Q{}\).
We adjoin a square root to \(-108\), or equivalently to \(-3\) since \(-108=2^2 3^2 (-3)\), so \(\alpha=\sqrt{-3}\), and \(\Gal{k(\alpha)}{K}\) is the alternating group on three letters.
\end{example}

\section{Radical extensions}
Why is there a quadratic equation but not an equation like it to solve higher order polynomial equations in one variable?
A \emph{simple extension}\define{simple!extension}\define{extension!simple} of a field \(k\) is an extension \(K\), denoted \(k(\alpha)\), with an element \(\alpha\) of \(K\), so that every element of \(K\) is a rational function \(b(\alpha)/c(\alpha)\) with coefficients in \(k\).
A \emph{simple radical extension}\define{simple!radical extension}\define{extension!simple radical} of a field \(k\) is an extension \(k(\alpha)\) for which \(\alpha\) satisfies an equation \(\alpha^n=c\) for some element \(c\) of \(k\); we usually write such an extension as \(k(\sqrt[n]{c})\).
A \emph{radical extension}\define{radical extension}\define{extension!radical} is the result of repeated simple radical extensions \(k(\alpha_1)(\alpha_2)\dots(\alpha_n)\).
\begin{example}
The quadratic formula
\[
x=\frac{-b\pm\sqrt{b^2-4c}}{2},
\]
for a quadratic equation \(0=x^2+bx+c\) (over any field \(k\) not of characteristic \(2\)) puts the two roots into the simple radical extension field \(K=k(\sqrt{b^2-4c})\).
In terms of elementary symmetric polynomials, we can take variables \(t_1,t_2\) to represent roots, and then our polynomial equation is \(0=x^2-e_1x+e_2\).
Our quadratic formula tells us that
\[
k(t_1,t_2)=k(e_1,e_2)(\sqrt{e_1^2-4e_2}).
\]
\end{example}
\begin{example}
Any cubic equation of the form 
\[
0=t^3+pt+q
\]
(called a \emph{depressed cubic}\define{depressed cubic}) has the solutions
\[
C-\frac p{3C}\quad\text{with}\quad C=\sqrt[3]{-\frac q2+\sqrt{\frac{q^2}4+\frac{p^3}{27}}}
\]
over any field \(k\) not of characteristic \(2\) or \(3\), for any choice of square and cube roots in a suitable radical extension.
\end{example}
\begin{example}
We can ``depress'' a general cubic
\[
0=x^3-e_1x^2+e_2x-e_3
\]
as
\[
x=t+\frac{e_1}{3}
\]
so that \(t\) satisfies the depressed cubic with
\begin{align*}
p&=\frac{3e_2-e_1^2}{3},\\
q&=\frac{-2e_1^2+9e_1e_2-27e_3}{27}.
\end{align*}
This gives a complicated expression for the roots of the general cubic in terms of the coefficients, but one that lies in a radical extension
\[
k(t_1,t_2,t_3)=
k(e_1,e_2,e_3,\alpha,\beta_1,\dots,\beta_6)
\]
where \(\alpha\) is a square root for 
\[
\frac{q^2}4+\frac{p^3}{27}
\]
and the \(\beta_i\) are the \(6\) choices of values of \(C\), for the possible choices of cube roots of 
\[
-\frac{q}{2}\pm\alpha.
\]
\end{example}
\begin{example}
In the same way, any formula to solve a polynomial equation, over a given field \(k\), and of a given order \(n\), by taking coefficients, rational functions thereof, successive square roots, cube roots, \dots, and rational functions thereof, lies in some radical extension of \(k(e_1,\dots,e_n)\).
Since it solves for the roots, that radical extension contains \(k(t_1,\dots,t_n)\).
If it only solves for one root, we can make a radical extension in which we solve for all others by symmetry of the coefficients under permutations.
\end{example}
\begin{example}
The extension \(\Q{}(x,y)/\Q{}\) (adding two abstract variables) admits an automorphism \(x \mapsto y\), \(y \mapsto x\).
But the further extension \(\Q{}(\sqrt{x},y)\) doesn't admit this automorphism, since it has no \(\sqrt{y}\).
This is easy to fix: just add a \(\sqrt{y}\).
\end{example}
In the remainder of this section, we assume some familiarity with group theory \cite{Armstrong:1988,Bogopolski:2008,Dummit/Foote:2004,Ledermann:1953}.
\begin{lemma}\label{lemma:extend.Gal}
Suppose that \(K\) is an extension of a field \(k\), and that \(\Gal{k}{K}\) is a finite group.
For every radical extension \(E\) of \(K\), there is a radical extension \(F\) of \(E\) so that every automorphism \(\Gal{k}{K}\) extends to an automorphism on \(F\).
\end{lemma}
\begin{proof}
If we have some element \(\alpha\) of \(K\) and some radical \(\sqrt[m]{\alpha}\), we just need to add some element \(\sqrt[m]{g\alpha}\) for every \(g\) in \(\Gal{k}{K}\), and repeat as needed.
\end{proof}
Consider the equation \(x^n=1\) over some field \(k\).
Its roots in any splitting field are called the \(n^{\text{th}}\) roots of \(1\) or \(n^{\text{th}}\) roots of unity.\define{root of unity}
\begin{lemma}
If \(\omega\) is a \(p^{\text{th}}\) root of \(1\) over a field \(k\), and \(\omega\ne 1\), then the \(p^{\text{th}}\) roots of \(1\) are precisely
\[
1,\omega,\omega^2,\dots,\omega^{p-1}.
\]
\end{lemma}
\begin{proof}
Take a splitting field \(K\) for \(x^p-1\).
If \(\alpha,\beta\) are two roots in \(K\) of this equation, then clearly \(\alpha\beta\) is too.
Clearly \(0\) is not a root, so the roots have reciprocals, which are also roots.
So the roots form an abelian group \(A\) with precisely \(p\) elements, one of which is \(1\) in \(k\).
There might be others in \(k\) as well.
If we take any element of that group, it generates a subgroup.
By Lagrange's theorem, the order of a subgroup divides the order of the group, so this subgroup has order \(1\) or \(p\), i.e. the subgroup is \(\set{1}\) or \(A\).
\end{proof}
\begin{example}
If we allow repeated radical extensions, we might as well extend by a prime root each time, since \(\sqrt[6]{\alpha}\) can be added by first adding \(\beta=\sqrt[2]{\alpha}\) and then adding \(\sqrt[3]{\beta}\).
\end{example}
A simple radical extension \(k(\alpha)/k\) is \emph{tame} if \(\alpha\) is a \(p^{\text{th}}\) root of some element of \(k\), \(p\) prime, and either
\begin{itemize}
\item \(\alpha\) is a \(p^{\text{th}}\) root of \(1\) or
\item every \(p^{\text{th}}\) root of \(1\) in \(k(\alpha)\) already lies in \(k\).
\end{itemize}
\begin{lemma}
Every radical extension arises by a sequence of tame simple radical extensions.
\end{lemma}
\begin{proof}
Replace every simple radical extension by a prime root each time as above.
If introducing some \(\alpha\), not a \(p^{\text{th}}\) root of \(1\), also introduces some \(p^{\text{th}}\) root of \(1\), we can just first introduce that \(p^{\text{th}}\) root of \(1\) and then introduce \(\alpha\).
\end{proof}
\begin{lemma}
The Galois group of any tame simple radical extension is a finite abelian group.
\end{lemma}
\begin{proof}
Take a radical extension \(K=k(\alpha)\) of a field \(k\), say with \(\alpha\) a root of \(x^p=c\), each element the Galois group of \(K/k\) is determined by how it acts on \(\alpha\).
But it has to preserve the equation \(x^p=c\), so moves \(\alpha\) around to some element \(\beta\) with \(\beta^p=c\).
So then \(\beta/\alpha\) is a \(p^{\text{th}}\) root of \(1\).
Suppose that every \(p^{\text{th}}\) root of \(1\) in \(k(\alpha)\) already lies in \(k\), so \(\beta/\alpha\in k\).
We associate to each element \(g\) of the Galois group our root of \(1\): let
\[
\varphi\colon g\in\Gal{k}{K}\mapsto\frac{g\alpha}{\alpha}\in 
\set{\gamma\in k^{\times}|\gamma^p=1}\subseteq
k^{\times}.
\]
We would like to know whether this map is a group morphism.
If it is, it has image in the \(p^{\text{th}}\) roots of \(1\), an abelian group, and kernel trivial, so our Galois group is abelian, and is either 
\begin{itemize}
\item just the identity transformation or
\item the group of all \(p^{\text{th}}\) roots of \(1\), a group with \(p-1\) elements.
\end{itemize}
Note that
\[
\frac{\varphi(gh)}{\varphi(g)\varphi(h)}
=
\frac{\alpha}{h\alpha}g\frac{h\alpha}{\alpha}.
\]
But
\[
\frac{h\alpha}{\alpha}\in k
\]
is invariant under any \(g\) in \(\Gal{k}{K}\).
So \(\varphi\) is a group morphism.

Next suppose instead that \(\alpha\) is \(p^{\text{th}}\) root of \(1\).
We can assume that \(\alpha\) is a primitive \(p^{\text{th}}\) root of \(1\).
Every element \(g\) of \(\Gal{k}{K}\) takes \(\alpha\) to some other \(p^{\text{th}}\) root of \(1\), and these are of course just powers \(\alpha^i\).
So another, say \(h\) with \(h\alpha=\alpha^j\), has \(gh\alpha=hg\alpha=\alpha^{ij}\).
Since these automorphisms are determined by how they act on \(\alpha\), they commute.
\end{proof}
\begin{corollary}
The Galois group of any radical extension is a finite solvable group.
\end{corollary}
\begin{proof}
Any radical extension consists of repeated tame simple radical extensions, with abelian Galois groups, hence repeated extensions by abelian groups, so solvable.
The problem is to relate the Galois groups.
Imagine two steps of our extension: we extend \(k\subset k(\alpha) \subset k(\alpha,\beta)\subset\dots\).
We claim that \(k(\alpha)\) is invariant under \(\Gal{k}{k(\alpha,\beta)}\).

If \(\alpha\) is a \(p^{\text{th}}\) root of \(1\), then every \(g\in\Gal{k}{k(\alpha,\beta)}\) takes \(\alpha\) to some other \(p^{\text{th}}\) root of \(1\), but these are all just powers \(\alpha^i\), so already lie in \(k(\alpha)\).
Hence \(k(\alpha)\) is invariant under \(\Gal{k}{k(\alpha,\beta)}\).

If \(\alpha\) is not a \(p^{\text{th}}\) root of \(1\), then every \(p^{\text{th}}\) root of \(1\) in \(k(\alpha)\) already lies in \(k\).
Take any \(g\in\Gal{k}{k(\alpha,\beta)}\).
Then \(g\alpha/\alpha\) is a \(p^{\text{th}}\) root of \(1\).
But we have no reason why is belongs to \(k(\alpha)\).

We can assume that when we carry out our repeated tame extensions, we add all of the roots of unity first, and then all of the other roots later.
So we can suppose that \(g\alpha/\alpha\) is in \(k\), our earlier field.
Therefore \(g\alpha=(g\alpha/\alpha)\alpha\) is in \(k(\alpha)\).
Hence \(k(\alpha)\) is invariant under \(\Gal{k}{k(\alpha,\beta)}\).

So \(\Gal{k}{k(\alpha,\beta)}\) acts on \(k,k(\alpha),k(\alpha,\beta)\), each contained in the next, acting on \(k(\alpha)\) by a morphism of groups 
\[
\Gal{k}{k(\alpha,\beta)}\to\Gal{k}{k(\alpha)}
\]
with kernel precisely \(\Gal{k(\alpha)}{k(\alpha,\beta)}\), an exact sequence of groups
\[
1\to\Gal{k(\alpha)}{k(\alpha,\beta)}\to\Gal{k}{k(\alpha,\beta)}\to\Gal{k}{k(\alpha)}\to 1.
\]
The sequence starts and ends abelian, so the middle is solvable.
Repeating this argument, at each step we extend a solvable by another abelian.
\end{proof}
We want to prove that there is no analogue of the quadratic equation for high degree polynomials.
\begin{theorem}
The general equation of degree \(5\) or more is not solvable in radicals.
In other words, the radical extensions of \(k(e_1,\dots,e_n)\), for any field \(k\), do not contain any extension field isomorphic to \(k(t_1,\dots,t_n)\), where \(e_1,\dots,e_n\) are the elementary symmetric polynomials in variables \(t_1,\dots,t_n\).
\end{theorem}
\begin{proof}
Write \(e=(e_1,\dots,e_n)\) and \(t=(t_1,\dots,t_n)\).
Radical extensions have solvable Galois group.
We can make the extensions so that they are Galois extensions.
Suppose we have a radical extension \(K/k(e)\) which contains \(k(t)\).
By definition \(K\) is also a radical extension of \(k(t)\).
By lemma~\vref{lemma:extend.Gal}, we can extend further if needed so that the Galois group of \(k(t)/k(e)\) extends to lie in the Galois group of \(K/k(e)\).

The extension \(k(t)/k(e)\) is Galois with Galois group the symmetric group (the group of permutations of variables \(t_1,\dots,t_n\)).
We leave the reader to pursue sufficient group theory to prove that the symmetric group of permutations of \(5\) or more variables is not a solvable group (because, for example, it is simple and not abelian).
So the solvable Galois group of \(K/k(e)\) contains the nonsolvable symmetric group.
The reader can, once again, master sufficient group theory to prove that every subgroup of any solvable group is solvable.
\end{proof}
