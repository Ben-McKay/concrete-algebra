\chapter{The integers}\label{chapter:integers}
\epigraph[author={Leopold Kronecker}]{God made the integers; all else is the work of man.}\SubIndex{Kronecker, Leopold}
\section{Notation}
We will write numbers using notation like \(\num{1234567.12345}\), using a decimal point \({.}\) at the last integer digit, and using thin spaces to separate out every 3 digits before or after the decimal point.
You might prefer \(\num[group-separator={,},output-decimal-marker={\cdot}]{1234567.12345}\) or \(\num[group-separator={,}]{1234567.12345}\), which are also fine.
We reserve the \(\cdot\) symbol for multiplication, writing \(2 \cdot 3=6\) rather than \(2 \times 3 = 6\).

\section{The laws of integer arithmetic}
The integers are the numbers \(\dots, -2, -1, 0, 1, 2, \dots\).
Let us distill their essential properties, using only the concepts of addition and multiplication.
\smallskip
\begin{itemize}
\item[]\emph{Addition laws:}\define{addition laws}
\begin{enumerate}
\item The associative law:\define{associative law!addition} For any integers \(a, b, c\): \((a+b)+c=a+(b+c)\).
\item The identity law:\define{identity law!addition} There is an integer \(0\) so that for any integer \(a\): \(a+0=a\).
\item The existence of negatives:\define{existence of negatives} for any integer \(a\): there is an integer \(b\) (denoted by the symbol \(-a\)) so that \(a+b=0\).
\item The commutative law:\define{commutative law!addition} For any integers \(a, b\): \(a+b=b+a\).
\end{enumerate}
\smallskip
\item[]\emph{Multiplication laws:}\define{multiplication laws}
\begin{enumerate}
\item The associative law:\define{associative law!multiplication} For any integers \(a, b, c\): \((ab)c=a(bc)\).
\item The identity law:\define{identity law!multiplication} There is an integer \(1\) so that for any integer \(a\): \(a1=a\).
\item The zero divisors law: For any integers \(a,b\): if \(ab=0\) then \(a=0\) or \(b=0\).
\item The commutative law:\define{commutative law!multiplication} For any integers \(a, b\): \(ab=ba\).
\end{enumerate}
\smallskip
\item[]\emph{The distributive law:}\define{distributive law}
\begin{enumerate}
\item For any integers \(a, b, c\): \(a(b+c)=ab+ac\).
\end{enumerate}
\smallskip
\item[]\emph{Sign laws:}\define{sign laws} \smallskip \\ 
Certain integers are called \emph{positive}.\define{positive}
\begin{enumerate}
\item The succession law:\define{succession} An integer \(b\) is positive just when either \(b=1\) or \(b=c+1\) for a positive integer \(c\).
\item Determinacy of sign:\define{determinacy of sign} Every integer \(a\) has precisely one of the following properties: \(a\) is positive, \(a=0\), or \(-a\) is positive.
\end{enumerate}
We write \(a<b\) to mean that there is a positive integer \(c\) so that \(b=a+c\). 
\smallskip
\item[]\emph{The law of well ordering:}\define{law of well ordering}\define{well ordering}
\begin{enumerate}
\item
Any nonempty collection of positive integers has a least element; that is to say, an element \(a\) so that every element \(b\) satisfies \(a<b\) or \(a=b\).
\end{enumerate}
\end{itemize}
\smallskip\noindent{}All of the other arithmetic laws we are familiar with can be derived from these.
For example, the associative law for addition, applied twice, shows that \((a+b)+(c+d)=a+(b+(c+d))\), and so on, so that we can add up any finite sequence of integers, in any order, and get the same result, which we write in this case as \(a+b+c+d\).
A similar story holds for multiplication.

Of course, we write \(1+1\) as \(2\), and \(1+1+1\) as \(3\) and so on.
Write \(a>b\) to mean \(b<a\).
Write \(a\le b\) to mean \(a<b\) or \(a=b\). 
Write \(a\ge b\) to mean \(b \le a\).
Write \(|a|\)\define{absolute value} to mean \(a\), if \(a \ge 0\), and to mean \(-a\) otherwise, and call it the \emph{absolute value}\define{absolute value} of \(a\).
An integer \(a\) is \emph{negative}\define{negative} if \(-a\) is positive.

\vspace{2cm}

To understand mathematics, you have to solve a large number of problems.
\epigraph[author={Frederick Douglass, statesman and escaped slave}]{I prayed for twenty years but received no answer until I prayed with my legs.}\SubIndex{Douglass, Frederick}
\begin{problem}{integers:use.laws}
For each equation below, what law above justifies it?
\begin{enumerate}
\item \(7(3+1)=7 \cdot 3 + 7 \cdot 1\)
\item \(4(9 \cdot 2)=(4 \cdot 9)2\)
\item \(2 \cdot 3=3 \cdot 2\)
\end{enumerate}
\end{problem}
\begin{answer}{integers:use.laws}
\begin{enumerate}
\item distributive
\item associative multiplication
\item commutative multiplication
\end{enumerate}
\end{answer}
\begin{problem}{integers:right.distrib}
Use the laws above to prove that for any integers \(a, b, c\): \((a+b)c=ac+bc\).
\end{problem}
\begin{answer}{integers:right.distrib}
By the commutative law of multiplication, \((a+b)c=c(a+b)\).
By the distributive law, \(c(a+b)=ca+cb\).
By two applications of the commutative law, \(ca+cb=ac+cb=ac+bc\).
\end{answer}
\begin{problem}{integers:zero.plus.zero}
Use the laws above to prove that \(0+0=0\).
\end{problem}
\begin{answer}{integers:zero.plus.zero}
For any integer \(a\): \(a+0=a\). Pick \(a\) to be \(a=0\).
\end{answer}
\begin{problem}{integers:zero.times.zero}
Use the laws above to prove that \(0 \cdot 0=0\).
\end{problem}
\begin{answer}{integers:zero.times.zero}
\begin{twocolumnproof}
\pf{0}{0 + 0}[problem \ref{problem:integers:zero.plus.zero}] \\
\pf{0 \cdot 0}{0 \cdot (0 + 0)}[multiplying by \(0\)] \\
\pf{0 \cdot 0}{0 \cdot 0 + 0 \cdot 0}[the distributive law] \\
\pf{\text{Let } b}{-(0 \cdot 0)} \\
\pf{0 \cdot 0 + b}{(0 \cdot 0 + 0 \cdot 0) + b}[adding \(b\) to both sides] \\
\pf{0 \cdot 0 + b}{0 \cdot 0 + (0 \cdot 0 + b)}[the associative law for addition] \\
\pf{0}{0 \cdot 0 + 0}[the definition of \(b\)] \\
\lastpf{0}{0 \cdot 0}[the definition of \(0\)] \\
\end{twocolumnproof}
\end{answer}
\begin{problem}{integers:any.times.zero}
Use the laws above to prove that, for any integer \(a\): \(a \cdot 0=0\).
\end{problem}
\begin{answer}{integers:any.times.zero}
\begin{twocolumnproof}
\pf{0}{0 + 0}[problem \ref{problem:integers:zero.plus.zero}] \\
\pf{a \cdot 0}{a \cdot (0 + 0)}[multiplying by \(a\)] \\
\pf{a \cdot 0}{a \cdot 0 + a \cdot 0}[the distributive law] \\
\pf{\text{Let } b}{-(a \cdot 0)} \\
\pf{a \cdot 0 + b}{(a \cdot 0 + a \cdot 0) + b}[adding \(b\) to both sides] \\
\pf{a \cdot 0 + b}{a \cdot 0 + (a \cdot 0 + b)}[the associative law for addition] \\
\pf{0}{a \cdot 0 + 0}[the definition of \(b\)] \\
\lastpf{0}{a \cdot 0}[the definition of \(0\)]
\end{twocolumnproof}
\end{answer}
\begin{problem}{integers:unique.minus}
Use the laws above to prove that, for any integer \(a\): there is exactly one integer \(b\) so that \(a+b=0\); of course, we call this \(b\) by the name \(-a\).
\end{problem}
\begin{answer}{integers:unique.minus}
There is at least one such \(b\), by the existence of negatives.
Suppose that \(a+b=0\) and that \(a+c=0\).
\begin{twocolumnproof}
\pf{(a+b)+c}{0+c} \\
\pf{}{c+0}[commutativity of addition] \\
\pf{}{c}[the definition of \(0\)] \\
\pf{}{(a+b)+c}[returning to the start again] \\
\pf{}{a+(b+c)}[associativity of addition] \\
\pf{}{a+(c+b)}[commutativity of addition] \\
\pf{}{(a+c)+b}[associativity of addition] \\
\pf{}{0+b}[the definition of \(c\)] \\
\pf{}{b+0}[commutativity of addition] \\
\lastpf{}{b}[the definition of \(0\)]
\end{twocolumnproof}
Hence \(b=c\).
\end{answer}
\begin{problem}{integers:minus.one.times}
Use the laws above to prove that, for any integer \(a\): \((-1)a=-a\).
\end{problem}
\begin{answer}{integers:minus.one.times}
\begin{twocolumnproof}
\pf{a+(-1)a}{a+a(-1)}[commutativity of multiplication] \\
\pf{}{a(1+(-1))}[the distributive law on the right hand side] \\
\pf{}{a(0)}[the definition of \(-\)] \\
\lastpf{}{0}[the definition of \(0\)] 
\end{twocolumnproof}
So \((-1)a\) fits the definition of \(-a\).
By problem~\vref{problem:integers:unique.minus}, \((-1)a=-a\).
\end{answer}
\begin{problem}{integers:min.min}
Use the laws above to prove that \((-1)(-1)=1\).
\end{problem}
\begin{answer}{integers:min.min}
By problem~\vref{problem:integers:minus.one.times}, \((-1)(-1)=-(-1)\) is the unique integer which, added to \(-1\), gives zero, and we know that this integer is \(1\).
\end{answer}
\begin{problem}{integers:abs.product}
Use the laws above to prove that, for any integers \(b,c\): \(|bc|=|b||c|\).
\end{problem}
\begin{problem}{integer:unique.zero}
Our laws ensure that there is an integer \(0\) so that \(a+0=0+a=a\) for any integer \(a\).
Could there be two different integers, say \(p\) and \(q\), so that \(a+p=p+a=a\) for any integer \(a\), and also so that \(a+q=q+a=a\) for any integer \(a\)?
(Roughly speaking, we are asking if there is more than one integer which can ``play the role'' of zero.)
\end{problem}
\begin{problem}{integer:unique.one}
Our laws ensure that there is an integer \(1\) so that \(a \cdot 1=1 \cdot a=a\) for any integer \(a\).
Could there be two different integers, say \(p\) and \(q\), so that \(ap=pa=a\) for any integer \(a\), and also so that \(aq=qa=a\) for any integer \(a\)?
\end{problem}
\begin{theorem}[The equality cancellation law for addition\define{equality cancellation law!addition}]
Suppose that \(a, b\) and \(c\) are integers and that \(a+c=b+c\).
Then \(a=b\).
\end{theorem}
\begin{proof}
By the existence of negatives, there is a integer \(-c\) so that \(c+(-c)=0\).
Clearly \((a+c)+(-c)=(b+c)+(-c)\).
Apply the associative law for addition: \(a+(c+(-c))=b+(c+(-c))\), so \(a+0=b+0\), so \(a=b\). 
\end{proof}
\begin{problem}{integers:m.b.p.c}
Prove that \(-(b+c)=(-b)+(-c)\) for any integers \(b,c\).
\end{problem}
\begin{answer}{integers:m.b.p.c}
We need to see that \((-b)+(-c)\) has the defining property of \(-(b+c)\): adding to \(b+c\) to give zero.
\begin{twocolumnproof}
\pf{(b+c)+((-b)+(-c))}{b+c+(-b)+(-c)}[as above: parentheses not needed] \\
\pf{}{b+(-b)+c+(-c)}[commutativity of addition] \\
\pf{}{0+0}[existence of negatives] \\
\lastpf{}{0}[the definition of \(0\)]
\end{twocolumnproof}
\end{answer}
\begin{problem}{integers:unique.predecessor}
A \emph{predecessor}\define{predecessor} of an integer \(b\) is an integer \(c\) so that \(b=c+1\).
Prove that every integer has a unique predecessor.
\end{problem}
\begin{answer}{integers:unique.predecessor}
If \(b=c+1=d+1\) for two integers \(c,d\), then \(c+1=d+1\) so \((c+1)+(-1)=(d+1)+(-1)\), where the existence of negatives ensures that there is a negative \(-1\) of \(1\).
So \(c+(1+(-1))=d+(1+(-1))\) by the associative law for addition.
So \(c+0=d+0\), by the existence of negatives law.
So \(c=d\), by the identity law for addition.
\end{answer}
We haven't mentioned subtraction\define{subtraction} yet.
\begin{problem}{integers:define.subtraction}
Suppose that \(a\) and \(b\) are integers.
Prove that there is a unique integer \(c\) so that \(a=b+c\).
Of course, from now on we write this integer \(c\) as \(a-b\).
\end{problem}
\begin{problem}{integers:minus.and.subtraction}
Prove that, for any integers \(a,b\), \(a-b=a+(-b)\).
\end{problem}

\begin{problem}{integers:subtraction.distributive}
Prove that subtraction \emph{distributes over multiplication}: for any integers \(a,b,c\), \(a(b-c)=ab-ac\).
\end{problem}

\begin{problem}{integers:define.subtraction.order}
Prove that any two integers \(b\) and \(c\) satisfy just precisely one of the conditions \(b>c\), \(b=c\), \(b<c\).
\end{problem}
\begin{theorem} The equality cancellation law for multiplication:\define{equality cancellation law!multiplication} for an integers \(a,b,c\) if \(ab=ac\) and \(a\ne 0\) then \(b=c\).
\end{theorem}
\begin{proof}
If \(ab=ac\) then \(ab-ac=ac-ac=0\).
But \(ab-ac=a(b-c)\) by the distributive law.
So \(a(b-c)=0\). By the zero divisors law, \(a=0\) or \(b=c\).
\end{proof}
\begin{problem}{integers:two.by.two}
We know how to add and multiply \(2 \times 2\) matrices with integer entries.
Of the various laws of addition and multiplication and signs for integers, which hold true also for such matrices?
\end{problem}
\begin{problem}{integers:successor}
Use the laws above to prove that the sum of any two positive integers is positive.
\end{problem}
\begin{answer}{integers:successor}
Suppose that there are positive integers \(b,c\) with \(b+c\) not positive.
By the law of well ordering, there is a least possible choice of \(b\), and, for that given \(b\), a least possible choice of \(c\).
If \(b=1\) then \(b+c=1+c=c+1\) is positive by the succession law.
If \(b \ne 1\), then \(b=d+1\) for some positive \(d\) by the succession law, and then \(b+c=(d+1)+c=(d+c)+1\).
This is not positive, so by the succession law, \(d+c\) is not positive.
But then \(d\) is smaller than \(b\), so \(b\) is not the least possible choice.
\end{answer}
\begin{problem}{integers:pos.times.pos}
Use the laws above to prove that the product of any two positive integers is positive.
\end{problem}
\begin{answer}{integers:successor}
Suppose that there are positive integers \(b,c\) with \(bc\) not positive.
By the law of well ordering, there is a least possible choice of \(b\), and, for that given \(b\), a least possible choice of \(c\).
If \(b=1\) then \(bc=c\) is positive.
If \(b \ne 1\), then \(b=d+1\) for some positive \(d\) by the succession law, and then \(bc=(d+1)c=dc+c\).
This is not positive, so by problem~\vref{problem:integers:successor}, \(dc\) is not positive.
But then \(d\) is smaller than \(b\), so \(b\) is not the least possible choice.
\end{answer}
\begin{problem}{integers:pos.product}
Use the laws above to prove that the product of any two integers is positive just when (1) both are positive or (2) both are negative.
\end{problem}
\begin{answer}{integers:pos.product}
If \(b,c >0\) or if \(b,c < 0\), we know that \(bc>0\).
If \(b>0\) and \(c<0\), say \(c=-d\) with \(d>0\), then \(bc=b(-d\).
If we add \(bd+b(-d))=b(d+(-d))=b0=0\).
So \(b(-d)=-(bd)\) is negative.
Similarly, if \(b<0\) and \(c>0\), \(bc=cb\) is negative.
\end{answer}
\begin{problem}{integers:neg.times.neg}
Use the laws above to prove that the product of any two negative integers is positive.
\end{problem}
\begin{problem}{integers:cancel.plus}
Prove the inequality cancellation law for addition:\define{inequality cancellation law!addition} For any integers \(a, b, c\): if \(a+c<b+c\) then \(a<b\).
\end{problem}
\begin{answer}{integers:cancel.plus}
\begin{twocolumnproof}
\pf{(b+c)-(a+c)}{b+c+(-(a+c))}[\(d-e=d+(-e)\) as above] \\
\pf{}{b+c+(-a)+(-c)}[\(-(d+e)=(-d)+(-e)\) as above] \\
\pf{}{b+c+(-c)+(-a)}[commutativity of addition] \\
\pf{}{b+0+(-a)}[existence of negatives] \\
\pf{}{b+(-a)}[definition of zero] \\
\lastpf{}{b-a}[\(d+(-e)=d-e\) as above]
\end{twocolumnproof}
So \(a+c < b+c\) just when \(a<b\).
\end{answer}
\begin{problem}{integers:cancel.times}
Prove the inequality cancellation law for multiplication:\define{inequality cancellation law!multiplication} For any integers \(a, b, c\): if \(a<b\) and if \(0<c\) then \(ac<bc\).
\end{problem}
\begin{answer}{integers:cancel.times}
By distributivity of subtraction, proven in problem~\vref{problem:integers:subtraction.distributive}, \(bc-ac=(b-c)a\).
Apply the solution of problem~\vref{problem:integers:pos.product}.
\end{answer}
\begin{problem}{integers:bounded}
Suppose that \(S\) is a set of integers.
A \emph{lower bound}\define{lower bound} on \(S\) is an integer \(b\) so that \(b\le c\) for every integer \(c\) from \(S\); if \(S\) has a lower bound, \(S\) is \emph{bounded from below}.\define{bounded from below}
Prove that a nonempty set of integers bounded from below contains a least element.
\end{problem}

\section{Division of integers}
\epigraph[author={Lewis Carroll}, source={Through the Looking Glass}]{Can you do Division?  Divide a loaf by a knife---what's the answer to that?}\SubIndex{Carroll, Lewis}\SubIndex{Through the Looking Glass}
We haven't mentioned division\define{division} yet.
\emph{Danger:} although \(2\) and \(3\) are integers, 
\[
\frac{3}{2}=1.5
\]
is \emph{not} an integer.
\begin{problem}{integers:define.division}
Suppose that \(a\) and \(b\) are integers and that \(b \ne 0\).
Prove from the laws above that there is \emph{at most} one integer \(c\) so that \(a=bc\).
Of course, from now on we write this integer \(c\) as \(\frac{a}{b}\) or \(a/b\).
\end{problem}
\begin{problem}{integers:division.by.zero}
\emph{Danger:} why \emph{can't} we divide by zero\SubIndex{division!by zero}?
\end{problem}
\begin{answer}{integers:division.by.zero}
Dividing a nonzero integer \(a\) by zero would mean find an integer \(c\) so that \(a=0 \, c\).
But we have seen that \(0 \, c=0\), so \(a=0\), a contradiction.
Dividing zero by zero would mean find an integer \(c\) so that \(0=0 \, c\).
But any integer \(c\) satisfies this equation, so there is no way to pick out one value \(c\) to be \(0/0\).
\end{answer}
We already noted that \(3/2\) is \emph{not} an integer.
At the moment, we are trying to work with integers only.
An integer \(b\) \emph{divides} an integer \(c\) if \(c/b\) is an integer; we also say that \(b\) is a \emph{divisor}\define{divisor} of \(c\).
\begin{problem}{integers:divide.zero}
Explain why every integer divides into zero.
\end{problem}
\begin{problem}{integers:minus.divide}
Prove that, for any two integers \(b\) and \(c\), the following are equivalent:
\begin{enumerate}
\item 
\(b\) divides \(c\),
\item 
\(-b\) divides \(c\),
\item 
\(b\) divides \(-c\),
\item 
\(-b\) divides \(-c\),
\item 
\(|b|\) divides \(|c|\).
\end{enumerate}
\end{problem}
\begin{proposition}\label{proposition:divisors.smaller}
Take any two integers \(b\) and \(c\).
If \(b\) divides \(c\), then \(|b| < |c|\) or \(b=c\) or \(b=-c\).
\end{proposition}
\begin{proof}
By the solution of the last problem, we can assume that \(b\) and \(c\) are positive.
If \(b=c\) the proposition holds, so suppose that \(b>c\); write \(b=c+k\) for some \(k>0\).
Since \(b\) divides \(c\), say \(c=qb\) for some integer \(q\).
Since \(b>0\) and \(c>0\), \(q>0\) by problem~\vref{problem:integers:pos.product}.
If \(q=1\) then \(c=b\), so the proposition holds.
If \(q>1\), then \(q=n+1\) for some positive integer \(n\).
But then \(c=qb=(n+1)(c+k)=nc+c+nk+k\).
Subtract \(c\) from both sides: \(0=nc+nk+k\), a sum of positive integers, hence positive (by problem~\vref{problem:integers:pos.times.pos}) a contradiction.
\end{proof}
\begin{theorem}[Euclid\SubIndex{Euclid}]
Suppose that \(b\) and \(c\) are integers and \(c \ne 0\).
Then there are unique integers \(q\) and \(r\) (the \emph{quotient}\define{quotient} and \emph{remainder}\define{remainder}) so that \(b=qc+r\) and so that \(0 \le r < |c|\).
\end{theorem}
\begin{proof}
To make our notation a little easier, we can assume (by perhaps changing the signs of \(c\) and \(q\) in our statement above) that \(c>0\).

Consider all integers of the form \(b-qc\), for various integers \(q\).
If we were to take \(q=-|b|\), then \(b-qc = b+|b| + (c-1)|b| \ge 0\).
By the law\SubIndex{law of well ordering} of well ordering, since there is an integer of the form \(b-qc \ge 0\), there is a smallest integer of the form \(b-qc \ge 0\); call it \(r\).
If \(r \ge c\), we can replace \(r=b-qc\) by \(r-c=b-(q+1)c\), and so \(r\) was not smallest.

We have proven that we can find a quotient \(q\) and remainder \(r\).
We need to show that they are unique.
Suppose that there is some other choice of integers \(Q\) and \(R\) so that \(b=Qc+R\) and \(0 \le R < c\).
Taking the difference between \(b=qc+r\) and \(b=Qc+R\), we find \(0=(Q-q)c+(R-r)\).
In particular, \(c\) divides into \(R-r\).
Switching the labels as to which ones are \(q,r\) and which are \(Q,R\), we can assume that \(R \ge r\).
So then \(0 \le r \le R < c\), so \(R-r < c\).
By proposition~\vref{proposition:divisors.smaller}, since \(0 \le R-r < c\) and \(c\) divides into \(R-r\), we must have \(R-r=0\), so \(r=R\).
Plug into \(0=(Q-q)c+(R-r)\) to see that \(Q=q\).
\end{proof}
\begin{example}
How do we actually find quotient and remainder, by hand?
Long division:
\integerLongDivision{249}{17}
So if we start with \(b=249\) and \(c=17\), we carry out long division to find the quotient \(q=14\) and the remainder \(r=11\).
\end{example}
\begin{problem}{integers:quotient.and.remainder}
Find the quotient and remainder for \(b,c\) equal to:
\begin{enumerate}
\item \(-180, 9\)
\item \(-169, 11\)
\item \(-982, -11\)
\item \(279, -11\)
\item \(247, -27\)
\end{enumerate}
\end{problem}

\section{The greatest common divisor}
A \emph{common divisor}\define{common divisor} of some integers is an integer which divides them all.
\begin{problem}{integers:gcd.exists}
Given any collection of one or more integers, not all zero, prove that they have a \emph{greatest common divisor}\define{greatest common divisor}, i.e. a largest positive integer divisor.
\end{problem}
\begin{answer}{integers:gcd.exists}
Pick any nonzero number, say \(m\), from the collection.
By proposition~\vref{proposition:divisors.smaller}, \(m\) is not divisible by any integer larger than \(|m|\).
So all of the positive integer divisors are between \(1\) and \(|m|\).
Put these integers in a bag.
Repeatedly throw out the largest integer in our bag which is not a common divisor of the collection.
But \(1\) is a common divisor, so we can't throw them all out.
By induction, eventually we get a greatest common divisor, unique because it is the greatest.
\end{answer}
Denote the greatest common divisor of integers \(m_1, m_2, \dots, m_n\) as
\[
\gcd{m_1,m_2,\dots,m_n}.%
\Notation{gcd(m1,m2,...,mn)}{\gcd{m_1,m_2,\dots,m_n}}{greatest common divisor}
\]
If the greatest common divisor of two integers is \(1\), they are \emph{coprime}.\define{coprime}
We will also define \(\gcd{0,0,\dots,0}\defeq 0\).
\begin{lemma}
Take two integers \(b, c\) with \(c\ne 0\) and compute the quotient \(q\) and remainder \(r\), so that \(b=qc+r\) and \(0 \le r < |c|\).
Then the greatest common divisor of \(b\) and \(c\) is the greatest common divisor of \(c\) and \(r\).
\end{lemma}
\begin{proof}
Any divisor of \(c\) and \(r\) divides the right hand of the equation \(b=qc+r\), so it divides the left hand side, and so divides \(b\) and \(c\).
By the same reasoning, writing the same equation as \(r=b-qc\), we see that any divisor of \(b\) and \(c\) divides \(c\) and \(r\).
\end{proof}
\begin{example}
This makes very fast work of finding the greatest common divisor by hand.
For example, if \(b=249\) and \(c=17\) then we found that \(q=14\) and \(r=11\), so \(\gcd{249,17}=\gcd{17,11}\).
Repeat the process: taking \(17\) and dividing out \(11\), the quotient and remainder are \(1,6\), so \(\gcd{17,11}=\gcd{11,6}\).
Again repeat the process: the quotient and remainder for \(11,6\) are \(1,5\), so \(\gcd{11,6}=\gcd{6,5}\).
Again repeat the process: the quotient and remainder for \(11,6\) are \(1,5\), so \(\gcd{11,6}=\gcd{6,5}\).
In more detail, we divide the smaller integer (smaller in absolute value) into that the larger:
\integerLongDivision{249}{17}
Throw out the larger integer, \(249\), and replace it by the remainder, \(11\), and divide again:
\integerLongDivision{17}{11}
Again we throw out the larger integer (in absolute value), \(17\), and replace with the remainder, \(6\), and repeat:
\integerLongDivision{11}{6}
and again:
\integerLongDivision{6}{5}
and again:
\integerLongDivision{5}{1}
The remainder is now zero.
The greatest common divisor is therefore \(1\), the final nonzero remainder: 
\(\gcd{249,17}=1\).
\end{example}
This method to find the greatest common divisor is called the \emph{Euclidean algorithm}.\define{Euclidean algorithm}\define{algorithm!Euclidean}
\begin{example}
If the final nonzero integer is negative, just change its sign to get the greatest common divisor.
\[
\gcd{-4,-2}=\gcd{-2,0}=\gcd{-2}=2.
\]
\end{example}
\begin{problem}{integers:find.gcd}
Find the greatest common divisor of
\begin{enumerate}
\item \(4233, 884\)
\item \(-191, 78\)
\item \(253, 29\)
\item \(84, 276\)
\item \(-92, 876\)
\item \(147, 637\)
\item \(\num{266664}, \num{877769}\) %(answer: \num{11111})
\end{enumerate}
\end{problem}
\begin{answer}{integers:find.gcd}
\begin{enumerate}
\item \(\sage{gcd(4233, 884)}\)
\item \(\sage{gcd(-191, 78)}\)
\item \(\sage{gcd(253, 29)}\)
\item \(\sage{gcd(84, 276)}\)
\item \(\sage{gcd(-92, 876)}\)
\item \(\sage{gcd(147, 637)}\)
\item \(\sage{gcd(266664, 877769)}\).
\end{enumerate}
\end{answer}
\begin{problem*}{integers:gcd.least.sum}
Take a nonempty collection of integers, at least one of which is not zero.
Allow yourself to ``build new integers'' by adding or subtracting the ones you have from one another.
Prove that the greatest common divisor of the collection is the smallest positive integer you can ``build''.
\end{problem*}
\begin{answer}{integers:gcd.least.sum}
By repeated addition, you can build any positive integer multiple of any integer from your collection.
By subtracting an integer from your collection from itself,  you can build zero.
By repeated subtraction, you can build any negative integer multiple of any integer from your collection.
So by adding up, you can build any linear combination \(s_1m_1+s_2m_2+\dots+s_nm_n\) of integer multiples \(s_1,s_2,\dots,s_n\) of any elements \(m_1,m_2,\dots,m_n\) from the collection,  with any integers \(s_1,s_2,\dots,s_n\).
If all integers in our collection are divisible by some integer \(d\), then clearly so are any integer multiples or finite sums of integer multiples.
So the greatest common divisor divides all such sums.
Hence the greatest common divisor is unchanged if we replace the collection by the collection of all such sums.
So we can suppose that any such sum is already in our collection.
Our collection contains some positive integer, because if \(m_1\) is in our collection, then taking \(s_1\) to be \(1\) if \(m_1>0\) and \(-1\) if \(m_1<0\), \(s_1m_1\) is positive.
By well ordering, there is a least positive element \(c\) in our collection.
Take quotient and remainder of any element \(m\) in our collection by \(c\): \(m=qc+r\).
The remainder \(r=m-qc\) is also in the collection, not negative, but smaller than \(c\), and therefore is zero.
So \(c\) is a common divisor of every element in the collection.
Since \(d\) is the greatest common divisor, \(c\le d\).
But \(c\) is in the collection, and \(d\) divides everyhing in the collection, so \(d\) divides \(c\), so \(d\le c\).
Hence \(c=d\), i.e. the greatest common divisor is the smallest positive integer we can build.
\end{answer}

\section{The least common multiple}
The \emph{least common multiple}\define{least common multiple} of a finite collection of integers \(m_1, m_2, \dots, m_n\) is the smallest positive integer \(\ell=\lcm{m_1, m_2, \dots, m_n}\)\Notation{lcm(m1,m2,...,mn)}{\lcm{m_1,m_2,\dots,m_n}}{least common multiple} so that all of the integers \(m_1, m_2, \dots, m_n\) divide \(\ell\).
\begin{lemma}
The least common multiple of any two integers \(b,c\) (not both zero) is
\[
\lcm{b,c}=\frac{|bc|}{\gcd{b,c}}.
\]
\end{lemma}
\begin{proof}
For simplicity, assume that \(b>0\) and \(c>0\); the cases of \(b\le 0\) or \(c \le 0\) are treated easily by flipping signs as needed and checking what happens when \(b=0\) or when \(c=0\) directly; we leave this to the reader.
Let \(d\defeq \gcd{b,c}\), and factor \(b=Bd\) and \(c=Cd\).
Then \(B\) and \(C\) are coprime.
Write the least common multiple \(\ell\) of \(b,c\) as either \(\ell=b_1 b\) or as \(\ell=c_1 c\), since it is a multiple of both \(b\) and \(c\).
So then \(\ell=b_1 Bd=c_1 C d\).
Cancelling, \(b_1B = c_1 C\).
So \(C\) divides \(b_1B\), but doesn't divide \(B\), so divides \(b_1\), say \(b_1=b_2 C\), so \(\ell=b_1 Bd = b_2 CBd\).
So \(BCd\) divides \(\ell\).
So \((bc)/d=BCd\) divides \(\ell\), and is a multiple of \(b\): \(BCd=(Bd)C=bC\), and is a multiple of \(c\): \(BCd=B(Cd)=Bc\).
But \(\ell\) is the least such multiple.
\end{proof}

\section{Sage}
\epigraph[author={Pablo Picasso}]{Computers are useless. They can only give you answers.}\SubIndex{Picasso, Pablo}%
These lecture notes include optional sections explaining the use of the sage computer algebra system.
At the time these notes were written, instructions to install sage on a computer are at \url{www.sagemath.org}, but you should be able to try out sage on \url{sagecell.sagemath.org} or even create worksheets in sage on \url{cocalc.com} over the internet without installing anything on your computer.
If we type
\begin{sageblock}
gcd(1200,1040)
\end{sageblock}
and press \emph{shift--enter}, we see the greatest common divisor:  \(\sage{gcd(1200,1040)}\).
Similarly type
\begin{sageblock}
lcm(1200,1040)
\end{sageblock}
and press \emph{shift--enter}, we see the least common multiple:
\(\sage{lcm(1200,1040)}\).
Sage can carry out complicated arithmetic operations.
The multiplication symbol is \verb!*!:
\begin{sageblock}
12345*13579
\end{sageblock}
gives \(\sage{12345*13579}\).
Sage uses the expression \verb!2^3! to mean \(2^3\).
You can invent variables in sage by just typing in names for them.
For example, ending each line with \emph{enter}, except the last which you end with \emph{shift--enter} (when you want sage to compute results):
\begin{sageblock}
x=4
y=7
x*y
\end{sageblock}
to print \(\sage{x*y}\).
The expression \verb!15 % 4! 
means the remainder of 15 divided by 4, while \verb!15//4! means the quotient of 15 divided by 4.
We can calculate the greatest common divisor of several integers as
\begin{sageblock}
gcd([3800,7600,1900])
\end{sageblock}
giving us \(\sage{gcd([3800,7600,1900])}\).