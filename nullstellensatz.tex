\chapter{Polynomial equations have solutions}
\epigraph[author={Ahmad al-Fārūqī al-Sirhindī},source={Maktubat}]{If the followers of the prophets err in their proofs, have no fear; the foundation of their knowledge is the authority of the prophets. They provide proofs only for added strength. Following the authority of the prophets is sufficient for them, unlike the philosophers. The philosophers stray from the authority of the prophets and instead rely on their proofs to establish an argument. Misguided, they misguide.}\SubIndex{al-Sirhindī, Ahmad al-Fārūqī}
\epigraph[author={Thomas Paine},source={The Crisis V}]{To argue with a man who has renounced the use and authority of reason, and whose philosophy consists in holding humanity in contempt, is like administering medicine to the dead.}\SubIndex{Paine, Thomas}
\section{Affine space}
For any field \(k\), \emph{affine space}\define{affine space} \(k^n\) is the set of all points
\[
c=
\begin{pmatrix}
c_1 \\
c_2 \\
\vdots \\
c_n
\end{pmatrix}
\]
with coordinates \(c_1, c_2, \dots, c_n\) in \(k\).

We want to say: if you have infinitely many polynomial equations, in a finite set of variables, you can throw away all but finitely many without changing the solutions.
Let us be more precise.
Recall that an \emph{ideal} \(I\) in \(k[x]\) is a collection of  polynomials so that we can add and multiply polynomials in \(I\) without leaving \(I\), and \(I\) is \emph{sticky}: if \(f \in I\) and \(g \in k[x]\) then \(fg \in I\).
Any set \(S\) of polynomials generates an ideal \((S)\): the ideal is the set of all expressions
\[
g_1 f_1 + g_2 f_2 + \dots + g_s f_s
\]
where \(f_1, f_2, \dots, f_s\) is any finite collection of polynomials in \(S\) and \(g_1, g_2, \dots, g_s\) are any polynomials at all in \(k[x]\).
Given any set of polynomials, we can replace that set by the ideal of polynomials it generates inside \(k[x]\) without changing the roots of the polynomials.

\begin{theorem}[Hilbert basis theorem]\label{theorem:Hilbert.basis}\define{Hilbert basis theorem}\define{theorem!Hilbert basis}
For any field \(k\) and any variables \(x=\pr{x_1,x_2,\dots,x_n}\), every ideal \(I\) in \(k[x]\) is finitely generated.
In other words, there are finitely many polynomials 
\[
f_1(x), f_2(x), \dots, f_s(x)
\]
in \(I\) so that the polynomials in \(I\) are precisely those of the form
\[
g_1(x) f_1(x) + g_2(x) f_2(x) + \dots + g_s(x) f_s(x),
\]
for any polynomials \(g_1(x), g_2(x), \dots, g_s(x)\) in \(k[x]\).
\end{theorem}
\begin{proof}
The result is obvious if \(n=0\) since the only ideals in \(k\) are \(k\) and \(0\).
By induction, suppose that the result is true for any number of variables less than or equal to \(n\).
Take an ideal \(I \subset k[x,y]\) where \(x=\pr{x_1,x_2,\dots,x_n}\) and \(y\) is a variable.
Write each element \(f \in I\) as
\[
f(x,y)=f_0(x)+f_1(x)y+\dots+f_d(x)y^d,
\]
so that \(f_d(x)\) is the \emph{leading coefficient}.
Let \(J\) be the set of all leading coefficients of elements of \(I\).
Clearly \(J\) is an ideal in \(k[x]\), so finitely generated by induction.
So we can pick polynomials \(f_1(x,y), f_2(x,y), \dots, f_s(x,y)\) in \(I\) whose leading coefficients generate the ideal \(J\) of all leading coefficients.

Let \(N\) be the maximum degree in \(y\) of any of the \(f_j(x,y)\).
Take some polynomial \(p(x,y)\) from \(I\) of degree greater than \(N\) in \(y\), say with leading coefficient \(\check{p}(x)\).
Express that leading coefficient in terms of the leading coefficients \(\check{f}_j(x)\) of the polynomials \(f_j(x,y)\) as
\[
\check{p}(x)=\sum p_j(x) \check{f}_j(x).
\]
If we raise the \(y\) degree of \(f_j(x,y)\) suitably to match the \(y\) degree of \(p(x,y)\), say by multiplying by \(y^{k_j}\), some positive integer \(k_j\), then
\[
\sum p_j(x) y^{k_j} f_j(x,y)
\]
has leading coefficient the same as \(p(x,y)\), so 
\[
p(x,y)=\sum y^{k_j} p_j(x) f_j(x,y)  + \dots
\]
up to lower order in \(y\).
Inductively we can lower the order in \(y\) of \(p\), replacing \(p\) by the \(\dots\), until we get to order in \(y\) less than \(N\).
At each step, we modify \(p\) by something in the ideal generated by the \(f_j\).
So every polynomial in \(I\) of degree \(N\) or less in \(y\) is in the ideal generated by the \(f_j\).
Add to our list of \(f_j(x,y)\) some additional \(f_j(x,y) \in I\) to span all of the polynomials in \(I\) of degree up to \(N\) in \(y\).
By induction we have generated all of the polynomials in \(I\).
\end{proof}

\begin{theorem}[Weak nullstellensatz]
Given a collection of polynomial equations in variables
\[
x=\pr{x_1, x_2, \dots, x_n}.
\]
over a field \(k\), either there is a point \(x=c\) defined in a finite algebraic extension field of \(k\) where all of these polynomials vanish or else the ideal generated by the polynomials is all of \(k[x]\).
\end{theorem}
\begin{proof}
Follows from theorem~\vref{theorem:Hilbert.basis} and corollary~\vref{corollary:Null}.
\end{proof}

\begin{problem}{nullstellensatz:max.ideal}
Given a point \(c \in k^n\), prove that the ideal 
\[
I_c \defeq \Set{f(x)|f(c)=0}
\]
of functions vanishing at \(x=c\) is a maximal ideal.
Prove that if \(x=p\) and \(x=q\) are distinct points of \(k^n\), then \(I_p \ne I_q\).
\end{problem}

\begin{corollary}
Take variables
\[
x=\pr{x_1, x_2, \dots, x_n}.
\]
over an algebraically closed field \(k\).
Every maximal ideal in \(k[x]\) is the set of all polynomial functions vanishing at some point \(c\), for a unique point \(c\).
\end{corollary}
\begin{proof}
Take any ideal \(I \subset k[x]\)
Take a point \(c\) so that all polynomials in \(I\) vanish there, using theorem~\vref{theorem:nullstellensatz}.
Then \(I \subset I_c\).
\end{proof}
An \emph{affine variety}\define{affine variety}\define{variety!affine} \(X=X_S\) is the collection of zeroes of a set \(S\) of polynomials in variables 
\[
x=\pr{x_1, x_2, \dots, x_n}.
\]
Note that we allow these zeroes to lie in any finite field extension.
If \(I=(S)\) is the ideal generated by \(S\), then clearly \(X_I=X_S\).
Given any set \(X\) of points, let \(I_X\) be the ideal of polynomials vanishing on \(X\).
\begin{theorem}[Nullstellensatz]\label{theorem:nullstellensatz}\define{nullstellensatz}\define{theorem!nullstellensatz}
Take variables
\[
x=\pr{x_1, x_2, \dots, x_n}.
\]
over an algebraically closed field \(k\).
For any ideal \(J\) in \(k[x]\), \(\sqrt{J}=I_X\) where \(X=X_J\).
\end{theorem}
\begin{proof}
In other words, we have to prove that, given any polynomials \(p_1(x),\dots,p_{\ell}(x)\), 
a polynomial \(p(x)\) vanishes everywhere where all of those polynomials do, at points \(x\) in any finite field extension of \(k\), just when 
\[
p(x)^s=a_1(x)p_1(x)+\dots+a_{\ell}(x)p_{\ell}(x),
\]
for some integer \(s\) and some polynomials \(a_1(x),\dots,a_{\ell}(x)\).

If not, consider 
\[
1-tp(x),p_1(x),\dots,p_{\ell}(x),
\]
as polynomials in one more variable \(t\).
At a common zero \((x,t)\), all \(p_1(x),\dots,p_{\ell}(x)\) vanish, so \(p(x)\) vanishes, so \(1-tp(x)\) doesn't.
Hence there is no common zero.
By the nullstellensatz above, there are polynomials 
\[
g_0(t,x),g_1(t,x),\dots,g_{\ell}(t,x),
\]
so that
\[
1=g_0(t,x)(1-tp(x))
+
g_1(t,x)p_1(x)
+
\dots
+
g_{\ell}(t,x)p_{\ell}(x).
\]
Substitute \(t\defeq 1/p(x)\), so that in the field of fractions, i.e. in the field of rational functions,
\[
1=g_1(t,x)p_1(x)
+
\dots
+
g_{\ell}(t,x)p_{\ell}(x),
\]
where \(t=1/p(x)\).
Expand into a common denominator, say of \(p(x)^s\):
\[
1=\frac{a_1(x)p_1(x)
+
\dots
+
a_{\ell}(x)p_{\ell}(x)}{p(x)^s}.
\]
Clear denominators.
\end{proof}

\section{Closed sets}
A \emph{closed set}\define{closed set} in affine space \(k^n\) (over a field \(k\)) is just another name for an affine variety.
\begin{example}
The entire set \(k^n\) is closed: it is the zero locus of the polynomial function \(p(x)=0\).
\end{example}
\begin{example}
Every finite set in \(k\) is closed: the set \(\Set{c_1,c_2,\dots,c_n}\) is the zero locus of 
\[
p(x)=\pr{x-c_1}\pr{x-c_2}\dots\pr{x-c_n}.
\]
\end{example}
\begin{example}
Any infinite subset of \(k\), except perhaps \(k\) itself, is \emph{not} closed.
We know that every ideal \(I\) in \(k[x]\) is generated by a single polynomial (the greatest common divisor of the polynomials in \(I\)), say \(I=p(x)\), so the associated closed set is the finite set of roots of \(p(x)\).
\end{example}
\begin{example}
A subset of \(k^2\) is closed just when it is either (a) all of \(k^2\) or (b) a finite union of algebraic curves and points.
\end{example}

An \emph{open set}\define{open set} is the complement of a closed set.
An open set containing a point \(c\) of \(k^n\) is a \emph{neighborhood}\define{neighborhood} of \(c\).
The intersection of any collection of closed sets is a closed set: just put together all of the equations of each set into one set of equations.
In particular, the intersection of all closed sets containing some collection \(X\) of points of \(k^n\) is a closed set, the \emph{closure}\define{closure} of \(X\).
A set \(S\) is dense in a closed set \(X\) if \(X\) is the closure of \(S\).

If the set \(X\) in \(k^n\) is the zero locus of some polynomials \(f_i(x)\) and the set \(Y\) in \(k^n\) is the zero locus of some polynomials \(g_j(x)\), then \(X \cup Y\) in \(k^n\) is the zero locus of the polynomials \(f_i(x)g_j(x)\).
Hence the union of finitely many closed sets is closed.

\begin{problem}{nullstellensatz:closed.image}
Give an example of a field \(k\) and a polynomial \(b(x,y)\) over \(k\) so that the points in the image of the polynomial (i.e. the points of the form \(z=b(x,y)\) for any \(x\) and \(y\) in \(k\)) do \emph{not} form an affine subvariety of \(k\).
\end{problem}
\begin{answer}{nullstellensatz:closed.image}
\(p(x,y)=(1-xy)^2+x^2\) over the field \(k=\R{}\).
\end{answer}


\section{Projective varieties}
Just as an algebraic curve in the projective plane is, by definition, the zero locus of nonzero homogeneous polynomial in three variables, so a \emph{projective variety}\define{variety!projective}\define{projective variety} \(X\) in projective space \(\Proj{n}\) is the collection of all lines on which some homogeneous polynomials in \(n+1\) variables vanish.
Note that these zeroes can lie in the projective space of the algebraic closure of the field of definition of \(X\), as was our definition for algebraic curves.
By the Hilbert basis theorem, we can assume that the collection of polynomials is finite.
As before, taking affine coordinates, i.e. setting one of the variables to \(1\), gives an affine chart on projective space, and \(X\) intersects that chart in an affine variety.
\begin{example}
If our set of homogeneous polynomials is empty, we find that projective space is a projective variety inside itself.
\end{example}
\begin{example}
In the projective plane, with homogeneous coordinates \([x,y,z]\), the homogeneous polynomials \(x^2,y^2,z^2\) don't vanish at any point of projective space, as their common zero locus in \(k^3\) is at the origin, so they don't vanish on any line in \(k^3\).
\end{example}
\begin{example}
The points of any projective variety are allowed to arise in some field extension.
So a projective variety has no points just when the homogeneous polynomials cutting it out vanish nowhere or only at the origin.
By the nullstellensatz, they generate an ideal containing a nonzero constant function, or the radical ideal they generate is the ideal generated by all homogeneous linear functions.
\end{example}
\begin{example}
To each symmetric \((n+1)\times(n+1)\) matrix \(A\) over any field, the associated \emph{quadric hypersurface}\define{quadric hypersurface} \(Q=Q_A\) in projective space \(\Proj{n}\) is the set of all null lines of \(A\); a line is \emph{null} for \(A\) if it is the span of a vector \(x\) in the kernel of \(A\).
\end{example}
\begin{example}
Among the \(p \times q\) matrices \(A\), those which have at most a given rank, say rank \(k\), are precisely those all of whose \((k+1)\times(k+1)\) minor determinants vanish.
These determinants are homogeneous in the coordinates of the matrix, so a collection of homogeneous polynomials.
In the projective space \(\Proj{pq-1}\), we find a variety \(X\).
Each point of \(X\) is a nonzero matrix \(A\), defined up to rescaling, and with rank at most \(k\), and every such matrix, up to rescaling, gives a different point of \(X\).
\end{example}
\begin{example}
The intersection of projective varieties is a projective variety, with equations given by those of each of the original varieties put together.
\end{example}
A projective variety is \emph{reducible}\define{reducible} if it is the union of two projective subvarieties, neither contained in the other.
A projective variety \(X\) is \emph{reducible} in an open set \(U\) if \(X\cap U\) is the union of \(X_1\cap U\) and \(X_2\cap U\), for two projective subvarieties \(X_1,X_2\), neither contained in the other.
A projective variety is \emph{irreducible} at a point \(p_0\) if irreducible in every open set containing \(p_0\).
A \emph{projective hypersurface}\define{hypersurface!projective}\define{projective!hypersurface} is a projective variety \(X\) locally cut out by a single equation, i.e. every point of \(X\) lies in an open set \(U\) of projective space in which there is a nonconstant regular function \(f\) for which \(X\cap U\) is the variety given by \(f=0\).
\begin{problem}{projective.varieties:irred.hyp}
Suppose that \(X\) is a projective hypersurface, and \(U\) an open subset of projective space in which \(X\cap U\) is the variety given by an equation \(f=0\).
Suppose that \(f\) is irreducible.
Prove that \(X\) is irreducible.
\end{problem}
\begin{answer}{projective.varieties:irred.hyp}
By the nullstellensatz, \(X\) is irreducible in \(U\), since splitting \(X\) into a union of distinct subvarieties \(X_1,X_2\), each would satisfy an equation not satisfied on the other, say \(f_1=0\) and \(f_2=0\).
But then \(f=0\) lies inside \(f_1f_2=0\), so \(f\) divides \(f_1\) or \(f_2\).
\end{answer}
\begin{problem}{projective.varieties:irred.hyp.2}
Prove that every analytic hypersurface is, near each of its points, the union of finitely many  hypersurfaces irreducible at that point.
\end{problem}
We use the same terminology of \emph{closed sets} for projective varieties, and \emph{open sets} for their complements, as we did for affine varieties.

