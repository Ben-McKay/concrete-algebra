\chapter{Polynomial equations have solutions}
\epigraph[author={Ahmad al-Fārūqī al-Sirhindī},source={Maktubat}]{If the followers of the prophets err in their proofs, have no fear; the foundation of their knowledge is the authority of the prophets. They provide proofs only for added strength. Following the authority of the prophets is sufficient for them, unlike the philosophers. The philosophers stray from the authority of the prophets and instead rely on their proofs to establish an argument. Misguided, they misguide.}\SubIndex{al-Sirhindī, Ahmad al-Fārūqī}
\epigraph[author={Thomas Paine},source={The Crisis V}]{To argue with a man who has renounced the use and authority of reason, and whose philosophy consists in holding humanity in contempt, is like administering medicine to the dead.}\SubIndex{Paine, Thomas}
\section{Affine space}
For any field \(k\), \emph{affine space}\define{affine space} \(k^n\) is the set of all points
\[
c=
\begin{pmatrix}
c_1 \\
c_2 \\
\vdots \\
c_n
\end{pmatrix}
\]
with coordinates \(c_1, c_2, \dots, c_n\) in \(k\).

We want to say: if you have infinitely many polynomial equations, in a finite set of variables, you can throw away all but finitely many without changing the solutions.
Let us be more precise.
Recall that an \emph{ideal} \(I\) in \(k[x]\) is a collection of  polynomials so that we can add and multiply polynomials in \(I\) without leaving \(I\), and \(I\) is \emph{sticky}: if \(f \in I\) and \(g \in k[x]\) then \(fg \in I\).
Any set \(S\) of polynomials generates an ideal \((S)\): the ideal is the set of all expressions
\[
g_1 f_1 + g_2 f_2 + \dots + g_s f_s
\]
where \(f_1, f_2, \dots, f_s\) is any finite collection of polynomials in \(S\) and \(g_1, g_2, \dots, g_s\) are any polynomials at all in \(k[x]\).
Given any set of polynomials, we can replace that set by the ideal of polynomials it generates inside \(k[x]\) without changing the roots of the polynomials.

\begin{theorem}[Hilbert basis theorem]\label{theorem:Hilbert.basis}\define{Hilbert basis theorem}\define{theorem!Hilbert basis}
For any field \(k\) and any variables \(x=\pr{x_1,x_2,\dots,x_n}\), every ideal \(I\) in \(k[x]\) is finitely generated.
In other words, there are finitely many polynomials 
\[
f_1(x), f_2(x), \dots, f_s(x)
\]
in \(I\) so that the polynomials in \(I\) are precisely those of the form
\[
g_1(x) f_1(x) + g_2(x) f_2(x) + \dots + g_s(x) f_s(x),
\]
for any polynomials \(g_1(x), g_2(x), \dots, g_s(x)\) in \(k[x]\).
\end{theorem}
\begin{proof}
The result is obvious if \(n=0\) since the only ideals in \(k\) are \(k\) and \(0\).
By induction, suppose that the result is true for any number of variables less than or equal to \(n\).
Take an ideal \(I \subset k[x,y]\) where \(x=\pr{x_1,x_2,\dots,x_n}\) and \(y\) is a variable.
Write each element \(f \in I\) as
\[
f(x,y)=f_0(x)+f_1(x)y+\dots+f_d(x)y^d,
\]
so that \(f_d(x)\) is the \emph{leading coefficient}.
Let \(J\) be the set of all leading coefficients of elements of \(I\).
Clearly \(J\) is an ideal in \(k[x]\), so finitely generated by induction.
So we can pick polynomials \(f_1(x,y), f_2(x,y), \dots, f_s(x,y)\) in \(I\) whose leading coefficients generate the ideal \(J\) of all leading coefficients.

Let \(N\) be the maximum degree in \(y\) of any of the \(f_j(x,y)\).
Take some polynomial \(p(x,y)\) from \(I\) of degree greater than \(N\) in \(y\), say with leading coefficient \(\check{p}(x)\).
Express that leading coefficient in terms of the leading coefficients \(\check{f}_j(x)\) of the polynomials \(f_j(x,y)\) as
\[
\check{p}(x)=\sum p_j(x) \check{f}_j(x).
\]
If we raise the \(y\) degree of \(f_j(x,y)\) suitably to match the \(y\) degree of \(p(x,y)\), say by multiplying by \(y^{k_j}\), some positive integer \(k_j\), then
\[
\sum p_j(x) y^{k_j} f_j(x,y)
\]
has leading coefficient the same as \(p(x,y)\), so 
\[
p(x,y)=\sum y^{k_j} p_j(x) f_j(x,y)  + \dots
\]
up to lower order in \(y\).
Inductively we can lower the order in \(y\) of \(p\), replacing \(p\) by the \(\dots\), until we get to order in \(y\) less than \(N\).
At each step, we modify \(p\) by something in the ideal generated by the \(f_j\).
So every polynomial in \(I\) of degree \(N\) or less in \(y\) is in the ideal generated by the \(f_j\).
Add to our list of \(f_j(x,y)\) some additional \(f_j(x,y) \in I\) to span all of the polynomials in \(I\) of degree up to \(N\) in \(y\).
By induction we have generated all of the polynomials in \(I\).
\end{proof}



\begin{theorem}[Nullstellensatz]\label{theorem:nullstellensatz}\define{nullstellensatz}\define{theorem!nullstellensatz}
Given a collection of polynomial equations in variables
\[
x=\pr{x_1, x_2, \dots, x_n}.
\]
over a field \(k\), either there is a point \(x=c\) defined in a finite algebraic extension field of \(k\) where all of these polynomials vanish or else the ideal generated by the polynomials is all of \(k[x]\).
\end{theorem}
\begin{proof}
Follows from theorem~\vref{theorem:Hilbert.basis} and corollary~\vref{corollary:Null}.
%Let \(I\) be the ideal generated by the polynomials in the equations, and let \(X\) be the set of simultaneous roots of all of those polynomials.
%If the ideal \(I\) is the zero ideal, then \(X=k^n\) and the result is clear.
%
%In one dimension, i.e. if \(n=1\), take a smallest degree nonzero polynomial in the ideal, say \(f(x)\).
%Any other polynomial \(g(x)\) in the ideal has larger or the same degree; take its quotient and remainder:\(g(x)=q(x)f(x)+r(x)\), where \(r(x)\) has degree less than \(f(x)\), and \(r(x)=g(x)-q(x)f(x)\) lies in our ideal.
%If \(r(x)\) is not zero then our ideal has a smaller degree nonzero polynomial, a contradiction.
%So \(r(x)=0\), i.e. every element of the ideal is divisible by that one nonzero polynomial \(f(x)\).
%Hence \(X\) is the set of zeroes of \(f(x)\), so not empty (after perhaps replacing \(k\) by a finite algebraic extension field) by theorem~\vref{theorem:splitting.fields.exist}.
%
%Suppose that that we have already proven the result for all number of variables up to and including \(n\).
%Write polynomials in \(n+1\) variables as \(f(x,y)\) with \(x=\pr{x_1,x_2,\dots,x_n}\).
%Pick any nonzero element \(g(x,y)\) from \(I\).
%By lemma~\vref{lemma:linear.normalization}, we can make a linear change of variables to arrange that \(g(x,y)=y^m+\dots\) is monic in the \(y\) variable (after perhaps replacing \(k\) by a finite algebraic extension field).
%Let \(I'=I \cap k[x]\) be the ideal in \(k[x]\) consisting of polynomials in \(I\) that don't involve the \(y\) variable.
%Note that \(I=k[x,y]\) just when \(1 \in I\) so just when \(1 \in I'\) so just when \(I'=k[x]\).
%We can suppose that \(I\ne k[x,y]\), so \(I' \ne k[x]\).
%By induction, there is a point \(x=c\) at which all polynomials in \(I'\) vanish (after perhaps replacing \(k\) by a finite algebraic extension field).
%
%Let 
%\[
%J\defeq \Set{f(c,y)|f \in I} \subset k[y].
%\]
%It is clear that \(J\) is an ideal in \(k[y]\).
%We will now prove that \(J \ne k[y]\).
%If \(J=k[y]\), we can write \(1=f(c,y)\) for some \(f \in I\).
%Expand out as a polynomial in \(y\):
%\[
%f(x,y)=f_0(x) + f_1(x)y + \dots + f_{\ell}(x)y^{\ell}.
%\]
%Then at \(x=c\),
%\[
%1=f_0(c) + f_1(c)y + \dots + f_{\ell}(c)y^{\ell}.
%\]
%So \(1=f_0(c)\) and \(0=f_1(c)=f_2(c)=\dots=f_{\ell}(c)\).
%Take the resultant in the \(y\) variable:
%\(r(x)\defeq \resultant{f}{g}\).
%By lemma~\vref{lemma:resultant.over.rings},
%\[
%r(x)=u(x,y)f(x,y)+v(x,y)g(x,y)
%\]
%for polynomials \(u(x,y), v(x,y)\).
%Therefore \(r(x)\) belongs to \(I'\).
%But \(g(c,y)=y^m+\dots\) doesn't vanish.
%Looking at the determinant that gives the resultant \(r(x)\), it is upper triangular with 1 in every diagonal entry, so \(r(x)=1\).
%But then \(1 \in I' \subset I\) and therefore \(I=k[x,y]\).
%This is a contradiction, so our assumption that \(J=k[y]\) has led us astray.
%So \(J \subsetneq k[y]\) is an ideal in one variable polynomials, and therefore \(J\) is generated by a single polynomial \(h(y)\).
%All elements of \(I\) vanish at \((x,y)=(c,b)\) for any of the zeroes \(y=b\) of \(h(y)\).
\end{proof}

\begin{problem}{nullstellensatz:max.ideal}
Given a point \(c \in k^n\), prove that the ideal 
\[
I_c \defeq \Set{f(x)|f(c)=0}
\]
of functions vanishing at \(x=c\) is a maximal ideal.
Prove that if \(x=p\) and \(x=q\) are distinct points of \(k^n\), then \(I_p \ne I_q\).
\end{problem}

\begin{corollary}
Take variables
\[
x=\pr{x_1, x_2, \dots, x_n}.
\]
over an algebraically closed field \(k\).
Every maximal ideal in \(k[x]\) is the set of all polynomial functions vanishing at some point \(c\), for a unique point \(c\).
\end{corollary}
\begin{proof}
Take any ideal \(I \subset k[x]\)
Take a point \(c\) so that all polynomials in \(I\) vanish there, using theorem~\vref{theorem:nullstellensatz}.
Then \(I \subset I_c\).
\end{proof}


\section{Closed sets}

An \emph{affine variety}\define{affine variety}\define{variety!affine} \(X=X_S\) is the collection of zeroes of a set \(S\) of polynomials in variables 
\[
x=\pr{x_1, x_2, \dots, x_n}.
\]
A \emph{closed set}\define{closed set} in affine space \(k^n\) (over a field \(k\)) is just another name for an affine variety.
\begin{example}
The entire set \(k^n\) is closed: it is the zero locus of the polynomial function \(p(x)=0\).
\end{example}
\begin{example}
Every finite set in \(k\) is closed: the set \(\Set{c_1,c_2,\dots,c_n}\) is the zero locus of 
\[
p(x)=\pr{x-c_1}\pr{x-c_2}\dots\pr{x-c_n}.
\]
\end{example}
\begin{example}
Any infinite subset of \(k\), except perhaps \(k\) itself, is \emph{not} closed.
We know that every ideal \(I\) in \(k[x]\) is generated by a single polynomial (the greatest common divisor of the polynomials in \(I\)), say \(I=p(x)\), so the associated closed set is the finite set of roots of \(p(x)\).
\end{example}
\begin{example}
A subset of \(k^2\) is closed just when it is either (a) all of \(k^2\) or (b) a finite union of algebraic curves and points.
\end{example}

An \emph{open set}\define{open set} is the complement of a closed set.
An open set containing a point \(c\) of \(k^n\) is a \emph{neighborhood}\define{neighborhood} of \(c\).
The intersection of any collection of closed sets is a closed set: just put together all of the equations of each set into one set of equations.
In particular, the intersection of all closed sets containing some collection \(X\) of points of \(k^n\) is a closed set, the \emph{closure}\define{closure} of \(X\).
A set \(S\) is dense in a closed set \(X\) if \(X\) is the closure of \(S\).

If the set \(X\) in \(k^n\) is the zero locus of some polynomials \(f_i(x)\) and the set \(Y\) in \(k^n\) is the zero locus of some polynomials \(g_j(x)\), then \(X \cup Y\) in \(k^n\) is the zero locus of the polynomials \(f_i(x)g_j(x)\).
Hence the union of finitely many closed sets is closed.

\begin{problem}{nullstellensatz:closed.image}
Give an example of a field \(k\) and a polynomial \(b(x,y)\) over \(k\) so that the points in the image of the polynomial (i.e. the points of the form \(z=b(x,y)\) for any \(x\) and \(y\) in \(k\)) do \emph{not} form an affine subvariety of \(k\).
\end{problem}
\begin{answer}{nullstellensatz:closed.image}
\(p(x,y)=(1-xy)^2+x^2\) over the field \(k=\R{}\).
\end{answer}


