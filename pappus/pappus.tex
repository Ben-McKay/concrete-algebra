\documentclass{article}
\usepackage{etoolbox}
\usepackage{xstring}
\usepackage{tikz}
\usetikzlibrary{
arrows,
backgrounds,
decorations.pathmorphing,
fit,
intersections,
petri,
positioning,
through}
\begin{document}

% highest x value of any coordinate of any point in the picture:
\newcommand*{\maxX}{5.5}
% lowest x value of any coordinate of any point in the picture:
\newcommand*{\minX}{0}
\newcommand*{\tikzscale}{1}
\newcommand*{\dotr}{2pt}
\newcommand*{\AstepX}{1}
\newcommand*{\AstepY}{.3}
\newcommand*{\Ax}{1}
\newcommand*{\Ay}{0}
\newcommand*{\Ap}[1]{{\Ax+(#1)*\AstepX},{\Ay+(#1)*\AstepY}}
\newcommand*{\BstepX}{1}
\newcommand*{\BstepY}{-.2}
\newcommand*{\Bx}{1}
\newcommand*{\By}{-1}
\newcommand*{\Bp}[1]{{\Bx+(#1)*\BstepX},{\By+(#1)*\BstepY}}
\newcommand*{\lineColour}{gray,opacity=.5}
\newcommand*{\dotColour}{gray,opacity=.5}

\tikzstyle{line}=[gray!75]
\tikzstyle{point}=[gray!75]
\tikzstyle{lineOne}=[brown!30!gray]
\tikzstyle{lineTwo}=[blue!30!gray]
\tikzstyle{lineThree}=[green!30!gray]

\newcommand*{\drawFirstLine}
{
\draw[line] (\Ap{0}) -- (\Ap{1}) -- (\Ap{2}) -- (\Ap{3}) -- (\Ap{4});
\fill[point] (\Ap{1}) circle (\dotr) node (a1) {};
\fill[point] (\Ap{2}) circle (\dotr) node (a2) {};
\fill[point] (\Ap{3}) circle (\dotr) node (a3) {};
}


\newcommand*{\drawSecondLine}
{
\draw[line] (\Bp{0}) -- (\Bp{1}) -- (\Bp{2}) -- (\Bp{3}) -- (\Bp{4});
\fill[point] (\Bp{1}) circle (\dotr) node (b1) {};
\fill[point] (\Bp{2}) circle (\dotr) node (b2) {};
\fill[point] (\Bp{3}) circle (\dotr) node (b3) {};
}

\newcommand*{\drawCorrespondence}
{
\foreach \i in {1,...,3}
  {
    \draw[line] (\Ap{\i}) -- (\Bp{\i});
  }
}


\newcommand*{\drawConnectingLines}
{
  \draw[line] (\Ap{1}) -- (\Bp{2});
  \draw[line] (\Ap{1}) -- (\Bp{3});
  \draw[line] (\Ap{2}) -- (\Bp{1});
  \draw[line] (\Ap{2}) -- (\Bp{3});
  \draw[line] (\Ap{3}) -- (\Bp{1});
  \draw[line] (\Ap{3}) -- (\Bp{2});
}

\newcommand*{\drawConnectingLinesColoured}
{
  \draw[lineOne] (\Ap{1}) -- (\Bp{2});
  \draw[lineTwo] (\Ap{1}) -- (\Bp{3});
  \draw[lineOne] (\Ap{2}) -- (\Bp{1});
  \draw[lineThree] (\Ap{2}) -- (\Bp{3});
  \draw[lineTwo] (\Ap{3}) -- (\Bp{1});
  \draw[lineThree] (\Ap{3}) -- (\Bp{2});
}


% \newcommand*{\drawSecondTriangle}
% {
% \fill[brown!15] (\Bone) -- (\Btwo) -- (\Bthree) -- cycle;
% \fill[brown!50] (\Bone) circle (\dotr) node (b1) {};
% \fill[brown!50] (\Btwo) circle (\dotr) node (b2) {};
% \fill[brown!50] (\Bthree) circle (\dotr) node (b3) {};
% }
% 
% \newcommand*{\connectVantageAndFirstTwoTriangles}
% {
% \draw[gray,opacity=.5] (p) -- (a1) -- (b1);
% \draw[gray,opacity=.5] (p) -- (a2) -- (b2);
% \draw[gray,opacity=.5] (p) -- (a3) -- (b3);
% }
% 
% \newcommand*{\findFirstIntersection}
% {
% \node (c1) at (intersection of a2--a3 and b2--b3) {};
% }
% \newcommand*{\findSecondIntersection}
% {
% \node (c2) at (intersection of a1--a3 and b1--b3) {};
% }
% \newcommand*{\findThirdIntersection}
% {
% \node (c3) at (intersection of a1--a2 and b1--b2) {};
% }
% 
% \newcommand*{\findAllIntersections}
% {
% \findFirstIntersection
% \findSecondIntersection
% \findThirdIntersection
% }
% 
% \newcommand*{\showFirstIntersectionPoint}
% {
% %\draw[white] (c1) circle (1.4pt);
% \fill[gray!30] (c1) circle (\dotr);
% }
% 
% \newcommand*{\showSecondIntersectionPoint}
% {
% %\draw[white] (c2) circle (1.4pt);
% \fill[gray!30] (c2) circle (\dotr);
% }
% 
% \newcommand*{\showThirdIntersectionPoint}
% {
% %\draw[white] (c3) circle (1.4pt);
% \fill[gray!30] (c3) circle (\dotr);
% }
% 
% 
% \newcommand*{\showAllIntersectionPoints}
% {
% \showFirstIntersectionPoint
% \showSecondIntersectionPoint
% \showThirdIntersectionPoint
% }
% 
% \newcommand*{\showHowFirstIntersectionIsConstructed}
% {
% \draw[gray] (a3) -- (c1); % line through c1
% \draw[gray] (b3) -- (c1); % line through c1
% \showFirstIntersectionPoint
% }
% 
% \newcommand*{\showHowSecondIntersectionIsConstructed}
% {
% \draw[gray] (a3) -- (c2); % line through c1
% \draw[gray] (b3) -- (c2); % line through c1
% \showSecondIntersectionPoint
% }
% 
% \newcommand*{\showHowThirdIntersectionIsConstructed}
% {
% \draw[gray] (a2) -- (c3); % line through c1
% \draw[gray] (b2) -- (c3); % line through c1
% \showThirdIntersectionPoint
% }
% 
% \newcommand*{\showHowAllIntersectionsAreConstructed}
% {
% \showHowFirstIntersectionIsConstructed
% \showHowSecondIntersectionIsConstructed
% \showHowThirdIntersectionIsConstructed
% }
% 
% \newcommand*{\showTheLine}
% {
% \draw[green!50!gray] (c1) -- (c2);
% \draw[green!50!gray] (c2) -- (c3);
% \showAllIntersectionPoints
% }
% 
% \newcommand*{\showThirdTriangle}
% {
% \fill[green!50!gray,opacity=.1] (c1) to[bend left=45] (c2) to[bend right=45] (c3) to[bend left=90] (c1);
% %\draw[green!50!gray] (c1) to[bend left=45] (c2);
% %\draw[green!50!gray] (c2) to[bend right=45] (c3);
% %\draw[green!50!gray] (c3) to[bend left=90] (c1);
% \showAllIntersectionPoints
% }
% 
\newcommand*{\firstPicture}
{
\drawFirstLine
\drawSecondLine
}

\newcommand*{\secondPicture}
{
\drawCorrespondence
\firstPicture
}

\newcommand*{\thirdPicture}
{
\drawConnectingLines
\firstPicture
}

\newcommand*{\fourthPicture}
{
\drawConnectingLinesColoured
\firstPicture
}

%\newcommand*{\fifthPicture}
% {
% \drawOrigin
% \drawFirstTriangle
% \drawSecondTriangle
% \findThirdIntersection
% \showHowThirdIntersectionIsConstructed
% }
% 
% \newcommand*{\sixthPicture}
% {
% \drawOrigin
% \drawFirstTriangle
% \drawSecondTriangle
% \findAllIntersections
% \showTheLine
% }
% 
% \newcommand*{\seventhPicture}
% {
% \drawOrigin
% \drawFirstTriangle
% \drawSecondTriangle
% \connectVantageAndFirstTwoTriangles
% \findAllIntersections
% \showHowAllIntersectionsAreConstructed
% \showTheLine
% }
% 
% \newcommand*{\eighthPicture}
% {
% \drawOrigin
% \drawFirstTriangle
% \drawSecondTriangle
% \connectVantageAndFirstTwoTriangles
% \findAllIntersections
% \showHowAllIntersectionsAreConstructed
% \showThirdTriangle
% }
% 
\newcommand{\Pappus}[1]{%
\begin{center}
\begin{tikzpicture}[scale=\tikzscale]
% Draw a strut to make all of the diagrams have the same width, so the same x coordinate in each diagram gives a point the same distant across the page.
\draw[white] (\minX,0) -- (\maxX,0); 
\IfStrEqCase{#1}{%
{1}%
{%%
\firstPicture
}%%
{2}%
{%
\secondPicture
}%
{3}%
{%%
\thirdPicture
}%%
{4}%
{%%
\fourthPicture
}%%
{5}%
{%%
\fifthPicture
}%%
{6}%
{%%
\sixthPicture
}%%
{7}%
{%%
\seventhPicture
}%%
{8}%
{%%
\eighthPicture
}%%
}%% end IfStrEqCase
\end{tikzpicture}
\end{center}
}
Take any two lines and pick three points on each:
\Pappus{1}
and arbitrarily choose a correspondence between each of the points on the first line with some point on the second, bijectively:
\Pappus{2}
Erase those lines and instead draw a line connecting each of the three points on the first line to the \emph{noncorresponding} points on the second line.
\Pappus{3}
These 6 connecting lines have 15 intersection points, but only 13 are drawn.
The correspondence of the points gives a correspondence of the lines, 
\Pappus{4}
\end{document}
