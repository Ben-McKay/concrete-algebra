\chapter{Quotient rings}\label{chapter:quotient.rings}
\epigraph[author={E.~\OE.~Somerville \& Martin Ross}, source={The Irish R.M.}]{\ldots, whereas in Ireland, two and two are just as likely to make five, or three, or are still more likely to make nothing at all.}\SubIndex{Somerville, E.~\OE.}\SubIndex{Ross, Martin}
Recall once again that all of our rings are assumed to be commutative rings with identity.
Suppose that \(R\) is a ring and that \(I\) is an ideal.
For any element \(r \in R\), the \emph{translate} of \(I\) by \(r\), denoted \(r+I\), is the set of all elements \(r+i\) for any \(i \in I\).
\begin{example}
If \(R=\Z{}\) and \(I=12\Z{}\) is the multiples of \(12\), then \(7+I\) is the set of integers which are \(7\) larger than a multiple of \(12\), i.e. the set of all numbers 
\[
\dots,7-12,7,7+12,7+2 \cdot 12, 7+3 \cdot 12, \dots
\]
which simplifies to
\[
\dots,-5,7,19,31,53, \dots
\]
But this is the same translate as \(-5+I\), since \(-15+12=7\) so \(-5+I\) is
the set of all numbers
\[
\dots,-5-12,-5,-5+12,-5+2 \cdot 12, -5+3 \cdot 12, \dots
\]
which is just 
\[
\dots,-15,-5,7,19,31,53, \dots
\]
the same numbers.
\end{example}
\begin{example}
Working again inside \(\Z{}\), \(2+2\Z{}=2\Z{}\) is the set of even integers.
\end{example}
\begin{example}
If \(R=\Q{}[x]\) and \(I=(x)\), then \(\frac{1}{2}+I\) is the set of all polynomials of the form
\[
\frac{1}{2}+xp(x)
\]
for any polynomial \(p(x)\), in other words the set of polynomials whose constant term is \(\frac{1}{2}\).
\end{example}
\begin{example}
If \(i \) is in \(I\) then \(i+I=I\).
\end{example}
We add two translates by
\[
\pr{r_1+I}+\pr{r_2+I}\defeq \pr{r_1+r_2}+I,
\]
and multiply by
\[
\pr{r1+I}\pr{r_2+I}=\pr{r_1r_2}+I.
\]
\begin{lemma}\label{lemma:translates}
For any ideal \(I\) in any ring \(R\), two translates \(a+I\) and \(b+I\) are equal just when \(a\) and \(b\) differ by an element of \(I\).
\end{lemma}
\begin{proof}
We suppose that \(a+I=b+I\).
Clearly \(a\) belongs to \(a+I\), because \(0\) belongs to \(I\).
Therefore \(a\) belongs to \(a+I=b+I\), i.e. \(a=b+i\) for some \(i\) from \(I\).
\end{proof}

\begin{lemma}
The addition and multiplication operations on translates are well defined, i.e. if we find two different ways to write a translate as \(r+I\), the results of adding and multiplying translates don't depend on which choice we make of how to write them.
\end{lemma}
\begin{proof}
You write your translates as \(a_1+I\) and \(a_2+I\), and I write mine as \(b_1+I\) and \(sb_2+I\), but we suppose that they are the same subsets of the same ring: \(a_1+I=b_1+I\) and \(a_2+I=b_2+I\).
By lemma~\vref{lemma:translates}, \(a_1=b_1+i_i\) and \(a_2=b_2+i_2\) for some elements \(i_1, i_2\) of \(I\). 
So then
\begin{align*}
a_1+a_2+I
&=
b_1+i_1+b_2+i_2+I,
\\
&=
b_1+b_2+\pr{i_1+i_2}+I,
\\
&=
b_1+b_2+I.
\end{align*}
The same for multiplication.
\end{proof}
If \(I\) is an ideal in a commutative ring \(R\), the \emph{quotient ring}\define{quotient ring} \(R/I\)\Notation{R/I}{R/I}{quotient of a ring by an ideal} is the set of all translates of \(I\), with addition and multiplication of translates as above.
The reader can easily prove:
\begin{lemma}
For any ideal \(I\) in any commutative ring \(R\), \(R/I\) is a commutative ring.
If \(R\) has an identity element then \(R/I\) has an identity element.
\end{lemma}
\begin{example}
If \(R\defeq\Z{}\) and \(I\defeq m\Z{}\) for some integer \(m\) then \(R/I=\Zmod{m}\) is the usual ring of remainders modulo \(m\).
\end{example}
\begin{example}
If \(p(x,y)\) is an irreducible nonconstant polynomial over a field \(k\) and \(I=(p(x,y))=p(x,y)k[x,y]\) is the associated ideal, then \(k[x,y]/I=k[C]\) is the  ring of regular functions on the plane algebraic curve \(C=(p(x,y)=0)\).
\end{example}
\begin{lemma}
An ideal \(I\) in a ring \(R\) with identity is all of \(R\) just when \(1\) is in \(I\).
\end{lemma}
\begin{proof}
If \(1\) lies in \(I\), then any \(r\) in \(R\) is \(r=r1\) so lies in \(I\), so \(I=R\).
\end{proof}
An ideal \(I\) in a ring \(R\) is \emph{prime}\define{prime ideal}\define{ideal!prime} if \(I\) is not all of \(R\) and, for any two elements \(a, b\) of \(R\), if \(ab\) is in \(I\) then either \(a\) or \(b\) lies in \(I\).
\begin{example}
The ideal \(p\Z{}\) in \(\Z{}\) generated by any prime number is prime.
\end{example}
\begin{example}
The ring \(R\) is \emph{not} prime in \(R\), because the definition explicitly excludes it.
\end{example}
\begin{example}
Inside \(R\defeq \Z{}[x]\), the ideal \((x)\) is prime, because in order to have a factor of \(x\) sitting in a product, it must lie in one of the factors.
\end{example}
\begin{example}
Inside \(R\defeq \Z{}[x]\), the ideal \((2x)\) is \emph{not} prime, because \(a=2\) and \(b=x\) multiply to a multiple of \(2x\), but neither factor is a multiple of \(2x\).
\end{example}
\begin{example}
A polynomial \(p(x)\) in several variables, over any field, is irreducible just when \((p(x))\) is a prime ideal.
\end{example}
\begin{example}
The regular functions on an algebraic curve \(p(x,y)=0\) over a field \(k\) constitute the ring \(R/I\) where \(R=k[x,y]\) and \(I=(p(x,y))\).
The curve is irreducible just when the ideal is prime.
\end{example}
\begin{example}
More generally, intuitively, if we think of any ring \(R\) as being something like the polynomial functions in some variables, then an ideal is like the equations of some geometric object (so, roughly, ideals are like geometric objects), prime ideals are like irreducible geometric objects, and quotient rings are like the polynomial functions on those geometric objects.
\end{example}
\begin{example}
Take a field \(k\).
In the quotient ring \(R=k[x,y,z]/(z^2-xy)\), the ideal \((z)\) contains \(z^2=xy\), but does not contain \(x\) or \(y\), so is not prime.
\end{example}

\begin{lemma}
An ideal \(I\) in a commutative ring with identity \(R\) is prime just when \(R/I\) has no zero divisors, i.e. no nonzero elements \(\alpha, \beta\) can have \(\alpha\beta=0\).
\end{lemma}
\begin{proof}
Write \(\alpha\) and \(\beta\) as translates \(\alpha=a+I\) and \(\beta=b+I\).
Then \(\alpha\beta=0\) just when \(ab+I=I\), i.e. just when \(ab\) lies in \(I\).
On the other hand, \(\alpha=0\) just when \(a+I=I\), just when \(a\) is in \(I\).
\end{proof}

A ideal \(I\) in a ring \(R\) is \emph{maximal}\define{ideal!maximal}\define{maximal ideal} if \(I\) is not all of \(R\) but any ideal \(J\) which contains \(I\) is either equal to \(I\) or equal to \(R\) (nothing fits in between).
\begin{example}
The ideal of regular functions vanishing at a point of an algebraic curve is maximal.
Intuitively, we picture any ring as the ring of ``regular functions'' on some ``geometric object''.
The maximal ideals we picture as the ``points'' of that object, while the prime ideals are the ``geometric subobjects'', like curves in a surface, and so on.
\end{example}
\begin{example}
The maximal ideals in the integers are precisely the ideal generated by prime numbers, so the same as the prime ideals.
In this sense, the integers provide a poor example, as we don't see the difference between prime and maximal.
\end{example}
\begin{lemma}
If \(I\) is an ideal in a commutative ring \(R\) with identity, then \(I\) is a maximal ideal just when \(R/I\) is a field.
\end{lemma}
\begin{proof}
Suppose that \(I\) is a maximal ideal.
Take any element \(\alpha \ne 0\) of \(R/I\) and write it as a translate \(\alpha=a+I\).
Since \(\alpha \ne 0\), \(a\) is not in \(I\).
We know that \((a)+I=R\), since \(I\) is maximal.
But then \(1\) lies in \((a)+I\), so \(1=ab+i\) for some \(b\) in \(R\) and \(i\) in \(I\).
Expand out to see that 
\[
\frac{1}{\alpha}=b+I.
\]
So nonzero elements of \(R/I\) have reciprocals.

Suppose instead that \(R/I\) is a field.
Take an element \(a\) in \(R\) not in \(I\).
Then \(a+I\) has a reciprocal, say \(b+I\).
But then \(ab+I=1+I\), so that \(ab=1+i\) for some \(i\) in \(I\).
So \((a)+I\) contains \((ab)+I=(1)=R\). 
\end{proof}

\begin{example}
If \(p\) is a prime number, then \(I \defeq p\Z{}\) is maximal in \(R\defeq \Z{}\), because any other ideal \(J\) containing \(I\) and containing some other  integer, say \(n\), not a multiple of \(p\), must contain the greatest common divisor of \(n\) and \(p\), which is \(1\) since \(p\) is prime.
We recover the fact that \(\Zmod{p}\) is a field.
\end{example}
\begin{example}
The ideals in \(R\defeq \Zmod{4}\) are \((0),(1)=\Zmod{4}\) and \((2)\cong \Zmod{2}\), and the last of these is maximal.
\end{example}
\begin{example}
Take a field \(k\) and let \(R\defeq k[x]\).
Every ideal \(I\) in \(R\) has the form \(I=(p(x))\), because each ideal is generated by the greatest common divisor of its elements.
We have seen that \(R/I\) is prime just when \(p(x)\) is irreducible.
To further have \(R/I\) a field, we need \(I=(p(x))\) to be maximal.
Take some polynomial \(q(x)\) not belonging to \(I\), so not divisible by \(p(x)\).
Since \(p(x)\) is irreducible, \(p(x)\) has no factors, so \(\gcd{p(x),q(x)}=1\) in \(k[x]\), so that the ideal generated by \(p(x),q(x)\) is all of \(R\).
Therefore \(I=(p(x))\) is maximal, and the quotient \(R/I=k[x]/(p(x))\) is a field.
\end{example}
\begin{example}
If \(k\) is an algebraically closed field and \(R\defeq k[x]\) then every irreducible polynomial is linear, so every maximal ideal in \(R\) is \(I=(x-c)\) for some constant \(c\).
\end{example}
\begin{problem}{quotient.rings:ideals.in.Zm}
For any positive integer \(m \ge 2\), what are the prime ideals in \(\Zmod{m}\), and what are the maximal ideals?
\end{problem}
\begin{problem}{quotient.rings:rad.properties}
The \emph{radical} \(\sqrt{I}\) of an ideal \(I\) in a commutative ring \(R\) is the ideal generated by all elements \(r\) of \(R\) so that \(r^n\) is in \(I\) for some integer \(n\ge 1\).
Prove that \(\sqrt{I}\) is an ideal of \(R\).
A \emph{radical ideal} is an ideal \(I\) for which \(\sqrt{I}=I\).
Prove that every prime ideal is radical.
\end{problem}

\section{Sage}
We take the ring \(R=\Q{}[x,y]\) and then quotient out by the ideal \(I=(x^2y,xy^3)\) to produce the ring \(S=R/I\).
The generators \(x,y\) in \(R\) give rise to elements \(x+I,y+I\) in \(S\), which are renamed to \(a,b\) for notational convenience.
\begin{sageblock}
R.<x,y> = PolynomialRing(QQ)
I = R.ideal([x^2*y,x*y^3])
S.<a,b> = R.quotient_ring(I)
(a+b)^4
\end{sageblock}
which yields \(\sage{(a+b)^4}\).

We can define \(R=\Q{}[x,y]\), \(I=(y^2)\), \(S=R/I\) where we label \(x+I, y+I\) as \(a,b\), \(T=\Q{}[z]\) and define a morphism \(\phi \colon S \to T\) by  \(\phi(a)=z^3, \phi(b)=0\):
\begin{sageblock}
R.<x,y> = PolynomialRing(QQ)
I = R.ideal([y^2])
S.<a,b> = R.quotient_ring(I)
T.<z> = PolynomialRing(QQ)
phi = S.hom([z^3, 0],T)
phi(a^2+b^2)
\end{sageblock}
yielding \(\sage{phi(a^2+b^2)}\).


\section{Automorphisms of splitting fields}
\begin{theorem}\label{theorem:splitting.fields.exist}
Suppose that \(k\) is a field and \(p(x)\) is a nonconstant polynomial over \(k\), say of degree \(n\).
Any two splitting fields for \(p(x)\) over \(k\) are isomorphic by an isomorphism which is the identity on \(k\).
\end{theorem}
\begin{proof}
We already know that splitting fields exist, by adding roots to irreducible factors of \(p(x)\).
Each time we add a root, the dimension of the extension field over \(k\) is the degree of the irreducible factor.
We then split off at least one linear factor in the extension field, and repeat.
So at most \(n!\) degree in total.

Given any splitting field \(K\) for an irreducible polynomial \(p(x)\), pick any root \(\alpha \in K\) of \(p(x)\) and map
\[
f(x) \in k[x] \mapsto f(\alpha) \in K,
\]
to see that \(K \cong k[x]/I\).
Hence \(K\) is uniquely determined up to isomorphism.

Suppose instead that \(p(x)\) is reducible, and that \(K,L\) are splitting fields of \(p(x)\).
Pick an irreducible factor \(q(x)\) of \(p(x)\).
Then the subfields of \(K,L\) generated by adding a root of \(q(x)\) to \(k\) are isomorphic, since \(q(x)\) is irreducible, so we can assume that these subfields are the same field.
We need only split \(p(x)\) over that field, or quotient out all copies of \(q(x)\) from \(p(x)\) and split the result.
Apply induction on the degree of \(p(x)\).
\end{proof}
\begin{example}
Adding a root to \(x^2+1\) over \(\R{}\) yields the splitting field \(\C{}\).
\end{example}
\begin{theorem}\label{theorem:irreducible.orbit}
Suppose that \(K\) is a splitting field of an irreducible polynomial \(p(x)\) over a field \(k\).
If \(\alpha,\beta \in K\) are roots of \(p(x)\) then the Galois group of \(K\) over \(k\) has an element which takes \(\alpha\) to \(\beta\).
In particular, \(K\) is a Galois extension of \(k\).
\end{theorem}
\begin{proof}
As in the proof of theorem~\vref{theorem:splitting.fields.exist}, there is a field isomorphism \(k[x]/I \to K\) taking \(x \mapsto \alpha\), and another field isomorphism \(k[x]/I \to K\) taking \(x \mapsto \beta\).
\end{proof}
