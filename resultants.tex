\chapter{Resultants and discriminants}\label{chapter:resultants}

\section{The resultant}

For any polynomial 
\[
b(x) = b_0 + b_1 x + \dots + b_m x^m,
\]
denote its vector of coefficients as
\[
\vec{b}=
\begin{pmatrix}
b_0 \\
b_1 \\
\vdots \\
b_m
\end{pmatrix}.
\]
Clearly the vector of coefficients of \(xb(x)\) is the same, with an extra row added to the top, with a zero in that row.
When we multiply two polynomials \(b(x)\) and
\[
c(x) = c_0 + c_1 x + \dots + c_n x^n
\] 
the vector of coefficients of \(b(x)c(x)\) is
\[
\begin{pmatrix}
b_0 \\
b_1 & b_0 \\
\vdots & \ddots & \ddots \\
b_m & \ddots & \ddots & b_0 \\
    & b_m & \ddots & b_1 \\
    &     & \ddots & \vdots \\
    &    &        & b_m
\end{pmatrix}
\begin{pmatrix}
c_0 \\
c_1 \\
\vdots \\
c_n
\end{pmatrix},
\]
(with zeroes represented as blank spaces) and this \((m+n+1) \times (n+1)\) matrix we denote by
\[
[b]=
\begin{pmatrix}
b_0 \\
b_1 & b_0 \\
\vdots & \ddots & \ddots \\
b_m & \ddots & \ddots & b_0 \\
    & b_m & \ddots & b_1 \\
    &     & \ddots & \vdots \\
    &     &        & b_m
\end{pmatrix}.
\]
So the vectors of coefficient are related by \(\overrightarrow{bc}=[b]\vec{c}\).

The \emph{resultant}\define{resultant} of two polynomials
\begin{align*}
b(x) &= b_0 + b_1 x + \dots + b_m x^m \text{ and } \\
c(x) &= c_0 + c_1 x + \dots + c_n x^n,
\end{align*}
denoted \(\resultant{b}{c}\)\Notation{res(b,c)}{\resultant{b}{c}}{resultant of polynomials \(b(x), c(x)\)}, is the determinant of the matrix \(\begin{pmatrix}\relax [b] & [c] \end{pmatrix}\), given by stacking the matrices \([b]\) and \([c]\) of suitable sizes beside one another, to get a square matrix of size \((m+n) \times (m+n)\).
To remember the sizes: the number of columns in \([b]\) is the degree of \(c\), and vice versa.

\begin{example}
If
\begin{align*}
b(x) &= 4 +3x + 7x^2 + x^3, \\
c(x) &= 5+2x,
\end{align*}
then \(\resultant{b}{c}\) is the determinant of the \(4 \times 4\) matrix
\[
\begin{pmatrix}
  4 & 5 & 0 & 0 \\
  3 & 2 & 5 & 0 \\
  7 & 0 & 2 & 5 \\
  1 & 0 & 0 & 2
\end{pmatrix},
\]
which is \(\resultant{b}{c}=197\).
\end{example}
\begin{example}
If we swap \(b\) and \(c\), we swap columns, so clearly \(\resultant{c}{b}=(-1)^{mn}\resultant{b}{c}\).
\end{example}
\begin{example}
Take \(b(x)=x^2+4x+7\) and \(c(x)=x^3\) and compute
\[
\resultant{b}{c} = 
\det
\begin{pmatrix}
7 & 0 & 0 & 0 & 0 \\
4 & 7 & 0 & 0 & 0 \\
1 & 4 & 7 & 0 & 0 \\
0 & 1 & 4 & 1 & 0 \\
0 & 0 & 1 & 0 & 1
\end{pmatrix}=7^3.
\]
\end{example}
\begin{example}
The easiest examples of arbitrary degrees arise when we take any \(b(x)\) but try \(c(x)=x^n\):
\[
\resultant{b}{c} = 
\det
\begin{pmatrix}
b_0 \\
\vdots  & b_0 \\
\vdots  & \ddots  & \ddots & \\
      b_m & \ddots  & \ddots &  b_0 &  \\
        & b_m      &  \ddots  &  \vdots  &   1 \\
        &         & \ddots &  \vdots  &     & \ddots \\ 
        &         &        &  b_m       &     &         & 1  
\end{pmatrix}=b_0^n.
\]
\end{example}


\section{Sage}

Computing resultants in sage requires us to tell sage what sort of numbers arise as the coefficients of our polynomials.
\begin{sageblock}
P.<x> = PolynomialRing(QQ)
a=x^3+x+7
b=x^2+2
a.resultant(b) 
\end{sageblock}
yields \(\sage{a.resultant(b)}\).
The expression \verb!P.<x> = PolynomialRing(QQ)! tells sage that we are working with polynomials with rational coefficients (\verb!QQ! means rational) in a variable \verb!x!.

We could write our own resultant function, just to see how it might work:
\begin{sageblock}
def resultant(b,c):
    B=b.list()
    C=c.list()
    m=len(B)-1
    n=len(C)-1
    A=matrix(m+n,m+n)
    for j in range(0,n):    
        for i in range(0,m+n):
            if (0<=i-j) and (i-j<=m):
                A[i,j]=B[i-j]
            else:
                A[i,j]=0
    for j in range(n,m+n):    
        for i in range(0,m+n):
            if (0<=i-j+n) and (i-j+n<=n):
                A[i,j]=C[i-j+n]
            else:
                A[i,j]=0
    return det(A)
\end{sageblock}
Try it out:
\begin{sageblock}
t=var('t')
p = t^2+2*t+1
q = t+1
resultant(p,q)
\end{sageblock}
yields \(\sage{resultant(p,q)}\).


\section{Common factors and resultants}

\begin{lemma}\label{lemma:common.factor.polys}
Two polynomials \(b(x)\) and \(c(x)\) of degrees \(m\) and \(n\), over any field, have a nonconstant common factor just when
\[
0 = u(x)b(x)+v(x)c(x)
\]
for two polynomials \(u(x)\) and \(v(x)\) of degrees \(n-1\) and \(m-1\) at most, not both zero.
\end{lemma}
\begin{proof}
Two polynomials \(b(x)\) and \(c(x)\) of degrees \(m\) and \(n\) have a nonconstant common factor just when we can write the polynomials factored, say as \(b(x)=v(x)d(x)\) and \(c(x)=-u(x)d(x)\).
But then
\[
0 = u(x)b(x)+v(x)c(x).
\]
On the other hand, suppose that
\[
0 = u(x)b(x)+v(x)c(x)
\]
for two polynomials \(u(x)\) and \(v(x)\) of degrees \(n-1\) and \(m-1\) at most.
Suppose that \(b(x)\) and \(c(x)\) have no common factor, so their greatest common divisor is \(1\).
Write out B\'ezout coefficients
\[
1=s(x)b(x)+t(x)c(x)
\]
and compute
\begin{align*}
v
&=
vsb+vtc,
\\
&=
vsb+tvc,
\\
&=
vsb-tub,
\\
&=
b\pr{vs-tu}.
\end{align*}
But the degree of \(v\) is smaller than that of \(b\), so \(v=0\).
Swapping roles of \(b,c\) and of \(u,v\) and of \(s,t\), we get \(u=0\).
\end{proof}

\begin{proposition}\label{proposition:resultant.zero}
Two polynomials, over any field, have a nonconstant common factor just when their resultant vanishes.
\end{proposition}
\begin{proof}
Take two polynomials
\begin{align*}
u(x)&=u_0 + u_1 x + \dots + u_{n-1} x^{n-1}, \\
v(x)&=v_0 + v_1 x + \dots + v_{m-1} x^{m-1}.
\end{align*}
The vector of coefficients of \(u(x)b(x)+v(x)c(x)\) is 
\[
\overrightarrow{bu+cv}=
\begin{pmatrix}
\relax [b] & [c]
\end{pmatrix}
\begin{pmatrix}
\vec{u} \\
\vec{v}
\end{pmatrix}
\]
By linear algebra, the determinant vanishes just when the matrix has a nonzero null vector, i.e. a nonzero vector in its kernel.
So the resultant vanishes just when there is a choice of coefficients
\[
u_0, u_1, \dots, v_0, v_1, \dots,
\]
not all zero,  so that \(u(x)b(x)+v(x)c(x)=0\).
In other words, there are polynomials \(u(x)\) and \(v(x)\) of required degrees so that \(u(x)b(x)+v(x)c(x)=0\).
The theorem now follows from lemma~\vref{lemma:common.factor.polys}.
\end{proof}
\begin{problem}{polynomials:resultant}
Find the resultants, and use them to decide if there are nonconstant common factors.
\begin{enumerate}
\item \(x^2-5x+6, x^3-3x^2+x-3\)
\item \(x^2+1,x-1\)
\end{enumerate} 
\end{problem}
The resultant is brutal to compute, even for polynomials of fairly small degree.
Its importance, like that of the determinant of a matrix, arises from its theoretical power.
At this point, the reader should be annoyed: we already know how to find common divisors, even the greatest common divisor, by a fast calculation, so we surely don't need a slow resultant calculation to see if there is a common divisor of positive degree, since we can already actually find that divisor, and more quickly.
The reader is right, but surprisingly we will find many uses for the resultant.
\begin{problem}{resultants:mod.2}
Over the field of remainders modulo \(2\), calculate the resultant of \(b(x)=x^3+x^2+x+1, c(x)=x^3+x\).
\emph{Warning}: the \(6 \times 6\) determinant you get is actually easier than it looks at first, so look for tricks to find it without any hard calculation.
\end{problem}
\begin{answer}{resultants:mod.2}
Either note the common factor of \(x+1\): \(b(x)=(x+1)(x^2+x+1)\), \(c(x)=(x+1)^2x\), or compute the determinant of 
\[
\begin{pmatrix}
1&0&0&0&0&0\\
1&1&0&1&0&0\\
1&1&1&0&1&0\\
1&1&1&1&0&1\\
0&1&1&0&1&0\\
0&0&1&0&0&1
\end{pmatrix}
\]
by expanding across the first row:
\[
\begin{pmatrix}
1&0&1&0&0\\
1&1&0&1&0\\
1&1&1&0&1\\
1&1&0&1&0\\
0&1&0&0&1
\end{pmatrix}
\]
and note that two rows are the same so the determinant (and the resultant) is zero.
\end{answer}

\begin{problem}{resultants:modp.example}
Calculate a resultant to prove that, for any prime integer \(p\ge 2\), when we work over the field of remainders modulo \(p\), the polynomials \(x^2+1\) and \(x^2+3x\) have a common factor just exactly when \(p=2\) or \(p=5\).
\end{problem}
\begin{answer}{resultants:modp.example}
The resultant is
\[
\resultant{x^2+1}{x^2+3x}
=
\det
\begin{pmatrix}
1 & 0 & 0 & 0 \\
0 & 1 & 3 & 0 \\
1 & 0 & 1 & 3 \\
0 & 1 & 0 & 1
\end{pmatrix}
=10.
\]
But \(10\) only has \(2\) and \(5\) as prime factors, so this resultant vanishes just exactly when \(p=2\) or \(p=5\).
If \(p=2\) the factors are \(x^2+1=(x+1)^2\) and \(x^2+3x=x^2+x=x(x+1)\), so a common factor of \(x+1\).
If \(p=5\) the factors are \(x^2+1=(x-2)(x-3)\) and \(x^2+3x=x(x-2)\), so a common factor of \(x-2\).
\end{answer}
\begin{problem}{resultants:even.terms}
Suppose that \(b(x)\) and \(c(x)\) are polynomials in one variable \(x\), with coefficients in the field of integer remainders modulo \(2\). 
Suppose that \(b(x)\) is a sum of \(1126\) nonzero terms, while \(c(x)\) is a sum of \(8080\) nonzero terms. 
Note: we don't say anything about the degree of either one. 
Prove that they have a common factor of positive degree.
\end{problem}
\begin{answer}{resultants:even.terms}
The resultant is the determinant of a matrix whose every column has an even number of nonzero entries, so sum equal to zero. So the sum of the rows is zero, a linear relation among rows, hence zero determinant. 
\end{answer}
\begin{lemma}\label{lemma:resultant.over.integers}
Take two polynomials \(b(x), c(x)\) in a variable \(x\) of degrees \(m, n\).
The resultant \(r=\resultant{b}{c}\) is expressible as \(r=u(x)b(x)+v(x)c(x)\) where \(u(x)\) and \(v(x)\) are polynomials of degrees \(n-1, m-1\).
One can make a particular choice of \(u(x), v(x)\) so that the coefficients of \(u(x)\) and of \(v(x)\) are expressible as polynomial expressions in the coefficients of \(b(x)\) and \(c(x)\), and those polynomial expressions have only \(1\) or \(-1\) as coefficient in each term.
\end{lemma}
\begin{proof}
Look at our matrix:
\[
r
=
\det
\begin{pmatrix}
b_0    &        &        &        & c_0 \\
b_1    & b_0    &        &        & c_1    & c_0 \\
\vdots & b_1    & \ddots &        & \vdots & c_1 & \ddots & \\
\vdots & \ddots & \ddots  & b_0   & \vdots & \ddots & \ddots & c_0 \\
b_m    & \ddots & \ddots  & b_1   & c_n    & \ddots & \ddots & c_1 \\
       & b_m    & \ddots & \vdots &        & c_n    & \ddots & \vdots \\
       &        & \ddots & \vdots    &        &        & \ddots & \vdots \\
       &        &        & b_m    &        &        &        & c_n \\
\end{pmatrix}
\]
Add to the first row the second multiplied by \(x\) and then the third multiplied by \(x^2\) and so on, which doesn't change the determinant:
\[
r
=
\det
\begin{pmatrix}
b(x)   & xb(x)  & \dots  & x^{n-1}b(x) & c(x) & xc(x) & \dots & x^{m-1}c(x) \\
b_1    & b_0    &        &        & c_1    & c_0 \\
\vdots & b_1    & \ddots &        & \vdots & c_1 & \ddots & \\
\vdots & \ddots & \ddots & b_0   & \vdots & \ddots & \ddots & c_0 \\
b_m    & \ddots & \ddots & b_1   & c_n    & \ddots & \ddots & c_1 \\
       & b_m    & \ddots & \vdots &        & c_n    & \ddots & \vdots \\
       &        & \ddots & \vdots    &        &        & \ddots & \vdots \\
       &        &        & b_m    &        &        &        & c_n \\
\end{pmatrix}.
\]
Expand the determinant across the top row to see that
\[
r=u(x)b(x)+v(x)c(x)
\]
where
\[
u(x)
=
\det
\begin{pmatrix}
1      & x      & \dots  & x^{n-1} \\
b_1    & b_0    &        &         & c_1    & c_0 \\
\vdots & b_1    & \ddots &        & \vdots & c_1 & \ddots & \\
\vdots & \ddots & \ddots & b_0   & \vdots & \ddots & \ddots & c_0 \\
b_m    & \ddots & \ddots & b_1   & c_n    & \ddots & \ddots & c_1 \\
       & b_m    & \ddots & \vdots &        & c_n    & \ddots & \vdots \\
       &        & \ddots & \vdots    &        &        & \ddots & \vdots \\
       &        &        & b_m    &        &        &        & c_n \\
\end{pmatrix}
\]
and
\[
v(x)
=
\det
\begin{pmatrix}
       &        &        &        & 1      & x      & \dots & x^{m-1} \\
b_1    & b_0    &        &        & c_1    & c_0 \\
\vdots & b_1    & \ddots &        & \vdots & c_1 & \ddots & \\
\vdots & \ddots & \ddots & b_0    & \vdots & \ddots & \ddots & c_0 \\
b_m    & \ddots & \ddots & b_1    & c_n    & \ddots & \ddots & c_1 \\
       & b_m    & \ddots & \vdots &        & c_n    & \ddots & \vdots \\
       &        & \ddots & \vdots    &        &        & \ddots & \vdots \\
       &        &        & b_m    &        &        &        & c_n \\
\end{pmatrix}.
\]
Note that we can expand out any determinant using only multiplication and addition of entries and some \(\pm\) signs.
\end{proof}



If we scale up \(b(x)\) by a factor \(\lambda\), the matrix gets a \(\lambda\) in each column in the left \(n\) columns, where \(n\) is the degree of \(c(x)\).
The resultant of \(b(x),c(x)\) scales by \(\lambda^n\).
So practically we only need to compute resultants of monic polynomials.

\begin{proposition}\label{proposition:factored.resultants}
Given two monic polynomials over any field, each factored into linear factors,
\begin{align*}
b(x) &= \pr{x-\beta_1} \pr{x-\beta_2} \dots \pr{x-\beta_m}, \\
c(x) &= \pr{x-\gamma_1} \pr{x-\gamma_2} \dots \pr{x-\gamma_n}, \\
\end{align*}
then the resultant is
\begin{align*}
\resultant{b}{c} 
&=
\pr{\gamma_1-\beta_1}\pr{\gamma_1-\beta_2}\dots\pr{\gamma_n-\beta_m},
\\
&=
b\of{\gamma_1}b\of{\gamma_2}\dots b\of{\gamma_n}.
\\
&=
(-1)^{mn}c\of{\beta_1}c\of{\beta_2}\dots c\of{\beta_m},
\end{align*}
In particular, the resultant vanishes just when \(b(x)\) and \(c(x)\) have a common root.
\end{proposition}
\begin{proof}
If we expand out the expression for \(b(x)\) into a sum of monomials, with coefficients being expressed in terms of these \(\beta_i\), and similarly for \(c(x)\), then the resultant is a huge polynomial expression in terms of the various \(\beta_i\) and \(\gamma_j\).
This polynomial vanishes whenever \(\beta_i=\gamma_j\).
Thinking of \(\beta_i\) and \(\gamma_j\) as abstract variables, the expression \(\gamma_j-\beta_i\) is a linear function.
Assume for the moment that our field is infinite.
Then the resultant is divisible by \(\gamma_j-\beta_i\), for any \(i\) and \(j\), by lemma~\vref{lemma:linear.factor}.
Therefore the resultant is divisible by the product
\[
\pr{\gamma_1-\beta_1}\pr{\gamma_1-\beta_2}\dots\pr{\gamma_n-\beta_m}.
\]
Imagine expanding out the resultant as a determinant, with \(b(x),c(x)\) monic.
Each term in the determinant contains exactly \(n\) of the various \(b_j\), and each \(b_j\) has \(m-j\) \(\beta\)'s.
So the highest term in the \(\beta\)'s is the one from the diagonal.
The diagonal has \(b_0,b_0,\dots,b_0,1,1,\dots,1\), giving a term \(b_0^n\) in the determinant.
Expand in \(\beta\)'s as \(b_0=(-1)^m\beta_1\dots\beta_m\) to give a term \((-1)^{mn}\beta^n_1\dots\beta^n_m\) in the resultant.
But the product gives exactly the same highest term in \(\beta\)'s.
By swapping columns, we see that our product and resultant also give exactly the same highest term in \(\gamma\)'s, and so the resultant equals the product.
For a finite field, embed into an infinite field, as in chapter~\ref{chapter:fields}.
\end{proof}
\begin{example}
If \(b(x)\defeq x^7+2x+1\) and \(c(x)\defeq x-1\), then \(c(x)\) has root \(x=1\), and both are monic, so 
\[
\resultant{b}{c} = b(1) = 1^7+2+1=4.
\]
\end{example}
\begin{problem}{resultants:nonmonic}
Suppose that \(b(x)\) is monic.
Prove that 
\[
\resultant{b}{c} 
=(-1)^{mn}c\of{\beta_1}c\of{\beta_2}\dots c\of{\beta_m}.
\]
\end{problem}
\begin{lemma}\label{lemma:resultants.multiply}
For any polynomials \(b(x), c(x), d(x)\),
\[
\resultant{bd}{c}=\resultant{b}{c}\resultant{d}{c}.
\]
\end{lemma}
\begin{proof}
Split all of the polynomials into linear factors (over a splitting field):
\begin{align*}
\resultant{bd}{c}
&=
b\of{\gamma_1}d\of{\gamma_1}\dots b\of{\gamma_n}d\of{\gamma_n},
\\
&=b\of{\gamma_1}b\of{\gamma_2}\dots b\of{\gamma_n}
d\of{\gamma_1}d\of{\gamma_2}\dots d\of{\gamma_n},
\\
&=
\resultant{b}{c}\resultant{d}{c}.
\end{align*}
\end{proof}
\begin{problem}{resultants:add.stuff}
Suppose that \(c(x)\) is a monic polynomial.
Prove that 
\[
\resultant{b+dc}{c}=\resultant{b}{c}
\]
for any polynomials \(c(x)\) and \(d(x)\).
A surprise: note that the degree of \(b+dc\) could be either that of \(b\) or that of \(dc\), or something smaller than either, as any number of terms in \(b+dc\) might cancel one another.
But surprisingly, we don't need to know the degree of \(b+dc\).
\end{problem}
\begin{problem}{resultants:euclid.that.mofo}
A fast trick to find resultants: use the result of the previous problem to compute
\(\resultant{b}{c}\) where
\begin{align*}
b(x) &= x^8+4x^2+2x+1, \\
c(x) &= x^6+4.
\end{align*}
\end{problem}
\begin{answer}{resultants:euclid.that.mofo}
\[
b(x)=x^2 \, c(x) + 2x+1,
\]
so
\begin{align*}
\resultant{b(x)}{c(x)}
&=
\resultant{2x+1}{c(x)},
\\
&=
\resultant{2x+1}{x^6+4},
\\
&=
2^6
\resultant{x+\frac{1}{2}}{x^6+4},
\\
&=
2^6
\left.\pr{x^6+4}\right|_{x=-\frac{1}{2}},
\\
&=
2^6\pr{\frac{1}{2^6} + 4},
\\
&=
1+2^6 \cdot 4,
\\
&=257.
\end{align*}
\end{answer}

We can make sage code for our fast trick:
\begin{sageblock}
R.<x> = PolynomialRing(QQ)
def fast_resultant(b,c):
    m=b.degree()
    n=c.degree()
    if m<n:
        return (-1)^(m*n)*fast_resultant(c,b)
    if n==0:
        if m==0:
            return 1
        return c(0)^m
    lb=b.leading_coefficient()
    lc=c.leading_coefficient()
    B=b/lb
    C=c/lc
    r=B % C
    if (r==0):
        return 0
    return (-1)^(m*n)*lb^n*lc^m*fast_resultant(C,r)
\end{sageblock}
so that \verb!fast_resultant(x^3+x+7,x^2+2)! yields \(\sage{fast_resultant(x^3+x+7,x^2+2)}\).
We can run a speed test, 
\begin{verbatim}
b=sum([ZZ.random_element(0,100)*x^j for j in [0..100]])
c=sum([ZZ.random_element(0,100)*x^j for j in [0..100]])
print("The built in resultant function from sage:")
%timeit b.resultant(c)
print("Our slow resultant function:")
%timeit resultant(b,c)
print("Our fast resultant function:")
%timeit fast_resultant(b,c)
\end{verbatim}
Sage's built in routine easily wins, but we can see that the fast method is faster than the slow one:
\begin{verbatim}
The built in resultant function from sage:
4.48 ms ± 25.3 µs per loop (mean ± std. dev. of 7 runs, 100 loops each)
Our slow resultant function:
72.8 ms ± 171 µs per loop (mean ± std. dev. of 7 runs, 10 loops each)
Our fast resultant function:
53.2 ms ± 146 µs per loop (mean ± std. dev. of 7 runs, 10 loops each)
\end{verbatim}

\subsection{The discriminant}
For any polynomial \(p(x)\), with any sort of coefficients, say
\[
p(x) = a_0 + a_1 x + \dots + a_n x^n,
\] 
we define the \emph{derivative}\define{derivative}\Notation{pd(x)}{p'(x)}{derivative of polynomial}
\[
p'(x) = a_1 + 2a_2 x + 3 a_3 x^2 + \dots + n a_n x^n,
\]
just imitating calculus.
There is no interpretation of this as a ``rate of change'', since the coefficients could be remainders modulo some integer, or something much stranger.

The \emph{discriminant}\define{discriminant} \(\discriminant{p}\) of a polynomial \(p(x)\) is the resultant with the derivative: \(\discriminant{p} \defeq  \resultant{p}{p'}\).
By definition, the discriminant vanishes just when the polynomial has a common factor with its derivative.
For example, over the complex numbers, the discriminant vanishes just when the polynomial has a ``multiple root'', i.e. a repeated linear factor:
\[
p(z)=\pr{z-z_0}^2 \dots.
\]

\begin{problem}{resultants:discriminants}
Find the discriminants of (a) \(ax^2+b+c\), (b) \(x^3+x^2\), (c) \(x^3+x^2+1\), (d) \(x^3+2x-1\) and explain what they tell you about these polynomials.
\end{problem}
\begin{answer}{resultants:discriminants}
\begin{enumerate}
\item
The derivative of \(ax^2+bx+c\) is \(2ax+b\).
The discriminant of \(ax^2+bx+c\) is
\[
\det
\begin{pmatrix}
c & b & 0 \\
b & 2a & b \\
a & 0 & 2a
\end{pmatrix}
=
-a(b^2-4ac)
\]
which is surprisingly \emph{not} the usual expression.
So a quadratic polynomial has a common factor with its derivative just when \(b^2=4ac\).
\item
For \(p(x)=x^3+x^2\), \(p'(x)=3x^2+2x\), so
\[
\Delta_p
=
\det
\begin{pmatrix}
0 & 0 & 0 & 0 & 0 \\
0 & 0 & 2 & 0 & 0 \\
1 & 0 & 3 & 2 & 0 \\
1 & 1 & 0 & 3 & 2 \\
0 & 1 & 0 & 0 & 3
\end{pmatrix}
=0
\]
due to the row of zeroes, or to the double root at \(x=0\).
\item
For \(p(x)=x^3+x^2+1\), \(p'(x)=3x^2+2x\), so
\[
\Delta_p
=
\det
\begin{pmatrix}
1 & 0 & 0 & 0 & 0 \\
0 & 1 & 2 & 0 & 0 \\
1 & 0 & 3 & 2 & 0 \\
1 & 1 & 0 & 3 & 2 \\
0 & 1 & 0 & 0 & 3
\end{pmatrix}
=31
\]
so there is no common factor between \(p(x)\) and \(p'(x)\), and so no double roots among the complex numbers.
\item
For \(p(x)=x^3+2x-1\), \(p'(x)=3x^2+2\) so 
\[
\Delta_p
=
\det
\begin{pmatrix}
-1 &  0 & 2 & 0 & 0 \\
 2 & -1 & 0 & 2 & 0 \\
 0 &  2 & 3 & 0 & 2 \\
 3 &  0 & 0 & 3 & 0 \\
 0 &  3 & 0 & 0 & 3
\end{pmatrix}=27
\]
so there is no common factor between \(p(x)\) and \(p'(x)\), and so no double roots among the complex numbers.
\end{enumerate}
\end{answer}

\begin{example}
The polynomial \(f(x)=1+2x^2+x^4\) over the real numbers has no real roots.
Its derivative is \(f'(x)=4x+4x^3\), so its discriminant is
\[
\Delta_f =
\det
\begin{pmatrix}
1 & 0 & 0 & 0 & 0 & 0 & 0 \\
0 & 1 & 0 & 4 & 0 & 0 & 0 \\
2 & 0 & 1 & 0 & 4 & 0 & 0 \\
0 & 2 & 0 & 4 & 0 & 4 & 0 \\
1 & 0 & 2 & 0 & 4 & 0 & 4 \\
0 & 1 & 0 & 0 & 0 & 4 & 0 \\
0 & 0 & 1 & 0 & 0 & 0 & 4
\end{pmatrix}
=0
\]
(after a long calculation).
So \(f(x)\) has a common factor with \(f'(x)\).
We can see the common factor more clearly if we factor:
\begin{align*}
f(x) &= 1+2x^2+x^4 = \pr{1+x^2}^2, \\
f'(x) &= 4x+4x^3 = 4x\pr{1+x^2}.
\end{align*}
So \(f(x), f'(x)\) share a common factor of \(1+x^2\).
They don't share a root over the real numbers, but over the complex numbers
\begin{align*}
f(x) &= \pr{x-i}^2\pr{x+i}^2, \\
f'(x) &= 4x(x-i)(x+i),
\end{align*}
so they share roots at \(x=\pm i\), the double roots of \(f(x)\).
\end{example}
\begin{example}
The polynomial \(p(x)=x^3+2x^2+2x+2\) is irreducible over the rational numbers by Eisenstein's criterion, so has no rational roots.
It has odd degree so at least one real root.
There is no positive real root since all terms are positive.
The derivative is \(p'(x)=3x^2+4x+2\).
By the quadratic formula, the zeroes of \(p'(x)\) are at 
\[
x=-\frac{2}{3} \pm \frac{2 \sqrt{2}}{3} i,
\]
not real.
Since \(p'(x)\) has only positive coefficients, \(p'(x)>0\) for \(x>0\).
But \(p'(x)\) has no real zeroes, so \(p'(x)>0\) for all real values of \(x\).
Therefore \(p(x)\) is increasing, so has precisely one real root, irrational and negative.
The two remaining complex roots are nonreal conjugates.
The discriminant of \(p(x)\) is
\[
\Delta
=
\det
\begin{pmatrix}
2 & 0 & 2 & 0 & 0 \\
2 & 2 & 4 & 2 & 0 \\
2 & 2 & 3 & 4 & 2 \\
1 & 2 & 0 & 3 & 4 \\
0 & 1 & 0 & 0 & 3
\end{pmatrix}=-44.
\]
Since \(\Delta \ne 0\), we see that there are no double roots over the complex numbers, which is already clear.
\end{example}

\begin{problem}{resultants:cubic.discriminant.a}
Suppose that \(a\) is a real number.
Find the discriminant of \(q(x)=x^3+ax+1\).
For which real numbers \(a\) does \(q(x)\) have a multiple root?
What is the multiplicity of that root?
\end{problem}
\begin{answer}{resultants:cubic.discriminant.a}
The discriminant is the resultant of \(p(x),p'(x)\), so the determinant of
\[
\begin{pmatrix}
1&0&a&0&0\\
a&1&0&a&0\\
0&a&3&0&a\\
1&0&0&3&0\\
0&1&0&0&3
\end{pmatrix}
\]
which is \(27+4a^3\), so vanishes just when \(a=-3/\sqrt[3]{4}\).
For this value of \(a\), the polynomial actually factors:
\[
x^3-\frac{3}{\sqrt[3]{4}}x+1
=
(x-2^{-1/3})^2(x+2^{2/3}),
\]
so a double root at \(x=2^{-1/3}\) and a single root at \(x=-2^{2/3}\).
\end{answer}

\begin{problem}{resultants:cubic.discriminant}
Prove that the sign of the discriminant \(\Delta\) of a monic cubic polynomial tells us that the number of roots is:
\[
\begin{array}{@{}rl@{}}
\Delta > 0 & \text{\(3\) distinct roots} \\
\Delta = 0 & \text{a triple root or \(1\) real double root and \(1\) real single root} \\
\Delta < 0 & \text{\(1\) real and \(2\) complex roots}.
\end{array}
\]
\end{problem}
\begin{problem}{resultants:discriminant.as.product}
Prove that, for any monic polynomial \(p(x)\) with roots \(x_1,x_2,\dots,x_n\), 
\[
\Delta_{p(x)} = (-1)^{n(n-1)/2}\prod_{i<j}(x_i-x_j)^2.
\]
\end{problem}
\begin{answer}{resultants:discriminant.as.product}
Write out
\[
p(x)=(x-x_1)(x-x_2)\dots(x-x_n).
\]
Differentiate:
\begin{align*}
p'(x)
&=(x-x_2)(x-x_3)\dots(x-x_{n-1})(x-x_n)\\
&+(x-x_1)(x-x_3)\dots(x-x_{n-1})(x-x_n)\\
&+\vdots\\
&+(x-x_1)(x-x_2)(x-x_3)\dots(x-x_n)\\
&+(x-x_1)(x-x_2)(x-x_3)\dots(x-x_{n-1})
\end{align*}
Every term but the first has a factor of \(x-x_1\), so if we plug in \(x=x_1\):
\[
p'(x_1)=(x_1-x_2)(x_1-x_3)\dots(x_1-x_{n-1})(x_1-x_n)=\prod_{j \ne 1} (x_1-x_j).
\]
By the same reasoning, replacing \(x_1\) by any other root \(x_i\):
\[
p'(x_i)=\prod_{j \ne i} (x_i-x_j).
\]
So the discriminant is
\begin{align*}
\Delta_{p(x)}
&=
\resultant{p(x)}{p'(x)},
\\
&=
(-1)^{n(n-1)}\resultant{p'(x)}{p(x)},
\end{align*}
Note that either \(n\) is even or \(n-1\) is even, so \((-1)^{n(n-1)}=1\):
\begin{align*}
\Delta_{p(x)}
&=
\resultant{p'(x)}{p(x)},
\\
&=
\prod_i p'(x_i),
\\
&=
\prod_i \prod_{j\ne i}(x_i-x_j),
\\
&=
\prod_{i\ne j}(x_i-x_j).
\end{align*}
So now, for example, \(x_1-x_2\) occurs here, as does \(x_2-x_1\), so putting those together into one factor \(-(x_1-x_2)\):
\[
\Delta_{p(x)}
=
\prod_{i<j}(-1)(x_i-x_j)^2.
\]
There are \(n(n-1)/2)\) choices of \(i < j\), so finally
\[
\Delta_{p(x)}=(-1)^{n(n-1)/2}\prod_{i<j}(x_i-x_j)^2.
\]
\end{answer}

\section{Sage}
Sage computes discriminants:
\begin{sageblock}
R.<t> = PolynomialRing(QQ)
(t^2-t+7).discriminant()
\end{sageblock}
yields \(\sage{(t^2-t+7).discriminant()}\).

\section{Parameterisation}
\begin{example}
Take the curve in the plane parameterised by
\[
\pr{x(t),y(t)}=\pr{t^2-1,t\pr{t^2-1}}:
\]
\inputinexample{cubic-curve-1}
We want to find an equation for this curve, as a polynomial in \(x\) and \(y\).
\end{example}
\begin{proposition}\label{proposition:resultant.equation}
Draw a curve
\[
x(t)=\frac{p(t)}{q(t)}, y(t)=\frac{u(t)}{v(t)},
\]
where \(p(t), q(t), u(t), v(t)\) are polynomials with coefficients in a field \(k\).
The resultant in the variable \(t\) of the polynomials
\[
p(t)-xq(t), u(t)-yv(t)
\]
vanishes along the points \((x,y)\) in the plane which lie along this curve; we can allow the values of \(x\) and \(y\) to be in the algebraic closure \(\bar{k}\).
\end{proposition}
\begin{proof}
The resultant vanishes just at those points \((x,y)\) where the two polynomials have a common factor.
If there is a common factor of two polynomials, that factor has a root over \(\bar{k}\), so there is a common root \(t\) of both polynomials, in other words a value of \(t\) where \(x=p(t)/q(t)\) and \(y=u(t)/v(t)\).
On the other hand, a common root gives a common linear factor. 
So the resultant vanishes just along the image of the curve in the ``plane''. 
Careful: \(k=\R{}\) then this ``plane'' has points given as pairs \((x,y)\) of complex numbers, so it is not the complex plane, but two copies of the complex plane.
\end{proof}
\begin{example}
For the curve in the plane parameterised by
\[
\pr{x(t),y(t)}=\pr{t^2-1,t\pr{t^2-1}},
\]
we have to take the resultant of
\[
t^2-1-x, t\pr{t^2-1}-y,
\]
as a polynomial in \(t\); treat \(x\) and \(y\) as constants:
\[
\det 
\begin{pmatrix}
-(1+x) &        &        & -y \\ 
0      & -(1+x) & 0      & -1 & -y \\
1      & 0      & -(1+x) & 0 & -1 \\
       & -1     & 0      & 1 & 0 \\
       &        & -1     &   & 1
\end{pmatrix}=y^2-x^2-x^3
\]
So the curve is precisely the set of points \((x,y)\) so that \(y^2=x^2+x^3\).
\end{example}
\begin{example}
For real variables, the curve \(x=t^2, y=t^4\) has both \(x \ge 0\) and \(y \ge 0\) since \(t^2 \ge 0\) and \(t^4\ge 0\).
So the curve is the right half of the parabola \(y=x^2\) on which \(x \ge 0\).
The resultant comes out as
\[
\det
\begin{pmatrix}
-x &  0 &  0 &  0 & -y &  0 \\
 0 & -x &  0 &  0 &  0 & -y \\
 1 &  0 & -x &  0 &  0 &  0 \\
 0 &  1 &  0 & -x &  0 &  0 \\
 0 &  0 &  1 &  0 &  1 &  0 \\
 0 &  0 &  0 &  1 &  0 &  1  
\end{pmatrix}
=\pr{x^2-y}^2.
\]
So the resultant vanishes along \(y=x^2\), the whole parabola.
So we don't quite get what we wanted: we get the algebraic equation \(y=x^2\) that that curve satisfies, but we miss out on the inequality \(x \ge 0\).
\end{example}
\begin{problem}{resultants:eliminate}
Find a nonconstant polynomial equation \(f(x,y)=0\) satisfied by the points of the curve 
\[
x(t)=t^2, y(t)=t^3-t.
\]
\end{problem}
\begin{answer}{resultants:eliminate}
Write these equations as polynomials in increasing powers of \(t\), with coefficients rational in \(x,y\):
\begin{align*}
0 &= -x + 0t + t^2, \\
0 &= -y -t + 0t^2 +t^3.
\end{align*}
The resultant is
\[
\det
\begin{pmatrix}
  -x &   0 &   0 &   -y &   0 \\
   0 &  -x &   0 &   -1 &  -y \\
   1 &   0 &  -x &    0 &  -1 \\
   0 &   1 &   0 &    1  &  0 \\
   0 &   0 &   1 &    0  &  1
\end{pmatrix}
\]
You can simplify this determinant by adding \(x(\text{row 3})\) to row 1, and similar tricks, to compute it out:
\(
y^2-(x-1)x(x+1)
\).
So the equation of the curve is \(y^2=(x-1)x(x+1)\).
\end{answer}
\begin{problem}{resultants:eliminate.2}
Find a nonconstant polynomial equation \(f(x,y)=0\) satisfied by the points of the curve 
\[
x(t)=t^2+t, y(t)=\frac{1}{t}.
\]
\end{problem}
\begin{problem}{resultants:eliminate.3}
Find a nonconstant polynomial equation \(f(x,y)=0\) satisfied by the points of the curve 
\[
x(t)=t^2+1, y(t)=t+\frac{1}{t}.
\]
\end{problem}
\begin{answer}{resultants:eliminate.3}\(x^2+(1-x)y^2\)\end{answer}
\begin{problem}{resultants:eliminate.4}
With coefficients over the field of remainders modulo \(2\), find a polynomial equation \(0=p(x,y)\), with \(p(x,y)\) not constant, satisfied by the parameterized plane curve \((x(t),y(t))=(t^3+t,1/t)\).
\end{problem}
\begin{answer}{resultants:eliminate.4}
The equations, in powers of \(t\), are
\begin{align*}
0 &= x+t+t^3, \\
0 &= 1+yt.
\end{align*}
The resultant is
\begin{align*}
\begin{vmatrix}
x & 1 & 0 & 0 \\
1 & y & 1 & 0 \\
0 & 0 & y & 1 \\
1 & 0 & 0 & y
\end{vmatrix}
=xy^3+y^2+1.
\end{align*}
So \(p(x,y)=xy^3+y^2+1\).
\end{answer}

\section{Sage}
Sage handles \(2\) variable polynomials:
\begin{sageblock}
P.<x,y> = PolynomialRing(QQ)
a = x + y
b = x^3 - y^3
a.resultant(b)
\end{sageblock}
yields \(\sage{a.resultant(b)}\).
If we want to use the resultant in the other variable, we specify which variable to get rid of: \verb!a.resultant(b, y)! yields \(\sage{a.resultant(b, y)}\).
Sage will take a parameterisation of a curve and give us the equation of the curve.
Our example above becomes:
\begin{sageblock}
R.<x,y,t> = PolynomialRing(QQ)
(t^2-1-x).resultant(t*(t^2-1)-y,t)
\end{sageblock}
yields \(\sage{(t^2-1-x).resultant(t*(t^2-1)-y,t)}\).

Sage computes resultants of polynomials over any field.
\begin{example}
Let \(K\) be the splitting field of \(x^2+x+1\) over \(k=\Zmod{2}\), with generator \(a\).
We compute the resultant and greatest common divisor of
\[
p(t) = (at+1)^2 (at-1)^2, \\
q(t) = (at+1) (at+a^2)^2.
\]
via
\begin{sageblock}
R.<x> = PolynomialRing(GF(2))
K.<a> = (x^2 + x + 1).splitting_field()
S.<t> = PolynomialRing(K)
p = S((a*t+1)^2*(a*t-1)^2)
q = S((a*t+1)*(a*t+a^2)^2)
print(p.gcd(q))
print(p.resultant(q))
\end{sageblock}
yielding \(\sage{p.gcd(q)}\) and \(\sage{p.resultant(q)}\).
\end{example}