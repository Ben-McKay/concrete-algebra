\chapter{Volumes in higher dimensions}

\optionalSection{Framings}

A collection of linearly independent vector fields on a manifold \(M\) forming a basis of \(T_p M\) at each point \(p\) is called a \emph{framing}.%
\define{framing}
In \(\R{n}\), the standard basis vector fields \(e_1, e_2, \dots, e_n\) are a framing.
\begin{marginfigure}
\centering
\setlength\fboxsep{0pt}
\setlength\fboxrule{0.5pt}
%\fbox{%
\includegraphics[width=3cm,trim=20mm 0mm 20mm 0mm,clip]{framing-sphere}%
%}%
\legend{A framing of an open subset of the sphere}
\end{marginfigure}

For each point of any manifold, there is a framing defined on a open set around that point: pick a chart \(h\) and use \(h_*\) to take the standard basis vector fields to a framing on an open subset of \(M\).

Take any framing \(X_1, X_2, \dots, X_n\) on an open set in \(\R{n}\).
Write each vector field in the standard basis, say
\[
X_1 = \sum a_{1j} e_j, X_2 = \sum a_{2j} e_j, \dots, X_n = \sum a_{nj} e_j.
\]
Let \(A=\left(a_{ij}\right)\).
A framing on \(\R{n}\) is \emph{positively oriented} if \(\det A > 0\),\SubIndex{determinant} and \emph{negatively oriented} otherwise.
We can swap the sign of any one of the vector fields in a framing, and the framing will change from positively oriented to negatively oriented or vice versa.

On an oriented manifold \(M\) of dimension \(p\), a framing is \emph{positively oriented}% 
\define{framing!positively oriented}
\define{positively oriented!framing}
if it is locally identified with a positively oriented framing on an open set in \(\R{p}\) by some positively oriented chart.
A basis \(v_1, v_2, \dots, v_n\) for a tangent space \(T_x M\) is \emph{positively oriented}%
\define{basis!positively oriented} 
\define{positively oriented!basis}
if it extends somehow to a positively oriented framing in some open set.

Once we have a framing, we can apply Gram--Schmidt orthogonalization to it to produce an orthonormal framing.
\begin{problem}{volume:Gram.Schmidt}
If \(X_1, X_2, \dots, X_n\) is a framing of continuously differentiable vector fields, and \(Y_1, Y_2, \dots, Y_n\) are the result of applying Gram--Schmidt orthogonalization to \(X_1, X_2, \dots, X_n\), prove that \(Y_1, Y_2, \dots, Y_n\) are continuously differentiable.
\end{problem}
Near each point on an oriented manifold, there is positively oriented orthonormal framing.
Any two framings, say \(X_1, X_2, \dots, X_n\) and \(Y_1, Y_2, \dots, Y_n\) are expressible in terms of one another, say as \(Y_i = \sum_j a_{ji} X_j\), for some functions \(a_{ij}\). 
For any two positively oriented orthonormal framings, the matrix \(A\) is a rotation matrix.

If \(X_1, X_2, \dots, X_n\) is a framing, the \emph{dual coframing}%
\define{coframing!dual}
\(\omega_1, \omega_2, \dots, \omega_n\) are the 1-forms so that 
\[
\omega_i\of{X_j}=
\begin{cases}
1 & \text{ if \(i=j\)}, \\
0 & \text{ otherwise.}
\end{cases}
\]
Clearly each framing has a unique dual coframing, and vice versa.

\begin{problem}{volume:orthonormality}
A coframing \(\omega_1, \omega_2, \dots, \omega_n\) is orthonormal (in other words, for any tangent vector \(v\), \(\sum \omega_j(v)^2=\norm{v}^2\)) just when its dual framing is orthonormal. 
\end{problem}
\begin{answer}{volume:orthonormality}
Take the dual framing \(X_1, X_2, \dots, X_n\), and write out your vector \(v\) as a linear combination, say \(v=\sum a_i X_i\).
Then \(\omega_i(v)=a_i\) so \(\sum_i \omega_i(v)^2=\sum a_i^2\).
If \(X_i\) are orthonormal, then \(\ip{v}{X_i}=a_i\) and \(\norm{v}^2=\ip{v}{v}=\sum_i a_i \ip{v}{X_i} = \sum_i a_i a_i = \sum_i \omega_i(v)^2\).
On the other hand, if \(\sum_i \omega_i(v)^2=\norm{v}^2\) for all \(v\), then taking \(v=X_j\) we find \(\norm{X_j}=1\), and taking \(v=X_i \pm X_j\) for \(i \ne j\): \(\norm{X_i \pm X_j}^2=2=\norm{X_i}^2+\norm{X_j}^2\pm 2\ip{X_i}{X_j}=2 \pm 2 \ip{X_i}{X_j}\), so \(\ip{X_i}{X_j}=0\).
\end{answer}

The framing is \emph{positively oriented}\define{coframing!positively oriented}\define{positively oriented!coframing}
if its dual framing is.
Coframings will be more useful to us than framings.

Take two framings, say \(X_1, X_2, \dots, X_p\) and \(Y_1, Y_2, \dots, Y_p\), with dual coframings \(\omega_1, \omega_2, \dots, \omega_p\) and \(\eta_1, \eta_2, \dots, \eta_p\).
\begin{problem}{volume:change.coframing}
If we change framings, say to \(Y_i = \sum a_{ji} X_j\), we change dual coframings as \(\omega_i = \sum_j a_{ij} \eta_j\) and wedge products as
\[
\omega_1 \wedge \omega_2 \wedge \dots \wedge \omega_p 
=
\det A \eta_1 \wedge \eta_2 \wedge \dots \wedge \eta_p
\]
where \(A=\pr{a_{ij}}\).
\end{problem}




\optionalSection{Normal vector fields}

Why did we say that an orientable surface was one which has ``two sides''? 
A \emph{normal vector}% 
\define{normal vector}
to a manifold \(M \subset \R{n}\) at a point \(x \in M\) is a vector \(v \in \R{n}\) which is perpendicular to \(M\) at \(x\), by which we mean perpendicular to every tangent vector in \(T_x M\), like \input{normal-vector-to-circle}.
A \emph{normal vector field}% 
\define{normal vector!field}
on a manifold \(M\) is a continuous map \(X \colon M \to \R{n}\) so that \(X(x)\) is a normal vector, for each \(x \in M\).
It is a \emph{unit} normal vector field if \(\norm{X}=1\) at every point.
\begin{center}
\includegraphics[width=.6\linewidth]{moebius-strip-normal}
\legend{We try, but fail, to put a unit normal vector field on the M\"obius strip.}%
\SubIndex{Moebius strip@M\"obius strip}
\end{center}

\begin{marginfigure}
\centering
\input{sphere-of-unit-vectors}
\end{marginfigure}

\begin{problem}{volumes:normal.sphere}
It is standard to write \(S^n \subset \R{n+1}\)%
\Notation{Sn}{S^n}{the \(n\)-dimensional unit sphere about \(0\) in \(\R{n+1}\)} to mean the unit sphere around the origin in \(\R{n+1}\). 
(Careful: the superscript \({}^{n-1}\) means that the sphere has dimension \(n-1\).
It doesn't mean \((n-1)\)-fold product of things called \(S\).)
Prove that on the unit sphere \(S^n \subset \R{n+1}\), the vector field \(X(x)=x\) is a normal vector field.
\end{problem}

A manifold \(M \subset \R{n+1}\) of dimension \(n\) (in other words, of one less dimension than the ambient space) is a \emph{hypersurface}.%
\define{hypersurface}

\begin{lemma}
For any hypersurface \(M \subset \R{n+1}\), and any point \(x \in M\), there is a collection \(X_1, X_2, \dots, X_{n+1}\) of vector fields defined in an open subset of \(M\) around \(x\), orthonormal and with \(X_1\) normal and \(X_2, X_3, \dots, X_{n+1}\) tangent.
If \(M\) is oriented, we can arrange that \(X_1, X_2, X_3, \dots, X_{n+1}\) is positively oriented (as a basis of \(\R{n+1}\)) at each point, and that \(X_2, X_3, \dots, X_{n+1}\) is positively oriented (as a framing of \(M\)).
\end{lemma}
\begin{proof}
At least one of the vector fields \(e_1, e_2, \dots e_{n+1}\) is not tangent to \(M\) at or near \(x\); they can't all be tangent, since the tangent space has dimension \(n\).
If not tangent at \(x\), then by continuity not tangent near \(x\).
So we can let \(X_1\) be some vector field not tangent to \(M\) near \(x\).
Along \(M\) pick an orthonormal framing \(X_2, X_3, \dots, X_{n+1}\) (defined only along \(M\), not everywhere near \(M\)).
Following Gram--Schmidt, replace \(X_1\) by
\[
X_1 - \ip{X_1}{X_2} X_2 - \ip{X_1}{X_3} X_3 - \dots - \ip{X_1}{X_{n+1}}X_{n+1}.
\]
so that we can assume that \(X_1\) is perpendicular to \(X_2, X_3, \dots, X_{n+1}\).
Replace \(X_1\) by \(X_1/\norm{X_1}\) to arrange that \(X_1\) in unit length.
If \(M\) is oriented, we could already pick \(X_2, X_3, \dots, X_{n+1}\) a positively oriented orthonormal framing, and replace \(X_1\) by \(-X_1\) if needed to arrange that \(X_1, X_2, \dots, X_{n+1}\) is a positively oriented basis of \(\R{n+1}\) at each point of \(M\) where it is defined.
Smoothly extend these vector fields any way at all to be defined nearby.
\end{proof}

\begin{theorem}
A hypersurface is orientable just when it admits a continuous nowhere vanishing normal
vector field. 
This occurs just when it admits a continuously differentiable unit normal vector field, which is then unique up to \(\pm\).
For each orientation there is a unique unit normal vector field \(\nu\) so that, for any positively oriented framing \(X_2, X_3, \dots, X_{n+1}\) on an open subset of \(M\), \(\nu, X_2, X_3, \dots, X_{n+1}\) is positively oriented.
Conversely, every unit normal vector field \(\nu\) determines a unique orientation of \(M\) by the same requirement.
\end{theorem}
\begin{proof}
Suppose that \(M\) is orientable.
As above, cover \(M\) by open sets on each of which we have a positively oriented orthonormal framing \(X_2, X_3, \dots, X_{n+1}\) with associated unit normal field \(X_1\) so that \(X_1, X_2, \dots, X_{n+1}\) is positively oriented.
Suppose that we have two such: \(X_1, X_2, X_3, \dots, X_{n+1}\) and \(Y_1, Y_2, Y_3, \dots, Y_{n+1}\).
Any two positively oriented orthonormal bases are obtained one from the other by a rotation of \(\R{n+1}\).
That rotation preserves the tangent space \(T_x M\), since \(T_x M\) is the span of \(X_2, X_3, \dots, X_{n+1}\) and also of \(Y_2, Y_3, \dots, Y_{n+1}\).
Because these are positively oriented orthonormal bases for \(T_x M\), they are just rotations of one another, so the rotation of \(\R{n+1}\) splits into a rotation of \(T_x M\) and a rotation of the normal vectors.
But the normal vectors lie in a 1-dimensional normal space.
The only rotation of a 1-dimensional space is the identity transformation.
Therefore \(X_1=Y_1\), so we can define \(\nu=X_1\) where \(X_1\) is defined, and define \(\nu=Y_1\) where \(Y_1\) is defined.

Suppose on the other hand that \(M\) admits a nonvanishing normal vector field \(X\).
Replace \(X\) by \(\nu=X/\norm{X}\) to arrange a unit normal vector field.
Pick any orthonormal framing \(X_2, X_3, \dots, X_{n+1}\) on \(M\) and change the sign of any one of the vector fields in the framing if needed to arrange that \(\nu, X_2, \dots, X_{n+1}\) is positively oriented.
If we take any two such orthonormal framings, each is a rotation of the other, by a rotation fixing \(\nu\).
Hence they are positively oriented with respect to one another.
Consider all charts for which these framings are positively oriented in those charts; these charts are an orientation of \(M\).
\end{proof}

\begin{problem}{volume:find.normal}
Suppose that \(M \subset \R{n}\) is the set of zeroes of a function:
\[
M = \Set{x \in \R{n}|f(x)=0}
\]
where \(f \colon \op{\R{n}} \to \R{}\) is continuously differentiable and the gradient doesn't vanish: \(\nabla f \ne 0\) at every point of \(M\). 
Prove that \(M\) is a manifold and that \(\nu = \frac{\nabla f}{\norm{\nabla f}}\) is a unit normal vector field to \(M\).
\end{problem}

\begin{problem}{volume:find.normal.ellipsoid}
Use the result of problem~\ref{problem:volume:find.normal} to find the unit normal vector field to the ellipsoid
\[
\sum_i \left(\frac{x_i}{a_i}\right)^2 = 1
\]
in \(\R{n}\), where \(a_1, a_2, \dots, a_n > 0\) are constants.
\end{problem}

\begin{problem}{volume:find.normal.graph}
Suppose that \(M \subset \R{n+1}\) is the graph of a function:
\[
M = \Set{(x,y) \in \R{n+1}|y=f(x)}
\]
where \(f \colon \op{\R{n}} \to \R{}\) is continuously differentiable.
Prove that \(M\) is a manifold and that the vector field
\[
\nu = \frac{1}{\sqrt{1+\sum_i \left(\pd{f}{x_i}\right)^2}}
\begin{pmatrix}[\tallmatrix]
\pd{f}{x_1} \\
\pd{f}{x_2} \\
\vdots \\
\pd{f}{x_{n}} \\
-1
\end{pmatrix}
\]
is a unit normal vector field to \(M\).
\end{problem}

\begin{problem*}{volume:oriented.covering.space}
Suppose that \(M \subset \R{n+1}\) is an \(n\)-dimensional manifold. 
Let \(M_1 \subset \R{2n}\) be the set of pairs \((x,v)\) where \(x \in M\) and \(v\) is a unit length vector normal to \(M\) at \(x\). 
Prove that \(M_1\) is an \(n\)-dimensional manifold. 
Define an orientation on \(M_1\). 
Prove that the map \((x,v) \in M_1 \mapsto x \in M\) is a 2-1 local diffeomorphism.
If \(M\) is connected, prove that either (1) \(M\) is unorientable and \(M_1\) is connected or (2) \(M\) is orientable and \(M_1\) is not connected.
\end{problem*}

\optionalSection{The volume form on an oriented hypersurface}

\begin{problem}{volume:hypersurf}
If \(M\) is an oriented hypersurface, say with unit normal vector field \(\nu\), prove that
\[
dV_M = \nu \hook dV_{\R{n+1}}.
\]
\end{problem}
\begin{answer}{volume:hypersurf}
Take an orthonormal coframing, which we write as \(\omega_2, \omega_3, \dots, \omega_{n+1}\) on \(M\) near some point of \(M\).
Take the dual orthonormal framing \(X_2, X_2, \dots, X_{n+1}\) and let \(\nu\) be the unit normal.
Then extend the orthonormal vector fields \(\nu, X_2, X_3, \dots, X_{n+1}\) to some vector fields defined in an open subset of \(\R{n}\) near our point of \(M\), and apply Gram--Schmidt orthogonalization to them.
Since they were already orthonormal along \(M\), their values at each point of \(M\) don't change, but they might change away from \(M\).
Now make the dual orthonormal coframing, say \(\omega_1, \omega_2, \dots, \omega_{n+1}\).
Then
\[
dV_{\R{n+1}} = \omega_1 \wedge \omega_2 \wedge \dots \wedge \omega_{n+1},
\]
while 
\[
\nu \hook dV_{\R{n+1}} = \omega_2 \wedge \dots \wedge \omega_{n+1} = dV_M.
\]
\end{answer}

\begin{example} 
On the unit sphere \(S^{n-1}\), we saw the volume form already: \(dV_{S^{n-1}}=\Omega\), the solid angle form restricted to the unit sphere. 
Our polar coordinate formula is thus 
\[
∫_{\R{n}} f(x) \, dV = 
∫_0^{\infty} \left(∫_{S^{n-1}} f(ru) dV_{S^{n-1}} \right) r^{n-1} \, dr.
\]
For example, a very important integral:
\[
∫_{\R{n}} e^{-\norm{x}^2} \, dV
=
\area{S^{n-1}} ∫_0^{\infty} e^{-r^2} r^{n-1} \, dr.
\]
Another way to calculate that integral: write each vector \(x \in \R{n}\) as \(x=(s,t)\), for some \(s \in \R{2}\) and \(t \in \R{n-2}\).
Clearly
\[
\norm{x}^2=\norm{s}^2+\norm{t}^2,
\]
so
\[
∫_{\R{n}} e^{-\norm{x}^2} \, dV
=
∫_{\R{2}}  e^{-\norm{s}^2} \,
∫_{\R{n-2}} e^{-\norm{t}^2}
\]
giving a relationship between areas of spheres, which we can use inductively to find the 
area of a sphere in any dimension from the length of the unit circle.
We leave the reader to prove:
\[
\area{S^n}
=
\begin{cases}
\frac%%
{%
\left(2 \pi\right)^{n/2}
}%
{%
2 \cdot 4 \cdot \dots \cdot (n-2)
}%
, & \text{ if \(n\) is even}, \\
\frac%%
{%
2 \left(2 \pi\right)^{(n-1)/2}
}%
{%
1 \cdot 3 \cdot \dots \cdot (n-2)
}%
, & \text{ if \(n\) is odd}.
\end{cases}
\]
\end{example}

\begin{problem}{volume:ellipsoid.volume}
Use the result of problem~\vref{problem:volume:find.normal.ellipsoid}
to find the volume form of the ellipsoid
\[
\sum_i \left(\frac{x_i}{a_i}\right)^2 = 1
\]
in \(\R{n}\), where \(a_1, a_2, \dots, a_n > 0\) are constants.
\end{problem}
\begin{answer}{volume:ellipsoid.volume}
\[
dV_{\text{ellipsoid}}
=
\frac{1}{\sqrt{\sum_i \left(\frac{x_i}{a_i}\right)^2}}
\sum_j \frac{x_j}{a_j}
dx_1 \wedge dx_2 \wedge \dots \wedge \widehat{dx}_j \wedge
\dots \wedge dx_n.
\]
\end{answer}

\begin{problem}{volume:surface}
Suppose that \(M \subset \R{3}\) is an oriented surface, with unit normal vector
\[
\nu
=
\begin{pmatrix}
\nu_x \\
\nu_y \\
\nu_z
\end{pmatrix}.
\]
Prove that the area form of \(M\) is
\[
dA_M = \nu_x dy \wedge dz + \nu_y dz \wedge dx + \nu_z dx \wedge dy.
\]
\end{problem}

\begin{problem}{volume:product}
Prove that \(\vol{M \times N}=\vol{M} \vol{N}\).
\end{problem}



\optionalSection{Tricks to find lengths, areas and volumes}

\begin{lemma}
If \(X\) is a vector field defined near an oriented hypersurface \(M\) then
\[
X \hook dV_{\R{n}} = \ip{X}{\nu} dV_M
\]
\end{lemma}
\begin{proof}
Along any oriented hypersurface \(M\) we can split any vector field \(X\) into 
\[
X = \ip{X}{\nu} \nu + Y
\]
where \(Y\) is the orthogonal projection of \(X\) to the tangent plane to \(M\), so \(Y\) is a tangent vector field to \(M\).
By definition \(\nu \hook dV_{\R{n}} = dV_M\), so
\begin{align*}
X \hook dV_{\R{n}} 
&= 
\ip{X}{\nu} \nu \hook dV_{\R{n}} + Y \hook dV_{\R{n}},
\\
&=
\ip{X}{\nu} dV_M + Y \hook dV_{\R{n}}.
\end{align*}
On \(M\), \(0=Y \hook dV_{\R{n}}\) since this \((n-1)\)-form eats a basis of tangent vector fields on \(M\), but then \(Y\) is already tangent, giving \(n\) tangent vector fields on an \((n-1)\)-dimensional manifold, so not linearly independent, so giving zero when plugged into \(dV_{\R{n}}\).
\end{proof}

\begin{lemma}
Suppose that \(h \colon U \to M\) is a positively oriented chart for an oriented manifold \(M\), perhaps with boundary and corners. 
If we write \(h\) as \(x=h(u)\), say for \(u \in U\), then\SubIndex{determinant}
\[
h^* dV_M = 
\sqrt{\det G} \, dV_{\R{n}}
\]
where \(G\) is the matrix
\[
G = 
\begin{pmatrix}[\tallmatrix]
\left<\pd{x}{u_1},\pd{x}{u_1}\right> & 
\left<\pd{x}{u_1},\pd{x}{u_2}\right> & 
\dots &
\left<\pd{x}{u_1},\pd{x}{u_n}\right> \\
\left<\pd{x}{u_2},\pd{x}{u_1}\right> & 
\left<\pd{x}{u_2},\pd{x}{u_2}\right> & 
\dots &
\left<\pd{x}{u_2},\pd{x}{u_n}\right> \\
\vdots &
\vdots &
\vdots &
\vdots \\
\left<\pd{x}{u_n},\pd{x}{u_1}\right> & 
\left<\pd{x}{u_n},\pd{x}{u_2}\right> & 
\dots &
\left<\pd{x}{u_n},\pd{x}{u_n}\right>
\end{pmatrix}.
\]
\end{lemma}
\begin{proof}
Let \(Y_i = \pd{h}{u_i}\). 
So \(Y_1, Y_2, \dots, Y_n\) is a basis for \(T_m M\) at each \(m \in M\), if \(M\) is \(n\)-dimensional. 
Apply the Gram--Schmidt process to get an orthonormal basis, say \(X_1, X_2, \dots, X_n\),
for \(T_m M\). 
Somehow write the \(Y_i\) as a linear combination of the \(X_i\):
\[
Y_i = \sum_j a_{ij} X_j.
\]
Let \(A = \left(a_{ij}\right)\), so
\[
ω\left(Y_1,Y_2,\dots,Y_n\right)=\det A.
\]
Take inner products:
\begin{align*}
\left<Y_i,Y_j\right>
&=
\left<\sum_k a_{ik} X_k,  \sum_{\ell} a_{j\ell} X_{\ell}\right>,
\\
&=
\sum_k a_{ik} a_{k\ell},
\\
&=
\left(AA^t\right)_{ij}.
\end{align*}
So \(G = AA^t\) and \(G\) is a positive definite symmetric matrix with \(\det G = \det A \det A^t\). 
But \(\det A^t = \det A\), so 
\begin{align*}
\sqrt{\det G} &= \det A,
\\
&= ω\left(Y_1,Y_2,\dots,Y_n\right),
\\
&=
ω\left(h_* e_1, h_* e_2, \dots, h_* e_n\right),
\\
&=
h^* ω \left(e_1, e_2, \dots, e_n\right).
\end{align*}
\end{proof}

\begin{problem}{intrinsic:length}
Use this to find an integral expression for the length of the curve \(y=f(x)\) in the plane \(\R{2}\) for \(f \colon [a,b] \to \R{}\) a function, \(a < b \in \R{}\).
\end{problem}


\begin{example} 
Take a curve \(M\) in the \((x,y)\)-plane and a chart \(h(t)=\left(x(t),y(t)\right)\):
\begin{align*}
G 
&= 
\begin{pmatrix}
\left<\pd{(x,y)}{t},\pd{(x,y)}{t}\right>
\end{pmatrix},
\\
&=
\left(\pd{x}{t}\right)^2 + \left(\pd{y}{t}\right)^2,
\end{align*}
so that
\[
h^* ds = 
\sqrt{ \left(\pd{x}{t}\right)^2 + \left(\pd{y}{t}\right)^2 } \, dt,
\]
and the length of the curve is
\[
\bigints \sqrt{ \left(\pd{x}{t}\right)^2 + \left(\pd{y}{t}\right)^2 } \, dt,
\]
the usual formula for length of a plane curve.
\end{example}



